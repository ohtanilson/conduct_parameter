\documentclass[11pt, a4paper]{article}
\usepackage[utf8]{inputenc}
\usepackage{amsmath,setspace,geometry}
\usepackage{amsthm}
\usepackage{amsfonts}
\usepackage{mathtools}
\mathtoolsset{showonlyrefs}
\usepackage[shortlabels]{enumitem}
\usepackage{rotating}
\usepackage{pdflscape}
\usepackage{graphicx}
\usepackage{bbm}
\usepackage[dvipsnames]{xcolor}
\usepackage[colorlinks=true, linkcolor= RawSienna, citecolor = RawSienna, filecolor = RawSienna, urlcolor = RawSienna, hypertexnames = true, backref = page]{hyperref}
\usepackage[]{natbib} 
\bibpunct[:]{(}{)}{,}{a}{}{,}
\geometry{left = 1.0in,right = 1.0in,top = 1.0in,bottom = 1.0in}
\usepackage[english]{babel}
\usepackage{float}
\usepackage{caption}
\usepackage{subcaption}
\usepackage{tikz}
\usepackage{booktabs}
\usepackage{pdfpages}
\usepackage{threeparttable}
\usepackage{framed}
\usepackage{comment}
\usepackage{lscape}
\usepackage{bm}
\setstretch{1.4}
%\usepackage[tablesfirst,nolists]{endfloat}

\usepackage[T1]{fontenc}
\usepackage{mlmodern}  % 太いComputer Modern
% MLmodernのバグを修正: cf. https://tex.stackexchange.com/questions/646333/size-of-integral-symbol-in-section-header-with-mlmodern
\DeclareFontFamily{OMX}{mlmex}{}
\DeclareFontShape{OMX}{mlmex}{m}{n}{<->mlmex10}{} 
\usepackage{tgtermes} % 数式以外の欧文をTXフォントで上書き

\newtheorem{theorem}{Theorem}
\newtheorem{assumption}{Assumption}
\newtheorem{lemma}{Lemma}
\newtheorem{definition}{Definition}
\newtheorem{proposition}{Proposition}
\newtheorem{claim}{Claim}
\newtheorem{corollary}{Corollary}
\newtheorem{example}{Example}
\DeclareMathOperator{\rank}{rank}

\theoremstyle{remark}
\newtheorem{remark}{Remark}

\title{Note on the identification of conduct parameters in homogeneous goods markets}
\author{Yuri Matsumura\thanks{\href{mailto:}{yuri.matsumura23@gmail.com}, Department of Economics, Rice University.} \and Suguru Otani \thanks{\href{mailto:}{suguru.otani@e.u-tokyo.ac.jp}, Market Design Center, Department of Economics, University of Tokyo
\\Declarations of interest: none %this is for Economics Letters
}}

\begin{document}

\maketitle
\begin{abstract}
    In this note, we revisit the identification result of conduct parameters in homogeneous goods markets by \citet{lau1982identifying}.
    The result is that the conduct parameter cannot be identified if and only if the demand function is separable but not a specific functional form.
    We show that the result is incorrect by providing a separable demand function that induces identification.
    This implies that the class of inverse demand functions that lead to identification is broader than \citet{lau1982identifying} considers.
    Therefore, when the market has enough variation, the conduct parameter can be identified in the broader class of inverse demand functions.
\end{abstract}

\noindent\textbf{Keywords:} Conduct parameters, Homogenous Goods Market
\vspace{0in}
\newline
\noindent\textbf{JEL Codes:} C5, C13, L1

\bigskip





\section{Introduction}
Measuring competitiveness is an important task in the empirical industrial organization literature.
The conduct parameter approach is one of the useful approaches to measure competitiveness.
However, the parameter cannot be directly measured from data because data generally lack information about marginal costs.
Therefore, researchers endeavor to identify and estimate conduct parameters.

In homogeneous goods markets, \cite{bresnahan1982oligopoly} considers the identification of the conduct parameter and the marginal cost function in linear demand and linear marginal cost model.
\citet{bresnahan1982oligopoly} finds that when the researcher can obtain a demand shifter, called a demand rotation instrument, that changes the slope and intercept of the inverse demand function, the conduct parameter and the marginal cost parameter can be identified.
Recently, \citet{matsumura2023resolving} provide a detailed condition for the identification.
\citet{lau1982identifying} considers a more general setting and shows that the conduct parameter is not identified if and only if the demand function is separable, except a specific functional form.

However, we show that the identification result in \citet{lau1982identifying} is incorrect by providing a counterexample where a separable demand function induces identification.
While the proof of \citet{lau1982identifying} is incorrect, we don't need to worry because the counterexample shows that the class of inverse demand functions that lead to identification is broader than the class considered in \citet{lau1982identifying}.
Thus, when the market has enough variation, the conduct parameter can be identified in a broader class of inverse demand functions.


\section{Model}
Consider a homogeneous product market with the aggregate inverse demand and aggregate marginal cost function as $P(Q, X^{d})$ and $MC(Q, X^{s})$ where $X^{d}$ and $X^{s}$ are the vector of exogenous variables.
Assume that $X^{d}$ and $X^{s}$ are exclusive; that is, there is no common variable in $X^{d}$ and $X^{s}$.
Thus, $X^{d}$ works as demand shifters, and $X^{s}$ works as cost shifters. 
Let $K_d$ and $K_s$ be the number of demand shifters and cost shifters, respectively.
The equilibrium quantity is characterized by the first-order condition
 \begin{align}
    P(Q, X^{d}) + \theta \frac{\partial P}{\partial Q}(Q, X^{d})Q &= MC(Q, X^{s}),
\end{align}
where $\theta$ is called the conduct parameter. 
Depending on the value of $\theta$, the relation can nest the first-order condition of several models, such as perfect competition ($\theta=0$) and Cournot competition ($\theta=1/N$). 
See the Appendix for the details of the interpretation.


Suppose that the researcher observes the aggregate price $P$ and the aggregate quantity $Q$, and the vector of exogenous variables $X^{d}$ and $X^{s}$.
In this setting, the researcher can independently identify the inverse demand function by using the demand and cost shifters.
Therefore, the researcher wants to identify the conduct parameter and the marginal cost function from the observed data.
\citet{lau1982identifying} considers the identification problem but takes an indirect approach and specifies the conditions under which the model is not identified. 
The definition of non-identification is as follows:

\begin{definition}\label{def:non_identification}
Non-identification implies for any $X^{d}$ and $X^{s}$,
\begin{align}
P(Q^e, X^{d}) + \theta \frac{\partial P}{\partial Q}(Q^e, X^{d})Q^e &= MC(Q^e, X^{s}),  \label{eq:foc_alpha}\\
P(Q^e, X^{d}) + \theta^{*} \frac{\partial P}{\partial Q}(Q^e, X^{d})Q^e &= MC^{*}(Q^e, X^{s}),\label{eq:foc_beta}
\end{align}
where $\theta \neq \theta^{*}$, $MC \ne MC^{*}$,\footnote{This condition is not stated in \citet{lau1982identifying}, but assuming $MC = MC^{*}$ makes the identification simple.} and the equilibrium quantity $Q^e$ has a reduced form functions $Q^e = h(X^{d}, X^{s})$ and $Q^e = h^{*}(X^{d}, X^{s})$ defined by \eqref{eq:foc_alpha} and \eqref{eq:foc_beta} respectively are identical.
\end{definition}

In other words, given two distinct pairs of the conduct parameter and the marginal cost, the equilibrium quantity $Q^e$ solves the first-order condition for both pairs.
Non-identification asks the following question: given an identified inverse demand function, is it possible to find two distinct pairs of a conduct parameter and a marginal cost that lead to observable equivalent equilibrium?
\citet{lau1982identifying} finds that the separability of the inverse demand function and non-identification are equivalent.

\begin{theorem}\label{theorem_lau}
    Under the assumption that the industry inverse demand and cost functions are twice continuously differentiable, the index of competitiveness $\theta$ cannot be identified from data on industry price and output and other exogenous variables alone if and only if the industry inverse demand function is separable in $X^{d}$, that is,
    \begin{align}
        p = P(Q, r(X^{d})), \label{eq:demand_separable}
    \end{align}
    but not take the form, 
    \begin{align}
        P = Q^{-1/\theta}r(X^{d}) + s(Q). \label{eq:identification_separable}
    \end{align}
\end{theorem}
The theorem implies that the conduct parameter is identified when the demand function is not separableor is separable given by \eqref{eq:identification_separable}.
The separable inverse demand \eqref{eq:demand_separable} has "weak separability" defined in \citet{goldmanNote1964}. 



\section{Counterexample to the sufficiency of separability}\label{sec:counterexample_sufficiency}
In this section, we show that the sufficiency does not hold; that is, the separable inverse demand function does not imply the nonidentification of the conduct parameter.


For the simplisity of the argument, we define the following function $F(Q, X^{d}, X^{s}; \theta, MC)$ as
\begin{align}
    F(Q, X^{d}, X^{s}; \theta, MC) \equiv P(Q, X^{d}) + \theta \frac{\partial P}{\partial Q}(Q, X^{d}) Q - MC(Q, X^{s}).
\end{align}
Note that by the first-order condition, we have at the equilibrium quantity $Q^e$ that
\begin{align}
    F(Q^e, X^{d}, X^{s}; \theta, MC) = 0.
\end{align}

Before proving the counterexample, we provide a lemma that characterizes the nonidentification.
\begin{lemma}\label{lemma:nonidentification_transformation}
    Nonidentification implies that there exists a function $\lambda(Q^e, X^{d}, X^{s})$ such that
    \begin{align}
        &\frac{\partial P}{\partial X^{d}_i}(Q^e, X^{d}) + \theta^{*}\frac{\partial^2 P}{\partial X^{d}_{i}\partial Q}(Q^e, X^{d})Q^e \\
        &\quad = \lambda(Q^e, X^{d}, X^{s}) \left[\frac{\partial P}{\partial X^{d}_i}(Q^e, X^{d}) + \theta\frac{\partial^2 P}{\partial X^{d}_{i}\partial Q}(Q^e, X^{d})Q^e \right] \quad \forall i = 1, \ldots, K_d,\label{eq:lambda_foc_demand}
    \end{align}
\end{lemma}

\begin{proof}
Assume that given $X^{d}$ and $X^{s}$, we can solve the first-order condition for the equilibrium quantity $Q^e$.
The derivatives of $F$ for each variable are
\begin{align}
    \frac{\partial F}{\partial X^{d}_i}(Q^e, X^{d}, X^{s}; \theta, MC) & =  \frac{\partial P}{\partial X^{d}_{i}}(Q^e, X^{d}) + \theta\frac{\partial^2 P}{\partial X^{d}_{i}\partial Q}(Q^e, X^{d})Q^e, \quad i = 1, \ldots, K_d,\\
    \frac{\partial F}{\partial X^{s}_i}(Q^e, X^{d}, X^{s}; \theta, MC) & =  -\frac{\partial MC}{\partial X^{s}_{i}}(Q^e, X^{s}), \quad i = 1, \ldots, K_s, \\
    \frac{\partial F}{\partial Q}(Q^e, X^{d}, X^{s}; \theta, MC) & = (1+\theta)\frac{\partial P}{\partial Q}(Q^e, X^{d}) + \theta\frac{\partial^2 P}{\partial Q^2}(Q^e, X^{d})Q^e - \frac{\partial MC}{\partial Q}(Q^e, X^{s}).
\end{align}
Assume that the derivative of $F$ with respect to $Q$, the last equation, at the equilibrium quantity $Q^e$ is nonzero.
Then, we can apply the implicit function theorem, and for the reduced-form equation $Q^e = h(X^{d}, X^{s})$, the derivative of $h$ with respect to the demand shifters and the cost shifters are
\begin{align}
    Dh(X^{d}, X^{s}) = \begin{pmatrix}
        \dfrac{\partial h}{\partial X^{d}}(X^{d}, X^{s})\\[1em]
        \dfrac{\partial h}{\partial X^{s}}(X^{d}, X^{s})
    \end{pmatrix} = \begin{pmatrix}
        -\dfrac{\frac{\partial F}{\partial X^{d}}(h(X^{d}, X^{s}), X^{d}, X^{s})}{\frac{\partial F}{\partial Q}(h(X^{d}, X^{s}), X^{d}, X^{s})}\\[1.5em]
        -\dfrac{\frac{\partial F}{\partial X^{s}}(h(X^{d}, X^{s}), X^{d}, X^{s})}{\frac{\partial F}{\partial Q}(h(X^{d}, X^{s}), X^{d}, X^{s})}
    \end{pmatrix}.\label{eq:foc_derivative_demand_supply}
\end{align}
Thus, the derivative of $h$ with respect to $X^{d}_i$ and $X^{s}_i$ are given as for $i = 1, \ldots, K_d$ and $i = 1, \ldots, K_s$, respectively,
\begin{align}
    \frac{\partial h}{\partial X^{d}_{i}}(X^{d}, X^{s}) = -\frac{\frac{\partial P}{\partial X^{d}_{i}}(Q^e, X^{d}_i) + \theta\frac{\partial^2 P}{\partial X^{d}_{i}\partial Q}(Q^e, X^{d}_i)Q^e }{(1+\theta)\frac{\partial P}{\partial Q}(Q^e, X^{d}_i) + \theta\frac{\partial^2 P}{\partial Q^2}(Q^e, X^{d}_i)Q^e - \frac{\partial MC}{\partial Q}(Q^e, X^{s})}, \label{eq:foc_derivative_demand}
\end{align}
and
\begin{align}
    \frac{\partial h}{\partial X^{s}_{i}}(X^{d}, X^{s}) & = \frac{\frac{\partial MC}{\partial X^{s}_{i}}(Q^e, X^{s}_i)}{(1+\theta)\frac{\partial P}{\partial Q}(Q^e, X^{d}_i) + \theta\frac{\partial^2 P}{\partial Q^2}(Q^e, X^{d}_i)Q^e - \frac{\partial MC}{\partial Q}(Q^e, X^{s}_i)}. \label{eq:foc_derivative_supply}
\end{align}

% Use the definition of non-identification
Note that the same argument can be applied to the alternative model, and we have $Dh^{*}(X^{d}, X^{s})$.
The non-identification implies that $Q^e = h(X^{d}, X^{s}) = h^{*}(X^{d}, X^{s})$ for all $X^{d}$ and $X^{s}$, and hence we must have
\begin{align}
    Dh^{*}(X^{d}, X^{s}) = Dh(X^{d}, X^{s}) \quad \forall X^{d}, X^{s}. \label{eq:observale_equivalence_derivative}
\end{align}
From \eqref{eq:foc_derivative_demand_supply}, this implies that for all $Q^e$, $X^{d}$, and $X^{s}$,
\begin{align}
    \begin{pmatrix}
        \frac{\partial F}{\partial X^{d}}(Q^e, X^{d}, X^{s}; \theta^{*}, MC^{*})\\
        \frac{\partial F}{\partial X^{s}}(Q^e, X^{d}, X^{s}; \theta^{*}, MC^{*})
    \end{pmatrix}
    = \lambda(Q^e, X^{d}, X^{s})
    \begin{pmatrix}
        \frac{\partial F}{\partial X^{d}}(Q^e, X^{d}, X^{s}; \theta, MC)\\
        \frac{\partial F}{\partial X^{s}}(Q^e, X^{d}, X^{s}; \theta, MC),
    \end{pmatrix},\label{eq:foc_derivative_demand_supply_lambda}
\end{align}
where $\lambda(Q^e, X^{d}, X^{s})$ is defined as
\begin{align}
    \lambda(Q^e, X^{d}, X^{s}) \equiv \frac{(1+\theta^{*})\frac{\partial P}{\partial Q}(Q^e, X^{d}) + \theta^{*}\frac{\partial^2 P}{\partial Q^2}(Q^e, X^{d})Q^e - \frac{\partial MC^{*}}{\partial Q}(Q^e, X^{s})}{(1+\theta)\frac{\partial P}{\partial Q}(Q^e, X^{d}) + \theta\frac{\partial^2 P}{\partial Q^2}(Q^e, X^{d})Q^e - \frac{\partial MC}{\partial Q}(Q^e, X^{s})}. \label{eq:lambda_foc}
\end{align}
It is straightforward to see that the first row in \eqref{eq:foc_derivative_demand_supply_lambda} is equivalent to \eqref{eq:lambda_foc_demand}.
\end{proof}

\subsection{Counterexample}

When the separability implies nonidentifcation, separability guarantees that we have the condition \eqref{eq:lambda_foc_demand} for all $Q^e$, $X^{d}$, and $X^{s}$.
Note that when the function is separable $P(Q, r(X^{d}))$, we can regard $r(X^{d})$ as a single variable.
Thus, it is enough to consider the case of the demand shifter is a scalar.
In the following counterexample, we show that there is an inverse demand function that violates the condition \eqref{eq:lambda_foc_demand} at some $Q^e$, $X^{d}$, and $X^{s}$.
Therefore, separability does not imply nonidentification.

Assume that the inverse demand function is given by
\begin{align}
    P(Q, X^{d}) = X^{d}\exp(-Q).
\end{align}
Suppose for the sake of a contradiction that given two pairs of the conduct parameter and the marginal cost, we have \eqref{eq:lambda_foc_demand} for all $Q^e$, $X^{d}$, and $X^{s}$.
Under the inverse demand function, \eqref{eq:lambda_foc_demand} is written as
\begin{align}
    \exp(-Q)(1 -\theta^{*} Q) = \lambda(Q^e, X^{d}, X^{s}) \exp(-Q)(1 - \theta Q).
\end{align}
First, when $\theta >0$, the left-hand side is nonzero but the right-hand side is zero when $Q = \frac{1}{\theta}$.
When $\theta = 0$ and $\theta^{*} >0$, the left-hand side is zero when $Q = \frac{1}{\theta^{*}}$.
In contrast, as we assume that the $\lambda(Q^e, X^{d}, X^{s})$ is always positive, the right-hand side is always nonzero for any $Q$.
Therefore, there is a separable inverse demand function that leads to identification at some $Q^e$, $X^{d}$, and $X^{s}$, which is a contradiction to the sufficiency of separability in Theorem \ref{theorem_lau}.






\section{Conclusion and discussion}

We present a counterexample to \citet{lau1982identifying}'s identification result, separability does not imply nonidentification.
Our finding is that even though non-identification implies the separability of the inverse demand function, the class of separable inverse demand functions that leads to non-identification is more restricted than the class considered in \citet{lau1982identifying}.

Based on the recent literature on distinguishing firm conduct, this result is not surprising.
For example, in the differentiated product environment, \citet{berry2014identification}  use a broader variation in markets to distinguish conduct beyond demand rotation.
In our counterexample, to violate observable equivalence, we need a variation in $X^{d}$ and $X^{s}$ that leads to a specific equilibrium quantity.
Of course, homogeneous product settings are more restricted than differentiated product settings, but it sounds too strong to claim that any separable demand function is a necessary and sufficient condition for non-identification of the conduct parameter.
Rather, even though the true inverse demand function is separable, the variation in markets may help to identify the conduct parameter.

In this note, we have not investigated the exact class of separable inverse demand functions that leads to non-identification of the conduct parameter.
However, in practice, this does not matter much because to estimate the conduct parameter, the researcher uses a parametric assumption on the inverse demand function and the marginal cost function (e.g., \citet{okazaki2022excess} and \citet{matsumura2024loglinear}).
Even though the researcher wants to non-parametrically estimate the marginal cost function, at least a non-separable demand function leads to the identification of the conduct parameter.

\paragraph{Acknowledgments}
We thank Jeremy Fox for his invaluable comments.
This work was supported by JST ERATO Grant Number JPMJER2301, Japan.  


\newpage
\bibliographystyle{aer}
\bibliography{conduct_parameter.bib}

\end{document}