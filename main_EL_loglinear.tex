\documentclass[11pt, a4paper]{article}
\usepackage[utf8]{inputenc}
\usepackage{amsmath,setspace,geometry}
\usepackage{amsthm}
\usepackage{amsfonts}
\usepackage[shortlabels]{enumitem}
\usepackage{rotating}
\usepackage{pdflscape}
\usepackage{graphicx}
\usepackage{bbm}
\usepackage[dvipsnames]{xcolor}
\usepackage{hyperref}
\hypersetup{colorlinks=true, linkcolor= BrickRed, citecolor = BrickRed, filecolor = BrickRed, urlcolor = BrickRed, hypertexnames = true}
\usepackage[]{natbib} 
\bibpunct[:]{(}{)}{,}{a}{}{,}
\geometry{left = 1.0in,right = 1.0in,top = 1.0in,bottom = 1.0in}
\usepackage[english]{babel}
\usepackage{float}
\usepackage{caption}
\usepackage{subcaption}
\usepackage{booktabs}
\usepackage{pdfpages}
\usepackage{threeparttable}
\usepackage{lscape}
\usepackage{bm}
\setstretch{1.4}
%\usepackage[tablesfirst,nolists]{endfloat}

\newtheorem{theorem}{Theorem}
\newtheorem{assumption}{Assumption}
\newtheorem{lemma}{Lemma}
\newtheorem{definition}{Definition}
\newtheorem{proposition}{Proposition}
\newtheorem{claim}{Claim}
\newtheorem{corollary}{Corollary}
\newtheorem{example}{Example}
\DeclareMathOperator{\rank}{rank}


\title{An MPEC Estimator for Conduct Parameter Estimation in Homogeneous Goods Markets}
\author{Yuri Matsumura\thanks{Department of Economics, Rice University. Email: Yuri.Matsumura@rice.edu} \and Suguru Otani \thanks{Department of Economics, Rice University. Email: so19@rice.edu
%Declarations of interest: none %this is for Economics Letters
}}

\begin{document}

\maketitle
\begin{abstract}
    We propose a constrained the Generalized Method of Moment (GMM) estimator for conduct parameter estimation in homogeneous goods markets, by formulating the estimation as an MPEC (Mathematical Programming with Equilibrium Constraints) problem. Our approach avoids the complex transformation within the equilibrium conditions and allows a broader set of specifications which could not be estimated by standard approaches. Monte Carlo simulations confirm that the proposed estimator works with small bias and Root-Mean-Squared-Error for the typical specification, i.e., log-linear model. %We conclude that MPEC enables us to estimate a broader set of specifications which are identified theoretically shown in Lau (1982) but practically and numerically difficult to estimate.
\end{abstract}


\section{Introduction}
Measuring competitiveness is one of the important tasks in empirical industrial organization literature.
Conduct parameter is considered to be a useful measure of competitiveness. 
However, it cannot be directly measured from data because data usually lack information about marginal cost.
Therefore, researchers endeavor to identify and estimate the conduct parameter.

As the simplest specification, \citet{bresnahan1982oligopoly} considers the identification of conduct parameter for the linear model. \cite{matsumura2023revisiting} resolves the conflict on identification problem between \cite{bresnahan1982oligopoly} and \cite{perloff2012collinearity}. 
On the other hand, researchers may want to implement a different set of specifications such as log-linear model, e.g., \cite{okazaki2022excess} and \cite{merel2009measuring}. As for the log-linear model, the identification strategy is provided by \citet{lau1982identifying}. 
Estimation problems arise for the case, however, when searching parameters using the standard solvers because the equilibrium condition given by demand and supply curves involves log-transformation. 
This is the obstacle to choosing the better specification of the demand and supply functions.

To overcome the problem, we propose a new estimator based on the mathematical program with equilibrium constraints (MPEC) approach advocated by \cite{su2012constrained}. 
MPEC is a constrained optimization problem whose constraint structure contains the equilibrium constraints. 
We show that MPEC estimator works well for the log-linear model. 
That is, MPEC allows a broader set of specifications than the standard approach using derivative or non-derivative solvers. 
We also show that increasing the sample size improves the accuracy of estimation.
We conclude that MPEC enables us to estimate a broader set of specifications which are identified theoretically shown in \cite{lau1982identifying} but practically and numerically difficult to estimate in the standard approach.


\section{Model}
The researcher has data with $T$ markets with homogeneous products.
Assume that there are $N$ firms in each market.
Let $t = 1,\ldots, T$ be the index of markets.
Then, we obtain the supply equation as follows:
\begin{align}
     P_t = -\theta\frac{\partial P_t(Q_{t})}{\partial Q_{t}}Q_{t} + MC_t(Q_{t}),\label{eq:supply_equation}
\end{align}
where $Q_{t}$ is the aggregate quantity, $P_t(Q_{t})$ is the demand function, $MC_{t}(Q_{t})$ is the marginal cost function, and $\theta\in[0,1]$, which is called conduct parameter. 
The equation nests perfect competition, $\theta=0$, Cournot competition, $\theta=1/N, N$ firm symmetric perfect collusion, $\theta=1$, etc.\footnote{See \cite{bresnahan1982oligopoly}.} 

Consider an econometric model of the above model.
Assume that the demand function and the marginal cost function are written as follows: 
\begin{align}
    P_t = f(Q_{t}, Y_t, \varepsilon^{d}_{t}, \alpha) \label{eq:demand}\\
    MC_t = g_{t}(Q_{t}, W_{t}, \varepsilon^{c}_{t}, \gamma)\label{eq:marginal_cost}
\end{align}
where $Y_t$ and $W_{t}$ are the vector of exogenous variables, $\varepsilon^{d}_{t}$ and $\varepsilon^{c}_{t}$ are the error terms, and $\alpha$ and $\gamma$ are the vector of parameters.
We also have the demand- and supply-side instrument variables $Z^{d}_{t}$ and $Z^{c}_{t}$, and assume that the error terms satisfy the mean independence condition $E[\varepsilon^{d}_{t}\mid Y_t, Z^{d}_{t}] = E[\varepsilon^{c}_{t} \mid W_{t}, Z^{c}_{t}] =0$.

\subsection{Log-linear demand and log-linear marginal cost}
We consider the most typical specification which is nonlinear-in-parameters as the log-linear model.
Assume that log-linear demand and cost functions are specified as:
\begin{align}
    \log P_{t} &= \alpha_0 - (\alpha_1 + \alpha_2 Z^{R}_{t}) \log Q_t + \alpha_3 \log Y_t + \varepsilon^{d}_{t},\label{eq:log_linear_demand}\\
    \log MC_t &= \gamma_0 + \gamma_1 \log Q_t +  \gamma_2 \log W_{t} + \gamma_3 \log R_t + \varepsilon^{c}_{t}.\label{eq:log_linear_marginal_cost}
\end{align}
Since $\partial P_t/\partial Q_t = - (\alpha_1 + \alpha_2 Z_{t}^R) (P_t/Q_t) $, Equation \eqref{eq:supply_equation} is written as:
\begin{align}
    P_t &= \theta (\alpha_1 + \alpha_2 Z^{R}_{t}) P_t + MC_t.\label{eq:log_linear_supply_equation_direct}
\end{align}
By reformulating this and taking logarithm, $\log P_t(1 -\theta (\alpha_1 + \alpha_2 Z^{R}_{t})) = \log MC_t.$
Then, we obtain:
\begin{align}
    \log P_t = - \log_{t}(1 - \theta(\alpha_1 + \alpha_2 Z^{R}_{t})) + \gamma_0 + \gamma_1 \log Q_t +  \gamma_2 \log W_{t} + \gamma_3 \log R_t + \varepsilon^{c}_{t}. \label{eq:log_linear_supply_equation}
\end{align}
By substituting the Equation \eqref{eq:log_linear_demand} into Equation \eqref{eq:log_linear_supply_equation} and solving it for $P_{t}$, the log aggregate quantity is given as: 
\begin{align}
    \log Q_t &= \frac{ \alpha_0 + \alpha_3 \log Y_t + \log (1 - \theta (\alpha_1 + \alpha_2 Z^{R}_{t})) - \gamma_0  -  \gamma_2 \log W_{t} - \gamma_3 \log R_t + \varepsilon^{d}_{t} - \varepsilon^{c}_{t}}{\gamma_1+ \alpha_1 + \alpha_2 Z^{R}_{t} }.\label{eq:quantity_loglinear}
\end{align}
%An obstacle to estimate parameters is log-transformation in Equation \eqref{eq:log_linear_supply_equation}. First, the standard numerical algorithm faces invalid errors when evaluating the term, $\log (1 - \theta (\alpha_1 + \alpha_2 Z^{R}_{t}))$, whose inner candidate value, $(1 - \theta (\alpha_1 + \alpha_2 Z^{R}_{t}))$, is negative. The problem is inevitable because the true values of parameters are unknown. As in practical situations, researchers may want to use multiple rotation demand instruments or conduct parameters varied with time- or market- fixed effects.\footnote{See \cite{michel2018estimating} for the case of differentiated goods markets.} Then, the problem regarding the term becomes severer because the number of parameters within the log term increases. Second, if researchers may want to specify the demand function as $P_t=\alpha_0-\left(\alpha_1+\alpha_2 Z_t^R\right) Q_t+\alpha_3 Y_t+\varepsilon_t^d$, then the supply equation \eqref{eq:log_linear_supply_equation} is changed to $P_t=\theta \alpha_2 Z_t^R Q_t+\theta \alpha_1Q_t+\exp(\gamma_0+\gamma_1 \log Q_t+\gamma_2 \log W_t+\gamma_3 \log R_t+\varepsilon_t^c)$ which is nonlinear in cost parameters and conduct parameter $\theta$. 
%When constructing moment conditions, researchers face the above problem on log-transformation which limit researchers' alternative set of potential specifications.

An obstacle in the estimation is log-transformation in Equation \eqref{eq:log_linear_supply_equation}.
The standard numerical algorithm stops with invalid errors when the inside of the log function, $(1 - \theta (\alpha_1 + \alpha_2 Z^{R}_{t}))$, becomes negative during the search for an optimizer.
This could happen even when researchers restrict the parameter space.
This problem is not specific to the above model and happens in other specification such as a model with linear demand and log-linear marginal cost.



% \subsection{Log-linear demand and log-quadratic marginal cost}
% Next, we consider more nonlinear specification.
% Assume that the cost function is a log-quadratic form specified as:
% \begin{align}
%     \log MC_t &= \gamma_0 + \gamma_1 \log Q_t + \tilde{\gamma}_1 (\log Q_t)^{2} +  \gamma_2 \log W_{t} + \gamma_3 \log R_t + \varepsilon^{c}_{t}.
% \end{align}
% Then, the log aggregate quantity $\log Q_t $ is a solution of: 
% \begin{align}
%     (\log Q_t)^{2} + \frac{\gamma_1}{\tilde{\gamma}_{1}}\log Q_{t} +\frac{C}{\tilde{\gamma}_{1}} &= 0,
% \end{align}
% where $C=\alpha_0 + \alpha_3 \log Y_t + \log (1 - \theta (\alpha_1 + \alpha_2 Z^{R}_{t})) - \gamma_0  -  \gamma_2 \log W_{t} - \gamma_3 \log R_t + \varepsilon^{d}_{t} - \varepsilon^{c}_{t}$.
% Note that although this specification does not have an analytical solution, we can find numerically find a single positive solution.

\section{An MPEC estimator for the conduct parameter model}

We propose a novel and simple estimator for the log-linear conduct parameter model by utilizing the Mathematical Programming with Equilibrium Constraints (MPEC) procedure advocated by \cite{su2012constrained}. Our setting is similar to \cite{dube2012improving} in which the GMM objective function with equilibrium constraints is constructed. 
MPEC estimator avoids the above log-transformation problem by evaluating the equilibrium equations involving log-transformation as constraints.
Let $\mathcal{T}$ be the set of markets. Let $\xi = (\alpha_0,\alpha_1, \ldots, \alpha_3, \gamma_0,\gamma_1, \ldots, \gamma_3, \theta)$ be the vector of parameters. Then, we obtain $\xi$ by solving the following constrained optimization problem:
\begin{align}
    &\min_{\xi, \{MC_t\}_{t\in \mathcal{T}}}\quad \left(\frac{1}{T}\sum_{t=1}^{T}g_{t}(\xi)\right)^{\top} W\left(\frac{1}{T}\sum_{t=1}^{T}g_{t}(\xi)\right) \label{eq:objective_gmm}\\
    \text{s.t.}\quad g_{t}(\xi)&=\left[\begin{array}{l}
\varepsilon^{d}_{t}(\xi)Z_{t}^{d} \\
\varepsilon^{c}_{t}(\xi)Z_{t}^{c}
\end{array}\right], \quad W=\frac{1}{T}(Z^\top Z)^{-1}, \quad Z=\left[\begin{array}{c}
Z_{1} \\
\vdots \\
Z_{T}
\end{array}\right],\quad Z_{t}=\left[\begin{array}{ll}
Z_{t}^{d} & 0 \\
0 & Z_{t}^{c}
\end{array}\right],\quad  \forall t \in \mathcal{T}\nonumber\\
    \varepsilon^{d}_{t}(\xi)&=\log P_{t} -[\alpha_0 - (\alpha_1 + \alpha_2 Z^{R}_{t}) \log Q_t + \alpha_3 \log Y_t],\quad  \forall t \in \mathcal{T} \label{eq:demand_gmm}\\
    \varepsilon^{c}_{t} (\xi)&= \log MC_t -[\gamma_0 + \gamma_1 \log Q_t +  \gamma_2 \log W_{t} + \gamma_3 \log R_t],\quad  \forall t \in \mathcal{T}\label{eq:supply_gmm}\\
    P_t &= \theta (\alpha_1 + \alpha_2 Z^{R}_{t})P_t + MC_t,\quad  \forall t \in \mathcal{T},\label{eq:equilibrium_constraint}\\
    &0\le\theta \le 1,\quad  0 \le \alpha_0,\quad 0 \le \alpha_1, \quad 0 \le \alpha_2, \quad  0\le \gamma_0,\quad  0 \le MC_t  \quad  \forall t \in \mathcal{T}\label{eq:parameter_constraint}
\end{align}
where $W$ is the weighting matrix, Equation \eqref{eq:objective_gmm} is the GMM objective function of the simultaneous equation model, 
Equations \eqref{eq:demand_gmm} and \eqref{eq:supply_gmm} are specifications of demand \eqref{eq:log_linear_demand} and supply \eqref{eq:log_linear_marginal_cost} curves, 
Equation \eqref{eq:equilibrium_constraint} is the equilibrium constraint from Equation \eqref{eq:log_linear_supply_equation_direct}, and Equation \eqref{eq:parameter_constraint} are constraints on the conduct parameter and constant term parameters of the demand and supply curves. 
Constraints \eqref{eq:parameter_constraint} are practically necessary to avoid searching theoretically implausible parameters from the theoretical model. Note that MPEC allows for more flexible specifications on Equations \eqref{eq:demand_gmm}, \eqref{eq:supply_gmm}, and \eqref{eq:equilibrium_constraint} that the standard optimization technique does not allow. Also, the programming cost of MPEC is lower than the standard approach because MPEC does not need analytical equilibrium expressions for constructing moment conditions.




%581words by Suguru

\section{Simulation results}\label{sec:results}

Table \ref{tb:loglinear_loglinear_sigma_2_mpec_non_constraint_theta_constraint} presents the results of the log-linear model.\footnote{See Appendix \ref{sec:appendix} for the simulation details and additional results. We confirm that the standard simultaneous equation approach using derivative free algorithms such as Nelder-Mead could not properly solve the model because the above log-transformation problem often arises.} 
First, MPEC solves all simulation data properly, which is the practical advantage over the standard approach.
Second, Panel (a) shows that when the standard deviation of the error terms in the demand and supply equation is $\sigma = 0.5$, the estimation of all parameters is reasonably accurate.
Panel (c) shows the case with $\sigma = 2.0$. 
As the sample size increases, the RMSE and bias except $\gamma_0$ decreases. These magnitudes shows similar levels of the results of the linear model, shown in Appendix. 
That is, MPEC for the log-linear model is as good as the linear model.



\begin{table}[!htbp]
  \begin{center}
      \caption{MPEC Results of the log-linear model}
      \label{tb:loglinear_loglinear_sigma_2_mpec_non_constraint_theta_constraint} 
      \subfloat[$\sigma=0.5$]{
\begin{tabular}[t]{llrrrrrrr}
\toprule
  & Bias & RMSE & Bias & RMSE & Bias & RMSE & Bias & RMSE\\
\midrule
$\alpha_{0}$ & -0.250 & 1.106 & -0.060 & 1.013 & 0.013 & 0.848 & -0.027 & 0.352\\
$\alpha_{1}$ & -0.146 & 0.642 & -0.040 & 0.568 & 0.004 & 0.481 & -0.015 & 0.203\\
$\alpha_{2}$ & 0.006 & 0.109 & 0.011 & 0.098 & 0.008 & 0.072 & -0.001 & 0.030\\
$\alpha_{3}$ & -0.057 & 0.431 & -0.015 & 0.316 & 0.003 & 0.266 & -0.007 & 0.114\\
$\gamma_{0}$ & -0.285 & 1.798 & -0.124 & 1.224 & -0.061 & 0.837 & 0.062 & 0.402\\
$\gamma_{1}$ & 0.036 & 0.566 & 0.034 & 0.384 & 0.018 & 0.246 & -0.001 & 0.104\\
$\gamma_{2}$ & 0.007 & 0.381 & 0.017 & 0.258 & 0.011 & 0.165 & 0.000 & 0.073\\
$\gamma_{3}$ & 0.017 & 0.393 & 0.021 & 0.262 & 0.009 & 0.175 & -0.001 & 0.077\\
$\theta$ & 0.066 & 0.356 & -0.019 & 0.285 & -0.027 & 0.245 & -0.061 & 0.187\\
Runs converged (\%) &  & 100.000 &  & 100.000 &  & 100.000 &  & 100.000\\
Sample size ($T$) &  & 50 &  & 100 &  & 200 &  & 1000\\
\bottomrule
\end{tabular}
}\\
      \subfloat[$\sigma=1.0$]{
\begin{tabular}[t]{llrrrrrrr}
\toprule
  & Bias & RMSE & Bias & RMSE & Bias & RMSE & Bias & RMSE\\
\midrule
$\alpha_{0}$ & -3.425 & 4.238 & -3.251 & 10.595 & -0.930 & 1.707 & -0.494 & 1.046\\
$\alpha_{1}$ & -0.946 & 1.169 & -0.955 & 2.979 & -0.241 & 0.467 & -0.138 & 0.288\\
$\alpha_{2}$ & -0.102 & 0.184 & 0.051 & 0.139 & -0.028 & 0.087 & -0.007 & 0.022\\
$\alpha_{3}$ & -0.801 & 1.086 & -0.672 & 1.913 & -0.080 & 0.300 & -0.131 & 0.224\\
$\gamma_{0}$ & 1.943 & 2.779 & 0.403 & 1.969 & -0.154 & 1.118 & -0.041 & 0.983\\
$\gamma_{1}$ & -0.383 & 0.603 & -0.014 & 0.293 & 0.011 & 0.242 & 0.025 & 0.193\\
$\gamma_{2}$ & -0.255 & 0.451 & -0.124 & 0.454 & 0.006 & 0.169 & -0.001 & 0.218\\
$\gamma_{3}$ & -0.379 & 0.682 & -0.297 & 0.395 & -0.077 & 0.344 & -0.005 & 0.180\\
$\theta$ & -0.023 & 0.358 & 0.103 & 0.425 & 0.079 & 0.356 & -0.072 & 0.227\\
Runs converged (\%) &  & 100.000 &  & 100.000 &  & 100.000 &  & 100.000\\
Sample size ($T$) &  & 50 &  & 100 &  & 200 &  & 1000\\
\bottomrule
\end{tabular}
}\\
    \subfloat[$\sigma=2.0$]{
\begin{tabular}[t]{llrrrrrrr}
\toprule
  & Bias & RMSE & Bias & RMSE & Bias & RMSE & Bias & RMSE\\
\midrule
$\alpha_{0}$ & -0.751 & 9.838 & -0.376 & 7.604 & 0.138 & 4.439 & 0.312 & 3.648\\
$\alpha_{1}$ & -0.096 & 1.553 & -0.048 & 1.145 & 0.023 & 0.679 & 0.048 & 0.561\\
$\alpha_{2}$ & -0.045 & 0.123 & -0.030 & 0.141 & -0.004 & 0.061 & 0.001 & 0.044\\
$\alpha_{3}$ & -0.052 & 0.899 & -0.064 & 0.799 & -0.005 & 0.409 & 0.019 & 0.321\\
$\gamma_{0}$ & -7.081 & 23.622 & -5.249 & 16.189 & -1.524 & 4.881 & -1.124 & 3.419\\
$\gamma_{1}$ & 0.473 & 3.114 & 0.371 & 2.033 & 0.139 & 0.591 & 0.117 & 0.411\\
$\gamma_{2}$ & 0.160 & 2.041 & 0.180 & 1.278 & 0.074 & 0.360 & 0.048 & 0.261\\
$\gamma_{3}$ & 0.239 & 2.345 & 0.205 & 1.437 & 0.070 & 0.382 & 0.056 & 0.275\\
$\theta$ & 0.250 & 0.460 & 0.220 & 0.448 & 0.098 & 0.358 & 0.081 & 0.312\\
Runs converged (\%) &  & 99.400 &  & 99.500 &  & 100.000 &  & 100.000\\
Sample size ($T$) &  & 100 &  & 200 &  & 1000 &  & 1500\\
\bottomrule
\end{tabular}
}
  \end{center}
  \footnotesize
  Note: The error terms in the demand and supply equation are drawn from a normal distribution, $N(0, \sigma)$.
\end{table} 



\section{Conclusion}
We propose a constrained GMM estimator for conduct parameter estimation in homogeneous goods markets, by formulating the estimation as an MPEC problem. 
Our approach avoids the complex transformation within the equilibrium conditions and allows a broader set of specifications which could not be used by standard approaches. 
We show that MPEC estimator works well for the log-linear model. 
%We conclude that MPEC enables us to estimate a broader set of specifications which are identified theoretically shown in \cite{lau1982identifying} but practically and numerically difficult to estimate.
% We revisit the conduct parameter estimation in homogeneous goods markets.
% There is a conflict between \citet{bresnahan1982oligopoly} and \citet{perloff2012collinearity} in terms of identification and estimation.
% We highlight the problems in the proof and simulation in \citet{perloff2012collinearity}.
% Our simulation shows that the estimation of the conduct parameter becomes accurate by appropriately introducing demand shifters into the supply estimation and increasing the sample size. 
% Based on the theoretical and numerical investigation, we support the argument in \citet{bresnahan1982oligopoly}.


\paragraph{Acknowledgments}
We thank Jeremy Fox and Isabelle Perrigne for their valuable advice. This research did not receive any specific grant from funding agencies in the public, commercial, or not-for-profit sectors. 

\newpage
\bibliographystyle{aer}
\bibliography{conduct_parameter}

\newpage
\appendix

\section{Online appendix}\label{sec:appendix}

\subsection{Simulation and estimation procedure}
We assess the performance of the MPEC estimator using Monte Carlo simulation.
We set the true parameters and distributions as shown in Table \ref{tb:parameter_setting}. 
To generate the simulation data, for each model, we first generate the exogenous variables $Y_t, Z^{R}_{t}, W_t, R_{t}, H_t$, and $K_t$ and the error terms $\varepsilon_{t}^c$ and $\varepsilon_{t}^d$ based on the data generation process in Table \ref{tb:parameter_setting}.
We compute the equilibrium quantity $Q_{t}$ for the log-linear model by \eqref{eq:quantity_loglinear}.
We then compute the equilibrium price $P_t$ by substituting $Q_{t}$ and other variables into the demand function \eqref{eq:log_linear_demand}.
We generate 1,000 data sets.
We jointly estimate the demand and supply equation by simultaneous equation model from the true values.
The instrument variables for the demand estimation are $Z^{d}_{t} = (Z^{R}_{t}, Y_t, H_{t}, K_{t})$ and the instrument variables for the supply estimation are $Z^{c}_{t} = (Z^{R}_{t}, W_{t}, R_{t}, Y_t)$. 
We estimate the model using the \texttt{Ipopt.jl} and \texttt{JuMP.jl} packages in \texttt{Julia}.




\begin{table}[!htbp]
    \caption{True parameters and distributions}
    \label{tb:parameter_setting}
    \begin{center}
    \subfloat[Parameters]{
    \begin{tabular}{crr}
            \hline
            $\alpha_0$  & $10.0$ &\\
            $\alpha_1$ & $1.0$  &\\
            $\alpha_2$ & $0.1$ &\\
            $\alpha_3$ & $1.0$ &\\
            $\gamma_0$ & $5.0$  &\\
            $\gamma_1$ & $1.0$  &\\
            $\gamma_2$ & $1.0$ &\\
            $\gamma_3$ & $1.0$ &\\
            $\theta$ & $0.2$  &\\
            \hline
        \end{tabular}
    }
    \subfloat[Distributions]{
    \begin{tabular}{crr}
            \hline
            Demand shifter&  &  \\
            $Y_t$ & $N(0,1)$ \\
            Demand rotation instrument&  &  \\
            $Z^{R}_{t}$ & $U(0,1)$ \\
            Cost shifter  &  \\
            $W_{t}$ & $U(1,3)$ \\
            $R_t$  & $U(1,3)$  \\
            $H_{t}$ & $W_{t}+U(0,1)$  \\
            $K_{t}$ & $R_{t}+U(0,1)$  \\
            Error&  &  \\
            $\varepsilon^{d}_{t}$ & $N(0,\sigma)$  \\
            $\varepsilon^{c}_{t}$ & $N(0,\sigma)$ \\
            \hline
        \end{tabular}
    }
    \end{center}
    \footnotesize
    Note: $\sigma=\{0.5, 1.0, 2.0\}$. $N:$ Normal distribution. $U:$ Uniform distribution.
\end{table}

\subsection{Standard Estimation Methods}
We summarize the results of the standard estimation using a derivative-free optimization algorithm such as Nelder-Mead algorithm and the standard estimation using state-of-the-art constrained optimization solvers, i.e., \texttt{Ipopt.jl} which implements an interior point line search filter method that aims to find a local solution of nonlinear programming problems. 
First, as mentioned in the main text, the standard simultaneous equation approach using derivative-free algorithms such as Nelder-Mead algorithm could not properly finish the optimization routine to solve the model because the above log-transformation problem arises even if the parameter search starts from the true values. 
Second, we examine the Two-Stage-Least-Square (2SLS) model in which Ipopt is implemented for minimizing its supply side moments and the simultaneous equation model in which Ipopt is implemented for minimizing both demand and supply side moments. 
We refer to the former as separate estimation and the latter as simultaneous estimation.
These two estimation procedures are considered to be a middle ground between MPEC and standard approaches because these do not use equilibrium constraints but state-of-the-art constrained optimization
solvers.
Tables \ref{tb:loglinear_loglinear_sigma_2_separate_non_constraint_theta_constraint_bias_rmse} and \ref{tb:loglinear_loglinear_sigma_2_simultaneous_non_constraint_theta_constraint_bias_rmse} shows that both estimations fail in estimation for about 1\% of the simulation samples even if the parameter search starts from the true values. 
Also, if $\sigma=2.0$ and $T=1000$, RMSEs are more than 10 for three parameters. Thus, MPEC is better than these approaches even for the typical and simple log-linear model. 





\begin{table}[!htbp]
  \begin{center}
      \caption{Results of separate estimation of the log-linear model}
      \label{tb:loglinear_loglinear_sigma_2_separate_non_constraint_theta_constraint_bias_rmse} 
      \subfloat[$\sigma=0.5$]{
\begin{tabular}[t]{llrrrrrrr}
\toprule
  & Bias & RMSE & Bias & RMSE & Bias & RMSE & Bias & RMSE\\
\midrule
$\alpha_{0}$ & -0.442 & 1.669 & -0.122 & 0.943 & -0.017 & 0.612 & 0.007 & 0.267\\
$\alpha_{1}$ & -0.246 & 0.949 & -0.071 & 0.529 & -0.013 & 0.348 & 0.005 & 0.151\\
$\alpha_{2}$ & 0.022 & 0.173 & 0.013 & 0.106 & 0.006 & 0.068 & -0.001 & 0.030\\
$\alpha_{3}$ & -0.109 & 0.544 & -0.030 & 0.297 & -0.007 & 0.205 & 0.003 & 0.090\\
$\gamma_{0}$ & -0.356 & 1.739 & -0.321 & 1.248 & -0.200 & 0.855 & -0.031 & 0.388\\
$\gamma_{1}$ & 0.088 & 0.516 & 0.060 & 0.350 & 0.039 & 0.236 & 0.006 & 0.096\\
$\gamma_{2}$ & 0.040 & 0.399 & 0.035 & 0.260 & 0.024 & 0.172 & 0.004 & 0.073\\
$\gamma_{3}$ & 0.042 & 0.391 & 0.035 & 0.262 & 0.021 & 0.179 & 0.003 & 0.077\\
$\theta$ & 0.125 & 0.409 & 0.120 & 0.363 & 0.068 & 0.285 & 0.000 & 0.163\\
Runs converged (\%) &  & 97.400 &  & 98.300 &  & 98.900 &  & 100.000\\
Sample size ($T$) &  & 50 &  & 100 &  & 200 &  & 1000\\
\bottomrule
\end{tabular}
}\\
      \subfloat[$\sigma=1.0$]{
\begin{tabular}[t]{llrrrrrrr}
\toprule
  & Bias & RMSE & Bias & RMSE & Bias & RMSE & Bias & RMSE\\
\midrule
$\alpha_{0}$ & -1.209 & 2.406 & -0.848 & 2.098 & -0.337 & 1.188 & -0.014 & 0.512\\
$\alpha_{1}$ & -0.716 & 1.377 & -0.488 & 1.208 & -0.188 & 0.680 & -0.011 & 0.292\\
$\alpha_{2}$ & -0.044 & 0.307 & -0.024 & 0.202 & -0.021 & 0.133 & 0.000 & 0.056\\
$\alpha_{3}$ & -0.403 & 0.954 & -0.250 & 0.717 & -0.095 & 0.412 & -0.010 & 0.178\\
$\gamma_{0}$ & 4.537 & 7.248 & 3.715 & 8.543 & 3.653 & 4.268 & 3.735 & 3.838\\
$\gamma_{1}$ & -0.189 & 2.213 & 0.072 & 3.184 & 0.043 & 0.687 & 0.028 & 0.233\\
$\gamma_{2}$ & -0.098 & 1.046 & 0.018 & 1.049 & 0.013 & 0.453 & 0.014 & 0.152\\
$\gamma_{3}$ & -0.142 & 1.112 & 0.007 & 1.105 & 0.019 & 0.437 & 0.005 & 0.151\\
$\theta$ & 0.038 & 0.390 & 0.073 & 0.393 & 0.110 & 0.403 & 0.075 & 0.314\\
Runs converged (\%) &  & 95.600 &  & 94.900 &  & 93.500 &  & 99.100\\
Sample size ($T$) &  & 50 &  & 100 &  & 200 &  & 1000\\
\bottomrule
\end{tabular}
}\\
    \subfloat[$\sigma=2.0$]{
\begin{tabular}[t]{llrrrrrrr}
\toprule
  & Bias & RMSE & Bias & RMSE & Bias & RMSE & Bias & RMSE\\
\midrule
$\alpha_{0}$ & -4.631 & 11.380 & -3.911 & 9.947 & -2.650 & 6.010 & -0.257 & 2.077\\
$\alpha_{1}$ & -1.037 & 2.628 & -0.866 & 2.192 & -0.587 & 1.349 & -0.057 & 0.464\\
$\alpha_{2}$ & -0.042 & 0.333 & -0.051 & 0.218 & -0.038 & 0.135 & -0.004 & 0.055\\
$\alpha_{3}$ & -0.496 & 1.096 & -0.475 & 1.706 & -0.302 & 0.703 & -0.027 & 0.240\\
$\gamma_{0}$ & 0.900 & 3.000 & 0.622 & 1.936 & 0.167 & 1.326 & -0.081 & 0.829\\
$\gamma_{1}$ & -0.155 & 0.650 & -0.093 & 0.410 & -0.016 & 0.280 & -0.006 & 0.135\\
$\gamma_{2}$ & -0.048 & 0.937 & -0.102 & 0.643 & -0.033 & 0.480 & -0.006 & 0.219\\
$\gamma_{3}$ & -0.076 & 0.485 & -0.056 & 0.291 & -0.013 & 0.195 & -0.003 & 0.094\\
$\theta$ & -0.058 & 0.396 & -0.027 & 0.416 & 0.023 & 0.411 & 0.039 & 0.364\\
Runs converged (\%) &  & 0.410 &  & 0.466 &  & 0.532 &  & 0.626\\
Sample size ($T$) &  & 50 &  & 100 &  & 200 &  & 1000\\
\bottomrule
\end{tabular}
}
  \end{center}
  \footnotesize
  %Note: 
\end{table} 


\begin{table}[!htbp]
  \begin{center}
      \caption{Results of simultaneous estimation of the log-linear model}
      \label{tb:loglinear_loglinear_sigma_2_simultaneous_non_constraint_theta_constraint_bias_rmse} 
      \subfloat[$\sigma=0.5$]{
\begin{tabular}[t]{llrrrrrrr}
\toprule
  & Bias & RMSE & Bias & RMSE & Bias & RMSE & Bias & RMSE\\
\midrule
$\alpha_{0}$ & -0.207 & 2.884 & -0.033 & 1.089 & 0.039 & 0.852 & 0.007 & 0.265\\
$\alpha_{1}$ & -0.113 & 1.605 & -0.022 & 0.614 & 0.019 & 0.482 & 0.005 & 0.150\\
$\alpha_{2}$ & -0.015 & 0.169 & 0.007 & 0.109 & 0.008 & 0.075 & -0.001 & 0.029\\
$\alpha_{3}$ & -0.053 & 0.661 & -0.006 & 0.331 & 0.009 & 0.266 & 0.003 & 0.090\\
$\gamma_{0}$ & -0.778 & 2.847 & -0.554 & 2.430 & -0.322 & 1.088 & -0.089 & 0.470\\
$\gamma_{1}$ & 0.149 & 0.846 & 0.095 & 0.832 & 0.034 & 0.258 & 0.006 & 0.105\\
$\gamma_{2}$ & 0.069 & 0.544 & 0.048 & 0.450 & 0.019 & 0.171 & 0.003 & 0.074\\
$\gamma_{3}$ & 0.074 & 0.540 & 0.049 & 0.387 & 0.017 & 0.180 & 0.002 & 0.078\\
$\theta$ & 0.135 & 0.411 & 0.108 & 0.373 & 0.082 & 0.330 & 0.022 & 0.215\\
Runs converged (\%) &  & 99.000 &  & 98.700 &  & 99.800 &  & 100.000\\
Sample size ($T$) &  & 50 &  & 100 &  & 200 &  & 1000\\
\bottomrule
\end{tabular}
}\\
      \subfloat[$\sigma=1.0$]{
\begin{tabular}[t]{llrrrrrrr}
\toprule
  & Bias & RMSE & Bias & RMSE & Bias & RMSE & Bias & RMSE\\
\midrule
$\alpha_{0}$ & -0.843 & 2.881 & -0.513 & 2.346 & -0.051 & 1.468 & 0.011 & 0.555\\
$\alpha_{1}$ & -0.521 & 1.607 & -0.304 & 1.329 & -0.029 & 0.838 & 0.003 & 0.315\\
$\alpha_{2}$ & -0.017 & 0.357 & -0.008 & 0.242 & -0.007 & 0.153 & 0.001 & 0.057\\
$\alpha_{3}$ & -0.323 & 1.031 & -0.171 & 0.796 & -0.018 & 0.502 & -0.004 & 0.186\\
$\gamma_{0}$ & -0.206 & 11.953 & -0.790 & 8.270 & -0.663 & 2.704 & -0.333 & 1.078\\
$\gamma_{1}$ & 0.032 & 4.260 & 0.191 & 3.329 & 0.089 & 0.775 & 0.030 & 0.234\\
$\gamma_{2}$ & -0.004 & 1.719 & 0.068 & 1.299 & 0.032 & 0.493 & 0.015 & 0.151\\
$\gamma_{3}$ & 0.019 & 3.272 & 0.054 & 1.224 & 0.047 & 0.544 & 0.006 & 0.152\\
$\theta$ & 0.063 & 0.406 & 0.120 & 0.421 & 0.140 & 0.420 & 0.087 & 0.324\\
Runs converged (\%) &  & 98.800 &  & 98.600 &  & 98.400 &  & 99.900\\
Sample size ($T$) &  & 50 &  & 100 &  & 200 &  & 1000\\
\bottomrule
\end{tabular}
}\\
    \subfloat[$\sigma=2.0$]{
\begin{tabular}[t]{llrrrrrrr}
\toprule
  & Bias & RMSE & Bias & RMSE & Bias & RMSE & Bias & RMSE\\
\midrule
$\alpha_{0}$ & -0.945 & 1.891 & -0.616 & 2.338 & -0.412 & 2.105 & -0.023 & 1.204\\
$\alpha_{1}$ & -0.603 & 0.937 & -0.401 & 1.282 & -0.281 & 1.178 & -0.025 & 0.678\\
$\alpha_{2}$ & 0.120 & 0.348 & 0.080 & 0.300 & 0.063 & 0.254 & 0.016 & 0.118\\
$\alpha_{3}$ & -0.275 & 1.115 & -0.178 & 0.855 & -0.133 & 0.830 & -0.019 & 0.420\\
$\gamma_{0}$ & 1.673 & 4.622 & 0.848 & 4.125 & 0.226 & 6.148 & 0.557 & 38.404\\
$\gamma_{1}$ & -0.707 & 1.659 & -0.489 & 1.533 & -0.251 & 2.221 & -0.376 & 13.878\\
$\gamma_{2}$ & -0.412 & 1.444 & -0.253 & 1.029 & -0.137 & 1.475 & -0.171 & 6.611\\
$\gamma_{3}$ & -0.375 & 1.392 & -0.223 & 1.097 & -0.118 & 1.496 & -0.293 & 10.254\\
$\theta$ & 0.199 & 0.422 & 0.213 & 0.442 & 0.199 & 0.440 & 0.176 & 0.409\\
Runs converged (\%) &  & 98.900 &  & 99.800 &  & 99.600 &  & 100.000\\
Sample size ($T$) &  & 50 &  & 100 &  & 200 &  & 1000\\
\bottomrule
\end{tabular}
}
  \end{center}
  \footnotesize
  %Note: 
\end{table} 



\subsection{MPEC for the linear model}

We illustrate that MPEC works for the linear model as in Two-Stage-Least-Square (2SLS) approach in \cite{matsumura2023revisiting}. 
% We confirm that the simultaneous equation model in which demand and supply parameters are jointly estimated and 2SLS model in which demand and supply parameters are separately estimated generate the similar results. 
We follow the setting of \cite{matsumura2023revisiting}.
We put an additional restriction such that $\theta\in[0,1]$ as a theoretical restriction. 
Table \ref{tb:linear_linear_sigma_2_mpec_linear_non_constraint_theta_constraint_bias_rmse} shows that MPEC estimator reduces bias and RMSE in particular when the sample size is large rather than 2SLS estimator shown in Table \ref{tb:linear_linear_sigma_1}.

\begin{table}[!htbp]
  \begin{center}
      \caption{MPEC Results of the linear model}
      \label{tb:linear_linear_sigma_2_mpec_linear_non_constraint_theta_constraint_bias_rmse} 
      \subfloat[$\sigma=0.5$]{
\begin{tabular}[t]{llrrrrrrr}
\toprule
  & Bias & RMSE & Bias & RMSE & Bias & RMSE & Bias & RMSE\\
\midrule
$\alpha_{0}$ & -0.013 & 0.462 & 0.008 & 0.322 & -0.008 & 0.213 & -0.006 & 0.097\\
$\alpha_{1}$ & -0.096 & 2.201 & 0.015 & 1.511 & 0.018 & 1.016 & -0.031 & 0.455\\
$\alpha_{2}$ & 0.006 & 0.247 & 0.001 & 0.174 & -0.004 & 0.115 & 0.001 & 0.051\\
$\alpha_{3}$ & -0.004 & 0.108 & 0.003 & 0.074 & -0.001 & 0.050 & -0.001 & 0.022\\
$\gamma_{0}$ & -0.054 & 0.724 & -0.002 & 0.472 & -0.021 & 0.346 & -0.005 & 0.152\\
$\gamma_{1}$ & -0.098 & 2.620 & -0.093 & 1.847 & -0.081 & 1.303 & -0.003 & 0.548\\
$\gamma_{2}$ & 0.008 & 0.108 & -0.002 & 0.070 & 0.003 & 0.051 & 0.000 & 0.023\\
$\gamma_{3}$ & 0.001 & 0.107 & 0.003 & 0.075 & 0.003 & 0.053 & 0.000 & 0.022\\
$\theta$ & 0.023 & 0.258 & 0.014 & 0.197 & 0.014 & 0.135 & 0.003 & 0.058\\
Sample size ($T$) &  & 50 &  & 100 &  & 200 &  & 1000\\
\bottomrule
\end{tabular}
}\\
      \subfloat[$\sigma=1.0$]{
\begin{tabular}[t]{llrrrrrrr}
\toprule
  & Bias & RMSE & Bias & RMSE & Bias & RMSE & Bias & RMSE\\
\midrule
$\alpha_{0}$ & 0.012 & 1.015 & 0.012 & 0.636 & 0.001 & 0.446 & -0.016 & 0.188\\
$\alpha_{1}$ & -0.383 & 4.216 & -0.291 & 2.872 & 0.015 & 2.012 & 0.011 & 0.905\\
$\alpha_{2}$ & 0.042 & 0.466 & 0.033 & 0.310 & -0.001 & 0.227 & -0.006 & 0.100\\
$\alpha_{3}$ & 0.000 & 0.222 & 0.004 & 0.152 & 0.000 & 0.099 & -0.003 & 0.045\\
$\gamma_{0}$ & -0.233 & 1.612 & -0.067 & 1.020 & -0.076 & 0.703 & -0.004 & 0.308\\
$\gamma_{1}$ & 0.406 & 4.332 & -0.024 & 3.174 & -0.178 & 2.585 & -0.050 & 1.110\\
$\gamma_{2}$ & 0.025 & 0.231 & 0.008 & 0.154 & 0.010 & 0.103 & 0.000 & 0.045\\
$\gamma_{3}$ & 0.034 & 0.233 & 0.010 & 0.146 & 0.009 & 0.103 & 0.002 & 0.045\\
$\theta$ & 0.014 & 0.380 & 0.018 & 0.311 & 0.035 & 0.260 & 0.009 & 0.113\\
Sample size ($T$) &  & 50 &  & 100 &  & 200 &  & 1000\\
\bottomrule
\end{tabular}
}\\
    \subfloat[$\sigma=2.0$]{
\begin{tabular}[t]{llrrrrrrr}
\toprule
  & Bias & RMSE & Bias & RMSE & Bias & RMSE & Bias & RMSE\\
\midrule
$\alpha_{0}$ & -0.138 & 2.592 & 0.142 & 1.671 & 0.004 & 0.936 & 0.000 & 0.410\\
$\alpha_{1}$ & -0.986 & 11.090 & -0.665 & 6.325 & -0.174 & 4.042 & 0.003 & 1.788\\
$\alpha_{2}$ & 0.065 & 1.255 & 0.112 & 0.756 & 0.023 & 0.447 & 0.000 & 0.207\\
$\alpha_{3}$ & -0.006 & 0.589 & 0.019 & 0.345 & 0.003 & 0.224 & 0.003 & 0.092\\
$\gamma_{0}$ & -0.123 & 3.350 & -0.298 & 2.455 & -0.081 & 1.379 & -0.047 & 0.628\\
$\gamma_{1}$ & 0.635 & 7.424 & 0.141 & 5.668 & -0.081 & 4.255 & -0.040 & 2.159\\
$\gamma_{2}$ & 0.016 & 0.484 & 0.037 & 0.349 & 0.009 & 0.205 & 0.005 & 0.093\\
$\gamma_{3}$ & 0.009 & 0.517 & 0.026 & 0.336 & 0.000 & 0.208 & 0.007 & 0.092\\
$\theta$ & -0.029 & 0.446 & 0.025 & 0.414 & 0.034 & 0.381 & 0.016 & 0.220\\
Sample size ($T$) &  & 50 &  & 100 &  & 200 &  & 1000\\
\bottomrule
\end{tabular}
}
  \end{center}
  \footnotesize
  Note: The data generating process follows \cite{matsumura2023revisiting}.
\end{table} 


\begin{table}[!htbp]
  \begin{center}
      \caption{2SLS Results of the linear model}
      \label{tb:linear_linear_sigma_1} 
      \subfloat[$\sigma=0.5$]{
\begin{tabular}[t]{llrrrrrrr}
\toprule
  & Bias & RMSE & Bias & RMSE & Bias & RMSE & Bias & RMSE\\
\midrule
$\alpha_{0}$ & -0.018 & 0.465 & 0.007 & 0.323 & -0.008 & 0.213 & -0.006 & 0.097\\
$\alpha_{1}$ & -0.045 & 2.257 & 0.024 & 1.523 & 0.018 & 1.016 & -0.031 & 0.455\\
$\alpha_{2}$ & -0.001 & 0.255 & -0.001 & 0.176 & -0.004 & 0.115 & 0.001 & 0.051\\
$\alpha_{3}$ & -0.005 & 0.108 & 0.003 & 0.075 & -0.001 & 0.050 & -0.001 & 0.022\\
$\gamma_{0}$ & -0.061 & 0.732 & -0.005 & 0.474 & -0.021 & 0.346 & -0.005 & 0.152\\
$\gamma_{1}$ & -0.311 & 3.450 & -0.124 & 1.928 & -0.081 & 1.303 & -0.003 & 0.548\\
$\gamma_{2}$ & 0.009 & 0.109 & -0.001 & 0.071 & 0.003 & 0.051 & 0.000 & 0.023\\
$\gamma_{3}$ & 0.001 & 0.108 & 0.003 & 0.075 & 0.003 & 0.053 & 0.000 & 0.022\\
$\theta$ & 0.047 & 0.354 & 0.017 & 0.209 & 0.014 & 0.135 & 0.003 & 0.058\\
Sample size (n) &  & 50 &  & 100 &  & 200 &  & 1000\\
\bottomrule
\end{tabular}
}\\
      \subfloat[$\sigma=1.0$]{
\begin{tabular}[t]{llrrrrrrr}
\toprule
  & Bias & RMSE & Bias & RMSE & Bias & RMSE & Bias & RMSE\\
\midrule
$\alpha_{0}$ & -0.027 & 1.023 & -0.002 & 0.641 & -0.004 & 0.448 & -0.016 & 0.188\\
$\alpha_{1}$ & -0.024 & 4.396 & -0.169 & 2.965 & 0.061 & 2.060 & 0.011 & 0.905\\
$\alpha_{2}$ & -0.006 & 0.494 & 0.016 & 0.325 & -0.007 & 0.234 & -0.006 & 0.100\\
$\alpha_{3}$ & -0.006 & 0.223 & 0.002 & 0.153 & -0.001 & 0.099 & -0.003 & 0.045\\
$\gamma_{0}$ & -0.318 & 1.769 & -0.091 & 1.059 & -0.086 & 0.714 & -0.004 & 0.308\\
$\gamma_{1}$ & 5.859 & 210.853 & -0.679 & 6.280 & -0.338 & 2.972 & -0.050 & 1.110\\
$\gamma_{2}$ & 0.035 & 0.247 & 0.011 & 0.157 & 0.011 & 0.104 & 0.000 & 0.045\\
$\gamma_{3}$ & 0.045 & 0.250 & 0.012 & 0.150 & 0.010 & 0.104 & 0.002 & 0.045\\
$\theta$ & -0.399 & 18.450 & 0.098 & 0.738 & 0.054 & 0.308 & 0.009 & 0.113\\
Sample size (n) &  & 50 &  & 100 &  & 200 &  & 1000\\
\bottomrule
\end{tabular}
}\\
    \subfloat[$\sigma=2.0$]{
\begin{tabular}[t]{llrrrrrrr}
\toprule
  & Bias & RMSE & Bias & RMSE & Bias & RMSE & Bias & RMSE\\
\midrule
$\alpha_{0}$ & -0.263 & 2.596 & 0.071 & 1.670 & -0.040 & 0.947 & -0.002 & 0.412\\
$\alpha_{1}$ & -0.271 & 10.820 & 0.008 & 6.492 & 0.236 & 4.263 & 0.021 & 1.809\\
$\alpha_{2}$ & -0.044 & 1.253 & 0.023 & 0.779 & -0.031 & 0.483 & -0.003 & 0.210\\
$\alpha_{3}$ & -0.024 & 0.584 & 0.008 & 0.343 & -0.004 & 0.225 & 0.003 & 0.092\\
$\gamma_{0}$ & -2.074 & 19.624 & -0.551 & 3.043 & -0.171 & 1.516 & -0.051 & 0.633\\
$\gamma_{1}$ & 58.209 & 1750.688 & -2.416 & 56.909 & -3.617 & 39.044 & -0.103 & 2.334\\
$\gamma_{2}$ & 0.242 & 2.430 & 0.065 & 0.409 & 0.020 & 0.220 & 0.006 & 0.093\\
$\gamma_{3}$ & 0.230 & 2.328 & 0.055 & 0.404 & 0.010 & 0.219 & 0.008 & 0.092\\
$\theta$ & -6.668 & 233.851 & 0.372 & 6.334 & 0.418 & 3.820 & 0.024 & 0.245\\
Sample size ($T$) &  & 50 &  & 100 &  & 200 &  & 1000\\
\bottomrule
\end{tabular}
}
  \end{center}
  \footnotesize
  Note: The data generating process follows \cite{matsumura2023revisiting}.
\end{table} 

\end{document}