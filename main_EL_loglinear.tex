\documentclass[11pt, a4paper]{article}
\usepackage[utf8]{inputenc}
\usepackage{amsmath,setspace,geometry}
\usepackage{amsthm}
\usepackage{amsfonts}
\usepackage[shortlabels]{enumitem}
\usepackage{rotating}
\usepackage{pdflscape}
\usepackage{graphicx}
\usepackage{bbm}
\usepackage[dvipsnames]{xcolor}
\usepackage{hyperref}
\hypersetup{colorlinks=true, linkcolor= BrickRed, citecolor = BrickRed, filecolor = BrickRed, urlcolor = BrickRed, hypertexnames = true}
\usepackage[]{natbib} 
\bibpunct[:]{(}{)}{,}{a}{}{,}
\geometry{left = 1.0in,right = 1.0in,top = 1.0in,bottom = 1.0in}
\usepackage[english]{babel}
\usepackage{float}
\usepackage{caption}
\usepackage{subcaption}
\usepackage{booktabs}
\usepackage{pdfpages}
\usepackage{threeparttable}
\usepackage{lscape}
\usepackage{bm}
\setstretch{1.4}
%\usepackage[tablesfirst,nolists]{endfloat}

\newtheorem{theorem}{Theorem}
\newtheorem{assumption}{Assumption}
\newtheorem{lemma}{Lemma}
\newtheorem{definition}{Definition}
\newtheorem{proposition}{Proposition}
\newtheorem{claim}{Claim}
\newtheorem{corollary}{Corollary}
\newtheorem{example}{Example}
\DeclareMathOperator{\rank}{rank}


\title{An MPEC Estimator for Conduct Parameter Estimation in Homogeneous Goods Markets}
\author{Yuri Matsumura\thanks{Department of Economics, Rice University. Email: Yuri.Matsumura@rice.edu} \and Suguru Otani \thanks{Department of Economics, Rice University. Email: so19@rice.edu
%Declarations of interest: none %this is for Economics Letters
}}

\begin{document}

\maketitle
\begin{abstract}
    \textcolor{blue}{We propose a constrained maximum likelihood estimator for conduct parameter estimation in homogeneous goods markets, by formulating the estimation as an MPEC (Mathematical Programming with Equilibrium Constraints) problem. Our approach avoids the complex transformation within the equilibrium conditions and allows more complex specifications which could not be estimated by standard approaches. Monte Carlo simulations confirm that the proposed estimator reduces bias and standard deviation of the estimator for the typical specification, i.e., log-linear model.%, especially when the sample is small/medium.
    }
\end{abstract}


\section{Introduction}
Measuring competitiveness is one of the important tasks in empirical industrial organization literature.
Conduct parameter is considered to be a useful measure of competitiveness. 
However, it cannot be directly measured from data because data usually lack information about marginal cost.
Therefore, researchers endeavor to identify and estimate the conduct parameter.

As the simplest specification, \citet{bresnahan1982oligopoly} considers the identification of conduct parameter for the linear model. \cite{matsumura2023revisiting} resolves the conflict on identification problem between \cite{bresnahan1982oligopoly} and \cite{perloff2012collinearity} numerically and theoretically. On the other hand, researchers may want to implement more flexible specification such as log-linear model, e.g., \cite{okazaki2022excess} and \cite{merel2009measuring}. As for the log-linear model, the identification strategy is provided by \citet{lau1982identifying}. 
\textcolor{blue}{Estimation problems arise, however, when searching parameters using the standard solver because the equilibrium condition given by demand and supply curves involve log-transformation. This is the obstacle to the flexible specification of the demand and supply function.}


To overcome the problem, we propose a new estimator based on the mathematical program with equilibrium constraints (MPEC) approach advocated by \cite{su2012constrained} and \cite{dube2012improving}. MPEC is a constrained optimization problem whose constraint structure contains the equilibrium constraints. The basic idea is that we estimate the structural parameters by maximizing the likelihood of the data with the constraints that endogenous economic variables are consistent with an equilibrium for the structural parameters. To implement MPEC, researchers write down expressions that define the objective function and the equilibrium equations as constraints and solve the constrained optimization problem using optimization solvers. 

\textcolor{blue}{We show that MPEC estimator works well for the log-linear model.}

% There are two conflicting results regarding the conduct parameter estimation in homogeneous goods markets in linear demand and marginal cost systems.
% On the one hand, \citet{bresnahan1982oligopoly} proposes an approach for identifying the conduct parameter by using the demand rotation instrument.
% As the identification is guaranteed, the conduct parameter can be estimated using standard linear regression.
% This result is extended to nonlinear cases by \citet{lau1982identifying}
% and to the differentiated product markets by  \citet{nevoIdentificationOligopolySolution1998}.


% On the other hand, \citet{perloff2012collinearity} (hereafter, PS) assert that the linear model considered in \citet{bresnahan1982oligopoly} suffers from the multicollinearity problem when the error terms in the demand and supply equations are zero, implying that the identification of the conduct parameter is impossible.
% PS also use simulations to demonstrate that parameters cannot be estimated accurately even when the error terms are nonzero. 
% This is a major obstacle in the literature. 
% Several papers and handbook chapters reference the result in PS. See \citet{claessensWhatDrivesBank2004, coccoreseMultimarketContactCompetition2013, coccoreseWhatAffectsBank2021, garciaMarketStructuresProduction2020, kumbhakarNewMethodEstimating2012, perekhozhukRegionalLevelAnalysisOligopsony2015} and \citet{shafferMarketPowerCompetition2017}.

% We revisit the identification and estimation of the conduct parameter in homogeneous product markets to determine which result is correct.
% First, we show that the proof of the multicollinearity problem in PS is incorrect and that the multicollinearity problem does not occur under standard assumptions that reflect the insight in \citet{bresnahan1982oligopoly}.
% Second, the simulation in PS lacks an excluded demand shifter in the supply equation estimation, and we confirm that the accuracy of the estimation holds by properly including a demand shifter in the supply equation estimation. 
% We also show that increasing the sample size improves the accuracy of estimation. 
% Hence, we support \cite{bresnahan1982oligopoly} theoretically and numerically.

\section{Model}
The researcher has data with $T$ markets with homogeneous products.
Assume that there are $N$ firms in each market.
Let $t = 1,\ldots, T$ be the index of markets.
Then, we obtain the supply equation as follows:
\begin{align}
     P_t = -\theta\frac{\partial P_t(Q_{t})}{\partial Q_{t}}Q_{t} + MC_t(Q_{t}),\label{eq:supply_equation}
\end{align}
where $Q_{t}$ is the aggregate quantity, $P_t(Q_{t})$ is the demand function, $MC_{t}(Q_{t})$ is the marginal cost function, and $\theta\in[0,1]$, which is called conduct parameter. 
The equation nests perfect competition, $\theta=0$, Cournot competition, $\theta=1/N, N$ firm symmetric perfect collusion, $\theta=1$, etc.\footnote{See \cite{bresnahan1982oligopoly}.} 

Consider an econometric model of the above model.
Assume that the demand function and the marginal cost function are written as follows: 
\begin{align}
    P_t = f(Q_{t}, Y_t, \varepsilon^{d}_{t}, \alpha) \label{eq:demand}\\
    MC_t = g(Q_{t}, W_{t}, \varepsilon^{c}_{t}, \gamma)\label{eq:marginal_cost}
\end{align}
where $Y_t$ and $W_{t}$ are the vector of exogenous variables, $\varepsilon^{d}_{t}$ and $\varepsilon^{c}_{t}$ are the error terms, and $\alpha$ and $\gamma$ are the vector of parameters.
We also have the demand- and supply-side instrument variables $Z^{d}_{t}$ and $Z^{c}_{t}$, and assume that the error terms satisfy the mean independence condition $E[\varepsilon^{d}_{t}\mid Y_t, Z^{d}_{t}] = E[\varepsilon^{c}_{t} \mid W_{t}, Z^{c}_{t}] =0$.

\subsection{Log-linear demand and log-linear marginal cost}
We consider the most typical nonlinear specification as the log-linear model.
Assume that log-linear demand and cost functions are specified as:
\begin{align}
    \log P_{t} &= \alpha_0 - (\alpha_1 + \alpha_2 Z^{R}_{t}) \log Q_t + \alpha_3 \log Y_t + \varepsilon^{d}_{t},\label{eq:log_linear_demand}\\
    \log MC_t &= \gamma_0 + \gamma_1 \log Q_t +  \gamma_2 \log W_{t} + \gamma_3 \log R_t + \varepsilon^{c}_{t}.\label{eq:log_linear_marginal_cost}
\end{align}
Since $\partial P_t/\partial Q_t = - (\alpha_1 + \alpha_2 Z_{t}^R) (P_t/Q_t) $, Equation \eqref{eq:supply_equation} is written as:
\begin{align}
    P_t &= \theta (\alpha_1 + \alpha_2 Z^{R}_{t}) P_t + MC_t.
\end{align}
By reformulating this and taking logarithm, $\log P_t(1 -\theta (\alpha_1 + \alpha_2 Z^{R}_{t})) = \log MC_t.$
Then, we obtain:
\begin{align}
    \log P_t = - \log(1 - \theta(\alpha_1 + \alpha_2 Z^{R}_{t})) + \gamma_0 + \gamma_1 \log Q_t +  \gamma_2 \log W_{t} + \gamma_3 \log R_t + \varepsilon^{c}_{t}. \label{eq:log_linear_supply_equation}
\end{align}
By substituting the Equation \eqref{eq:log_linear_demand} into Equation \eqref{eq:log_linear_supply_equation} and solving it for $P_{t}$, the log aggregate quantity is given as: 
\begin{align}
    \log Q_t &= \frac{ \alpha_0 + \alpha_3 \log Y_t + \log (1 - \theta (\alpha_1 + \alpha_2 Z^{R}_{t})) - \gamma_0  -  \gamma_2 \log W_{t} - \gamma_3 \log R_t + \varepsilon^{d}_{t} - \varepsilon^{c}_{t}}{\gamma_1+ \alpha_1 + \alpha_2 Z^{R}_{t} }.\label{eq:quantity_loglinear}
\end{align}
\textcolor{blue}{An obstacle to estimate  parameters is the term $\log (1 - \theta (\alpha_1 + \alpha_2 Z^{R}_{t}))$. The standard numerical}

\section{The MPEC estimator for the conduct parameter model}

We propose a novel estimator for the log-linear conduct parameter model by utilizing the Mathematical Programming with Equilibrium Constraints (MPEC) procedure advocated by \cite{su2012constrained} and \cite{dube2012improving}. Let $\Theta$ be the set of parameters such that $\xi = (\alpha_0,\alpha_1, \ldots, \alpha_3, \gamma_0,\gamma_1, \ldots, \gamma_3, \theta)$. Then, we obtain $\xi$ by solving the following constrained optimization problem:
\begin{align}
    &\min_{\xi \in \Xi}\quad g(\xi)' W(\xi) g(\xi) \\
    \text{s.t.}\quad g(\xi)&=\left[\begin{array}{l}
\varepsilon^{d}_{t}Z_{t}^{d} \\
\varepsilon^{c}_{t}Z_{t}^{c}
\end{array}\right]\\
    \varepsilon^{d}_{t}(\xi)&=\log P_{t} -[\alpha_0 - (\alpha_1 + \alpha_2 Z^{R}_{t}) \log Q_t + \alpha_3 \log Y_t] \\
    \varepsilon^{c}_{t} (\xi)&= \log MC_t -[\gamma_0 + \gamma_1 \log Q_t +  \gamma_2 \log W_{t} + \gamma_3 \log R_t]\\
    MC_t &= P_t(1 - \theta (\alpha_1 + \alpha_2 Z^{R}_{t})),\\
    W(\xi)&=Z_{t}'Z_{t}, \quad Z_{t}=\left[\begin{array}{ll}
Z_{t}^{d} & 0 \\
0 & Z_{t}^{c}
\end{array}\right]
\end{align}
where $W(\xi)$ is the weighting matrix. 



\subsection{Simulation and estimation Procedure}

We assess the performance of the MPEC estimator using Monte Carlo simulation.
We set the true parameters and distributions as in Table \ref{tb:parameter_setting}. 
For the simulation, we generate 1000 data sets. We jointly estimate the demand and supply equation by simultaneous equation model.
The instrument variables for the demand estimation are $Z^{d}_{t} = (Z^{R}_{t}, Y_t, H_{t}, K_{t})$ and the instrument variables for the supply estimation are $Z^{c}_{t} = (Z^{R}_{t}, W_{t}, R_t, Y_t)$. 


\begin{table}[!htbp]
    \caption{True parameters and distributions}
    \label{tb:parameter_setting}
    \begin{center}
    \subfloat[Parameters]{
    \begin{tabular}{cr}
            \hline
            & log-linear \\
            $\alpha_0$  & $10.0$ \\
            $\alpha_1$ & $1.0$  \\
            $\alpha_2$ & $0.1$ \\
            $\alpha_3$ & $1.0$ \\
            $\gamma_0$ & $1.0$  \\
            $\gamma_1$ & $1.0$  \\
            $\gamma_2$ & $1.0$ \\
            $\gamma_3$ & $1.0$ \\
            $\theta$ & $0.2$  \\
            \hline
        \end{tabular}
    }
    \subfloat[Distributions]{
    \begin{tabular}{crr}
            \hline
            &  log-linear \\
            Demand shifter&  &  \\
            $Y_t$ & $N(0,1)$ \\
            Demand rotation instrument&  &  \\
            $Z^{R}_{t}$ & $U(0,1)$ \\
            Cost shifter  &  \\
            $W_{t}$ & $U(1,3)$ \\
            $R_t$  & $U(1,3)$  \\
            $H_{t}$ & $W_{t}+U(0,1)$  \\
            $K_{t}$ & $R_{t}+U(0,1)$  \\
            Error&  &  \\
            $\varepsilon^{d}_{t}$ & $N(0,\sigma)$  \\
            $\varepsilon^{c}_{t}$ & $N(0,\sigma)$ \\
            \hline
        \end{tabular}
    }
    \end{center}
    \footnotesize
    Note: $\sigma=\{0.001, 0.5, 2.0\}$. $N:$ Normal distribution. $U:$ Uniform distribution.
\end{table}



%581words by Suguru

\section{Simulation results}\label{sec:results}

Table \ref{tb:loglinear_loglinear_sigma_2_mpec_non_constraint_theta_constraint} presents that [TBA]


\begin{table}[!htbp]
  \begin{center}
      \caption{MPEC Results of the log-linear model}
      \label{tb:loglinear_loglinear_sigma_2_mpec_non_constraint_theta_constraint} 
      \subfloat[$\sigma=0.001$]{
\begin{tabular}[t]{lrrrrrrrr}
\toprule
  & Mean & SD & Mean  & SD  & Mean   & SD   & Mean    & SD   \\
\midrule
$\alpha_{0}$ & 1.176 & 0.854 & 0.721 & 0.880 & 0.284 & 0.656 & 0.000 & 0.000\\
$\alpha_{1}$ & 0.660 & 0.487 & 0.403 & 0.494 & 0.158 & 0.364 & 0.000 & 0.000\\
$\alpha_{2}$ & 0.059 & 0.064 & 0.038 & 0.053 & 0.015 & 0.037 & 0.000 & 0.000\\
$\alpha_{3}$ & 1.221 & 0.890 & 0.748 & 0.914 & 0.292 & 0.675 & 0.000 & 0.000\\
$\gamma_{0}$ & -0.365 & 1.568 & -0.103 & 0.971 & -0.024 & 0.522 & 0.000 & 0.000\\
$\gamma_{1}$ & 0.359 & 0.269 & 0.224 & 0.276 & 0.088 & 0.203 & 0.000 & 0.000\\
$\gamma_{2}$ & 0.359 & 0.271 & 0.224 & 0.277 & 0.089 & 0.206 & 0.000 & 0.000\\
$\gamma_{3}$ & 0.409 & 0.306 & 0.256 & 0.316 & 0.102 & 0.238 & 0.000 & 0.000\\
$\theta$ & 0.213 & 0.276 & 0.105 & 0.196 & 0.047 & 0.122 & 0.010 & 0.000\\
Sample size ($T$) &  & 50 &  & 100 &  & 200 &  & 1000\\
\bottomrule
\end{tabular}
}\\
      \subfloat[$\sigma=0.5$]{
\begin{tabular}[t]{lrrrrrrrr}
\toprule
  & Mean & SD & Mean  & SD  & Mean   & SD   & Mean    & SD   \\
\midrule
$\alpha_{0}$ & 9.682 & 1.362 & 9.776 & 1.551 & 9.849 & 0.849 & 9.827 & 0.516\\
$\alpha_{1}$ & 0.831 & 0.801 & -7.913 & 278.438 & 0.909 & 0.492 & 0.897 & 0.304\\
$\alpha_{2}$ & 0.071 & 0.151 & 0.045 & 1.608 & 0.103 & 0.071 & 0.101 & 0.032\\
$\alpha_{3}$ & 0.921 & 0.499 & 0.943 & 0.430 & 0.956 & 0.271 & 0.950 & 0.155\\
$\gamma_{0}$ & 4.059 & 3.196 & 4.676 & 1.608 & 4.863 & 1.012 & 5.023 & 0.431\\
$\gamma_{1}$ & 1.087 & 1.003 & 1.039 & 0.402 & 1.021 & 0.252 & 1.003 & 0.105\\
$\gamma_{2}$ & 1.042 & 0.526 & 1.021 & 0.266 & 1.013 & 0.167 & 1.002 & 0.074\\
$\gamma_{3}$ & 1.048 & 0.530 & 1.022 & 0.272 & 1.011 & 0.178 & 1.001 & 0.077\\
$\theta$ & 0.367 & 0.384 & 0.263 & 0.332 & 0.248 & 0.292 & 0.201 & 0.214\\
Sample size ($T$) &  & 50 &  & 100 &  & 200 &  & 1000\\
\bottomrule
\end{tabular}
}\\
    \subfloat[$\sigma=2.0$]{
\begin{tabular}[t]{lrrrrrrrr}
\toprule
  & Mean & SD & Mean  & SD  & Mean   & SD   & Mean    & SD   \\
\midrule
$\alpha_{0}$ & 18.669 & 5.381 & 19.273 & 9.797 & 19.642 & 7.592 & 20.138 & 4.439\\
$\alpha_{1}$ & 0.823 & 0.850 & 0.907 & 1.548 & 0.954 & 1.143 & 1.023 & 0.679\\
$\alpha_{2}$ & 0.040 & 0.137 & 0.056 & 0.115 & 0.070 & 0.138 & 0.096 & 0.061\\
$\alpha_{3}$ & 0.887 & 1.052 & 0.947 & 0.897 & 0.937 & 0.795 & 0.995 & 0.409\\
$\gamma_{0}$ & -1.632 & 27.771 & -2.358 & 22.821 & -0.736 & 17.290 & 3.476 & 4.639\\
$\gamma_{1}$ & 1.381 & 3.979 & 1.513 & 3.119 & 1.441 & 2.320 & 1.139 & 0.574\\
$\gamma_{2}$ & 1.020 & 3.757 & 1.167 & 2.042 & 1.187 & 1.337 & 1.074 & 0.353\\
$\gamma_{3}$ & 1.078 & 2.961 & 1.263 & 2.362 & 1.228 & 1.478 & 1.070 & 0.376\\
$\theta$ & 0.748 & 0.391 & 0.749 & 0.386 & 0.720 & 0.389 & 0.598 & 0.345\\
Sample size ($T$) &  & 50 &  & 100 &  & 200 &  & 1000\\
\bottomrule
\end{tabular}
}
  \end{center}
  \footnotesize
  Note: 
\end{table} 



\section{Conclusion}
[TBA]
% We revisit the conduct parameter estimation in homogeneous goods markets.
% There is a conflict between \citet{bresnahan1982oligopoly} and \citet{perloff2012collinearity} in terms of identification and estimation.
% We highlight the problems in the proof and simulation in \citet{perloff2012collinearity}.
% Our simulation shows that the estimation of the conduct parameter becomes accurate by appropriately introducing demand shifters into the supply estimation and increasing the sample size. 
% Based on the theoretical and numerical investigation, we support the argument in \citet{bresnahan1982oligopoly}.


\paragraph{Acknowledgments}
We thank Jeremy Fox and Isabelle Perrigne for their valuable advice. This research did not receive any specific grant from funding agencies in the public, commercial, or not-for-profit sectors. 

\newpage

\bibliographystyle{aer}
\bibliography{conduct_parameter}

\newpage
\appendix

\section{MPEC for a linear model}

We illustrate that MPEC works for a linear model as in Two-Stage-Least-Square (2SLS) approach in \cite{matsumura2023revisiting}. 
% We confirm that the simultaneous equation model in which demand and supply parameters are jointly estimated and 2SLS model in which demand and supply parameters are separately estimated generate the similar results. 
We follow the setting of \cite{matsumura2023revisiting}.
We put an additional restriction such that $\theta\in[0,1]$ as a theoretical restriction. Table \ref{tb:linear_linear_sigma_2_mpec_linear_non_constraint_theta_constraint_bias_rmse} shows that MPEC estimator works well.



\begin{table}[!htbp]
  \begin{center}
      \caption{MPEC Results of the linear model}
      \label{tb:linear_linear_sigma_2_mpec_linear_non_constraint_theta_constraint_bias_rmse} 
      \subfloat[$\sigma=0.001$]{
\begin{tabular}[t]{llrrrrrrr}
\toprule
  & Bias & RMSE & Bias & RMSE & Bias & RMSE & Bias & RMSE\\
\midrule
$\alpha_{0}$ & 0.000 & 0.001 & 0.000 & 0.001 & 0.000 & 0.000 & 0.000 & 0.000\\
$\alpha_{1}$ & 0.000 & 0.004 & 0.000 & 0.003 & 0.000 & 0.002 & 0.000 & 0.001\\
$\alpha_{2}$ & 0.000 & 0.000 & 0.000 & 0.000 & 0.000 & 0.000 & 0.000 & 0.000\\
$\alpha_{3}$ & 0.000 & 0.000 & 0.000 & 0.000 & 0.000 & 0.000 & 0.000 & 0.000\\
$\gamma_{0}$ & 0.000 & 0.001 & 0.000 & 0.001 & 0.000 & 0.001 & 0.000 & 0.000\\
$\gamma_{1}$ & 0.000 & 0.005 & 0.000 & 0.004 & 0.000 & 0.002 & 0.000 & 0.001\\
$\gamma_{2}$ & 0.000 & 0.000 & 0.000 & 0.000 & 0.000 & 0.000 & 0.000 & 0.000\\
$\gamma_{3}$ & 0.000 & 0.000 & 0.000 & 0.000 & 0.000 & 0.000 & 0.000 & 0.000\\
$\theta$ & 0.000 & 0.001 & 0.000 & 0.000 & 0.000 & 0.000 & 0.000 & 0.000\\
Sample size ($T$) &  & 50 &  & 100 &  & 200 &  & 1000\\
\bottomrule
\end{tabular}
}\\
      \subfloat[$\sigma=0.5$]{
\begin{tabular}[t]{llrrrrrrr}
\toprule
  & Bias & RMSE & Bias & RMSE & Bias & RMSE & Bias & RMSE\\
\midrule
$\alpha_{0}$ & -0.013 & 0.462 & 0.008 & 0.322 & -0.008 & 0.213 & -0.006 & 0.097\\
$\alpha_{1}$ & -0.096 & 2.201 & 0.015 & 1.511 & 0.018 & 1.016 & -0.031 & 0.455\\
$\alpha_{2}$ & 0.006 & 0.247 & 0.001 & 0.174 & -0.004 & 0.115 & 0.001 & 0.051\\
$\alpha_{3}$ & -0.004 & 0.108 & 0.003 & 0.074 & -0.001 & 0.050 & -0.001 & 0.022\\
$\gamma_{0}$ & -0.054 & 0.724 & -0.002 & 0.472 & -0.021 & 0.346 & -0.005 & 0.152\\
$\gamma_{1}$ & -0.098 & 2.620 & -0.093 & 1.847 & -0.081 & 1.303 & -0.003 & 0.548\\
$\gamma_{2}$ & 0.008 & 0.108 & -0.002 & 0.070 & 0.003 & 0.051 & 0.000 & 0.023\\
$\gamma_{3}$ & 0.001 & 0.107 & 0.003 & 0.075 & 0.003 & 0.053 & 0.000 & 0.022\\
$\theta$ & 0.023 & 0.258 & 0.014 & 0.197 & 0.014 & 0.135 & 0.003 & 0.058\\
Sample size ($T$) &  & 50 &  & 100 &  & 200 &  & 1000\\
\bottomrule
\end{tabular}
}\\
    \subfloat[$\sigma=2.0$]{
\begin{tabular}[t]{llrrrrrrr}
\toprule
  & Bias & RMSE & Bias & RMSE & Bias & RMSE & Bias & RMSE\\
\midrule
$\alpha_{0}$ & -0.138 & 2.592 & 0.142 & 1.671 & 0.004 & 0.936 & 0.000 & 0.410\\
$\alpha_{1}$ & -0.986 & 11.090 & -0.665 & 6.325 & -0.174 & 4.042 & 0.003 & 1.788\\
$\alpha_{2}$ & 0.065 & 1.255 & 0.112 & 0.756 & 0.023 & 0.447 & 0.000 & 0.207\\
$\alpha_{3}$ & -0.006 & 0.589 & 0.019 & 0.345 & 0.003 & 0.224 & 0.003 & 0.092\\
$\gamma_{0}$ & -0.123 & 3.350 & -0.298 & 2.455 & -0.081 & 1.379 & -0.047 & 0.628\\
$\gamma_{1}$ & 0.635 & 7.424 & 0.141 & 5.668 & -0.081 & 4.255 & -0.040 & 2.159\\
$\gamma_{2}$ & 0.016 & 0.484 & 0.037 & 0.349 & 0.009 & 0.205 & 0.005 & 0.093\\
$\gamma_{3}$ & 0.009 & 0.517 & 0.026 & 0.336 & 0.000 & 0.208 & 0.007 & 0.092\\
$\theta$ & -0.029 & 0.446 & 0.025 & 0.414 & 0.034 & 0.381 & 0.016 & 0.220\\
Sample size ($T$) &  & 50 &  & 100 &  & 200 &  & 1000\\
\bottomrule
\end{tabular}
}
  \end{center}
  \footnotesize
  Note: The data generating process follows \cite{matsumura2023revisiting}.
\end{table} 




\end{document}