\documentclass[11pt, a4paper]{article}
\usepackage[utf8]{inputenc}
\usepackage{amsmath,setspace,geometry}
\usepackage{amsthm}
\usepackage{amsfonts}
\usepackage[shortlabels]{enumitem}
\usepackage{rotating}
\usepackage{pdflscape}
\usepackage{graphicx}
\usepackage{bbm}
\usepackage[dvipsnames]{xcolor}
\usepackage[colorlinks=true, linkcolor= BrickRed, citecolor = BrickRed, filecolor = BrickRed, urlcolor = BrickRed, hypertexnames = true]{hyperref}
\usepackage[]{natbib} 
\bibpunct[:]{(}{)}{,}{a}{}{,}
\geometry{left = 1.0in,right = 1.0in,top = 1.0in,bottom = 1.0in}
\usepackage[english]{babel}
\usepackage{float}
\usepackage{caption}
\usepackage{subcaption}
\usepackage{booktabs}
\usepackage{pdfpages}
\usepackage{threeparttable}
\usepackage{lscape}
\usepackage{bm}
\setstretch{1.4}
%\usepackage[tablesfirst,nolists]{endfloat}

\newtheorem{theorem}{Theorem}
\newtheorem{assumption}{Assumption}
\newtheorem{lemma}{Lemma}
\newtheorem{definition}{Definition}
\newtheorem{proposition}{Proposition}
\newtheorem{claim}{Claim}
\newtheorem{corollary}{Corollary}
\newtheorem{example}{Example}
\DeclareMathOperator{\rank}{rank}


\title{Resolving A Conflict on Conduct Parameter Estimation in Homogeneous Good Markets between Bresnahan (1982) and Perloff and Shen (2012)}
\author{Yuri Matsumura\footnote{Department of Economics, Rice University. Matsumura: \texttt{\href{mailto:Yuri.Matsumura@rice.edu}{Yuri.Matsumura@rice.edu}}} \and Suguru Otani \footnote{Department of Economics, Rice University. Otani: \texttt{\href{mailto:so19@rice.edu}{so19@rice.edu}}\\
Declarations of interest: none}}

\begin{document}

\maketitle

\begin{abstract}
    In this study, we reexamine the estimation of conduct parameters in homogeneous goods markets. Specifically, we highlight an error in the proof presented by \cite{perloff2012collinearity}. Unlike the simulation results of linear models in \cite{perloff2012collinearity}, which were pessimistic, our simulations demonstrate that accurate estimation is achievable by incorporating demand shifters in the supply estimation and increasing the sample size. Based on our numerical investigation, we have found that the linear model can, at a minimum, provide a proper estimation of the conduct parameter.
\end{abstract}


\section{Introduction}
Measuring competitiveness is an essential task in Empirical Industrial Organization literature, and the conduct parameter is a useful measure of competitiveness. 
However, since data usually lack information about marginal cost, it is not possible to measure the conduct parameter directly. 
Therefore, researchers have attempted to identify and estimate the conduct parameter.

There are two conflicting results regarding the estimation of the conduct parameter in homogeneous good markets in linear demand and marginal cost systems. 
On the one hand, \cite{bresnahan1982oligopoly} demonstrates the identification of the conduct parameter using an instrument called the demand rotation instrument.
By using this instrument, we can estimate the conduct parameter through standard linear regression.

On the other hand, \cite{perloff2012collinearity} (PS) point out that the linear model considered in  \cite{bresnahan1982oligopoly} suffers from a multicollinearity problem when the error terms in the demand and supply equations are zero, making the identification of the conduct parameter impossible. 
PS also show through simulations that the parameters cannot be accurately estimated even when the error terms are not zero, which is a significant obstacle in the literature.\footnote{Several papers and handbook chapter mention the multicollinearity problem in the conduct parameter estimation by citing PS. See \citet{claessensWhatDrivesBank2004, coccoreseMultimarketContactCompetition2013, coccoreseWhatAffectsBank2021, garciaMarketStructuresProduction2020, kumbhakarNewMethodEstimating2012, perekhozhukRegionalLevelAnalysisOligopsony2015} and \citet{shafferMarketPowerCompetition2017}.}

To determine which result is correct, we revisit the identification and estimation of the conduct parameter in homogeneous product markets. 
Firstly, we demonstrate that the proof of the multicollinearity problem in PS is incorrect and that the multicollinearity problem does not occur under standard assumptions that reflect the insight in \cite{bresnahan1982oligopoly}.
Secondly, the simulation in PS lacks an excluded demand shifter in the supply equation estimation. 
By including a demand shifter in the supply equation estimation properly, we confirm that the accuracy of the estimation holds. 
We also show that increasing the sample size improves the accuracy of estimation. 
Therefore, we support \cite{bresnahan1982oligopoly} theoretically and numerically.


\section{Model}
The researcher has data with $T$ markets with homogeneous products.
Assume that there are $N$ firms in each market.
Let $t = 1,\ldots, T$ be the index of markets.
Then, we obtain the supply equation:
\begin{align}
     P_t = -\theta\frac{\partial P_t(Q_{t})}{\partial Q_{t}}Q_{t} + MC_t(Q_{t}),\label{eq:supply_equation}
\end{align}
where $Q_{t}$ is the aggregate quantity, $P_t(Q_{t})$ the demand function, $MC_{t}(Q_{t})$ the marginal cost function, and $\theta\in[0,1]$ which is called conduct parameter. 
The equation nests perfect competition, $\theta=0$, Cournot competition, $\theta=1/N, N$ firm symmetric perfect collusion, $\theta=1$, etc.\footnote{See \cite{bresnahan1982oligopoly}.} 

Consider an econometric model of the above model.
Assume that the demand function and the marginal cost function are written as: 
\begin{align}
    P_t = f(Q_{t}, Y_t, \varepsilon^{d}_{t}, \alpha) \label{eq:demand}\\
    MC_t = g(Q_{t}, W_{t}, \varepsilon^{c}_{t}, \gamma)\label{eq:marginal_cost}
\end{align}
where $Y_t$ and $W_{t}$ are the vector of exogenous variables, $\varepsilon^{d}_{t}$ and $\varepsilon^{c}_{t}$ the error terms, and $\alpha$ and $\gamma$ are the vector of parameters.
We also have the demand- and supply-side instrument variables $Z^{d}_{t}$ and $Z^{c}_{t}$ and assume that the error terms satisfy the mean independence condition $E[\varepsilon^{d}_{t}\mid Y_t, Z^{d}_{t}] = E[\varepsilon^{c}_{t} \mid W_{t}, Z^{c}_{t}] =0$.

\subsection{Linear demand and linear cost}
Assume that linear demand and cost functions are specified as:
\begin{align}
    P_t &= \alpha_0 - (\alpha_1 + \alpha_2Z^{R}_{t})Q_{t} + \alpha_3 Y_t + \varepsilon^{d}_{t},\label{eq:linear_demand}\\
    MC_t &= \gamma_0  + \gamma_1 Q_{t} + \gamma_2 W_{t} + \gamma_3 R_{t} + \varepsilon^{c}_{t},\label{eq:linear_marginal_cost}
\end{align}
where $W_{t}$ and $R_{t}$ are excluded cost shifters and $Z^{R}_{t}$ is Bresnahan's demand rotation instrument. 
The supply equation is written as:
\begin{align}
    P_t 
    %&= \gamma_0 + [\theta(\alpha_1 + \alpha_2Z^{R}_{t})+ \gamma_1] Q_{t}   + \gamma_2 W_{t} + \gamma_3 R_{t} + \varepsilon^{c}_{t}\nonumber\\ 
    &= \gamma_0 + \theta \alpha_2 Z^{R}_tQ_{t} + (\theta\alpha_1 + \gamma_1) Q_{t} + \gamma_2 W_t + \gamma_3 R_{t} +\varepsilon^c_t\label{eq:linear_supply_equation}
\end{align}
By substituting Equation \eqref{eq:linear_demand} into Equation \eqref{eq:linear_supply_equation} and solving it for $P_t$, we can represent the aggregate quantity $Q_{t}$ based on the parameters and exogenous variables as:
\begin{align}
    Q_{t} =  \frac{\alpha_0 + \alpha_3 Y_t - \gamma_0 - \gamma_2 W_{t} - \gamma_3 R_{t} + \varepsilon^{d}_{t} - \varepsilon^{c}_{t}}{(1 + \theta) (\alpha_1 + \alpha_2 Z^{R}_{t}) + \gamma_1}.\label{eq:quantity_linear}
\end{align}

\subsection{Multicollinearity problem in PS is incorrect?}
To see the multicollinearity problem, PS try to show linear dependence between the variables in the supply equations. 
PS start the proof by stating the following in p137 in their appendix (we modify notations);
\begin{quote}
    ``We demonstrate that the $W_{t}, R_{t}, Z^{R}_{t}Q_{t}$, and $Q_{t}$ terms in Eq.4 are perfectly collinear for $\varepsilon_{t}^{d} = \varepsilon_{t}^{c} = 0$. We show this result by demonstrating that there exist nonzero coefficients $\chi_1,\chi_2,\chi_3,\chi_4$, and $\chi_5$ such that 
   \begin{align*}
    Z^{R}_{t} Q_{t} + \chi_1 Q_{t} + \chi_2 W_{t} + \chi_3 R_{t} + \chi_4 Y_{t} + \chi_5 = 0 \quad (\text{A1})."
    \end{align*}
\end{quote}
Eq.4 in the quotation corresponds to the supply equation \eqref{eq:linear_supply_equation}.
They show that there exists a nonzero vector of $\chi_1, \ldots, \chi_5$ that satisfies (A1).

An incorrect point in the proof is that while they want to show the linear dependence between $Z^{R}_{t}Q_{t}, Q_{t}, W_{t}$, and $R_{t}$, they show the linear dependence between $W_{t}, R_{t}, Z^{R}_{t}Q_{t}, Q_{t}$, and $Y_t$. 
However, the linear dependence between $W_{t}, R_{t}, Z^{R}_{t}Q_{t}, Q_{t}$, and $Y_t$ does not always imply the linear dependence between $Z^{R}_{t}Q_{t}, Q_{t}, W_{t}$, and $R_{t}$.

We formally state that the multicollinearity problem does not occur under the additional standard assumptions in Proposition 1.
\begin{proposition}
    Assume that (i) $\alpha_2$ and $\alpha_3$ are nonzero and (ii) $Z^R_t, W_t, R_t$, and $Y_t$ are linearly independent.
    Then, $Z^{R}_{t}Q_{t}, Q_{t}, W_{t}$, and $R_{t}$ are linearly independent.
\end{proposition}

See Appendix \ref{sec:appendix} for the proof.
Assumption (i) implies that when the demand rotation instrument and the demand shifter shift the demand equation, we can identify the conduct parameter.
This reflects the main result in \citet{bresnahan1982oligopoly}.
Assumption (ii) is standard in the regression model but not assumed in PS.


\section{Simulation results}\label{sec:results}

Table \ref{tb:linear_linear_sigma_1} presents the results of the linear model with demand shifters.
\footnote{See Appendix \ref{sec:appendix} for simulation details and additional results.} 
Panel (a) demonstrates that when the standard deviation of the error terms in the demand and supply equation is $\sigma = 0.001$, the estimation of all parameters is highly accurate. 
The Root-mean-squared errors (RMSEs) of all parameters are less than or equal to 0.001, even with large sample sizes. 
In contrast, Panel (c) shows the case with $\sigma = 2.0$, where the RMSE decreases dramatically as the sample size increases. 
Therefore, we can conclude that the imprecise results reported in PS are due to the absence of demand shifters and the small sample size.



\begin{table}[!htbp]
  \begin{center}
      \caption{Results of the linear model with demand shifter}
      \label{tb:linear_linear_sigma_1} 
      \subfloat[$\sigma=0.001$]{
\begin{tabular}[t]{llrrrrrrr}
\toprule
  & Bias & RMSE & Bias & RMSE & Bias & RMSE & Bias & RMSE\\
\midrule
$\alpha_{0}$ & 0.000 & 0.001 & 0.000 & 0.001 & 0.000 & 0.000 & 0.000 & 0.000\\
$\alpha_{1}$ & 0.000 & 0.004 & 0.000 & 0.003 & 0.000 & 0.002 & 0.000 & 0.001\\
$\alpha_{2}$ & 0.000 & 0.000 & 0.000 & 0.000 & 0.000 & 0.000 & 0.000 & 0.000\\
$\alpha_{3}$ & 0.000 & 0.000 & 0.000 & 0.000 & 0.000 & 0.000 & 0.000 & 0.000\\
$\gamma_{0}$ & 0.000 & 0.001 & 0.000 & 0.001 & 0.000 & 0.001 & 0.000 & 0.000\\
$\gamma_{1}$ & 0.000 & 0.005 & 0.000 & 0.004 & 0.000 & 0.002 & 0.000 & 0.001\\
$\gamma_{2}$ & 0.000 & 0.000 & 0.000 & 0.000 & 0.000 & 0.000 & 0.000 & 0.000\\
$\gamma_{3}$ & 0.000 & 0.000 & 0.000 & 0.000 & 0.000 & 0.000 & 0.000 & 0.000\\
$\theta$ & 0.000 & 0.001 & 0.000 & 0.000 & 0.000 & 0.000 & 0.000 & 0.000\\
Sample size (n) &  & 50 &  & 100 &  & 200 &  & 1000\\
\bottomrule
\end{tabular}
}\\
      \subfloat[$\sigma=0.5$]{
\begin{tabular}[t]{llrrrrrrr}
\toprule
  & Bias & RMSE & Bias & RMSE & Bias & RMSE & Bias & RMSE\\
\midrule
$\alpha_{0}$ & -0.018 & 0.465 & 0.007 & 0.323 & -0.008 & 0.213 & -0.006 & 0.097\\
$\alpha_{1}$ & -0.045 & 2.257 & 0.024 & 1.523 & 0.018 & 1.016 & -0.031 & 0.455\\
$\alpha_{2}$ & -0.001 & 0.255 & -0.001 & 0.176 & -0.004 & 0.115 & 0.001 & 0.051\\
$\alpha_{3}$ & -0.005 & 0.108 & 0.003 & 0.075 & -0.001 & 0.050 & -0.001 & 0.022\\
$\gamma_{0}$ & -0.061 & 0.732 & -0.005 & 0.474 & -0.021 & 0.346 & -0.005 & 0.152\\
$\gamma_{1}$ & -0.311 & 3.450 & -0.124 & 1.928 & -0.081 & 1.303 & -0.003 & 0.548\\
$\gamma_{2}$ & 0.009 & 0.109 & -0.001 & 0.071 & 0.003 & 0.051 & 0.000 & 0.023\\
$\gamma_{3}$ & 0.001 & 0.108 & 0.003 & 0.075 & 0.003 & 0.053 & 0.000 & 0.022\\
$\theta$ & 0.047 & 0.354 & 0.017 & 0.209 & 0.014 & 0.135 & 0.003 & 0.058\\
Sample size (n) &  & 50 &  & 100 &  & 200 &  & 1000\\
\bottomrule
\end{tabular}
}\\
    \subfloat[$\sigma=2.0$]{
\begin{tabular}[t]{llrrrrrrr}
\toprule
  & Bias & RMSE & Bias & RMSE & Bias & RMSE & Bias & RMSE\\
\midrule
$\alpha_{0}$ & -0.263 & 2.596 & 0.071 & 1.670 & -0.040 & 0.947 & -0.002 & 0.412\\
$\alpha_{1}$ & -0.271 & 10.820 & 0.008 & 6.492 & 0.236 & 4.263 & 0.021 & 1.809\\
$\alpha_{2}$ & -0.044 & 1.253 & 0.023 & 0.779 & -0.031 & 0.483 & -0.003 & 0.210\\
$\alpha_{3}$ & -0.024 & 0.584 & 0.008 & 0.343 & -0.004 & 0.225 & 0.003 & 0.092\\
$\gamma_{0}$ & -2.074 & 19.624 & -0.551 & 3.043 & -0.171 & 1.516 & -0.051 & 0.633\\
$\gamma_{1}$ & 58.209 & 1750.688 & -2.416 & 56.909 & -3.617 & 39.044 & -0.103 & 2.334\\
$\gamma_{2}$ & 0.242 & 2.430 & 0.065 & 0.409 & 0.020 & 0.220 & 0.006 & 0.093\\
$\gamma_{3}$ & 0.230 & 2.328 & 0.055 & 0.404 & 0.010 & 0.219 & 0.008 & 0.092\\
$\theta$ & -6.668 & 233.851 & 0.372 & 6.334 & 0.418 & 3.820 & 0.024 & 0.245\\
Sample size ($T$) &  & 50 &  & 100 &  & 200 &  & 1000\\
\bottomrule
\end{tabular}
}
  \end{center}
  \footnotesize
  Note: The error terms in the demand and supply equation are drawn from a normal distribution, $N(0,\sigma)$.
\end{table} 



\section{Conclusion}
We reassess the estimation of the conduct parameter in homogeneous goods markets, which has been debated between \citet{bresnahan1982oligopoly} and \citet{perloff2012collinearity}. 
We identify issues in the proof and simulation presented by \citet{perloff2012collinearity}. 
Our simulations demonstrate that accurate estimation of the conduct parameter can be achieved by including demand shifters in the supply estimation and increasing the sample size. 
Our findings support the conclusions drawn by \citet{bresnahan1982oligopoly} through both theoretical and numerical analysis.


\paragraph{Acknowledgments}
We thank Jeremy Fox and Isabelle Perrigne for their valuable advice. 
This research did not receive any specific grant from funding agencies in the public, commercial, or not-for-profit sectors. 

\newpage

\bibliographystyle{aer}
\bibliography{conduct_parameter}

\newpage
\appendix


% \section{Corrected proof of \cite{perloff2012collinearity}}\label{sec:corrected_proof_of_PS}

% To see the multicollinearity problem, they try to show linear dependence between the variables in the supply equations. 
% \cite{perloff2012collinearity} start the proof by saying the following in p137 in their appendix (we modify notations);
% \begin{quote}
%     "We demonstrate that the $W_{t}, R_{t}, Z^{R}_{t}Q_{t}$, and $Q_{t}$ terms in Eq.4 are perfectly collinear for $\varepsilon_{t}^{d} = \varepsilon_{t}^{c} = 0$. We show this result by demonstrating that there exist nonzero coefficients $\chi_1,\chi_2,\chi_3,\chi_4$, and $\chi_5$ such that 
%     \[Z^{R}_{t} Q_{t} + \chi_1 Q_{t} + \chi_2 W_{t} + \chi_3 R_{t} + \chi_4 Y_{t} + \chi_5 = 0.\quad \text{(A1)}"\]
% \end{quote}
% Eq.4 in the quotation corresponds to the supply equation \eqref{eq:linear_supply_equation}.
% They show that there exists a nonzero vector of $\chi_1, \ldots, \chi_5$ that satisfies (A1). 
% Although (A1) is incorrect, we replicate the flow of their proof by fixing several typos.
% \begin{proof}
%     First, by substituting the equilibrium quantity with $\varepsilon^{d}_{t} = \varepsilon^{c}_{t} = 0$,
%     \begin{align*}
%         Q_{t} =  \frac{\alpha_0 + \alpha_3 Y_t - \gamma_0 - \gamma_2 W_{t} - \gamma_3 R_{t}}{(1 + \theta) (\alpha_1 + \alpha_2 Z^{R}_{t}) + \gamma_1},
%     \end{align*}
%     into (A1) we obtain
%     \begin{align*}
%         0&=\left[\frac{\alpha_0 + \alpha_3 Y_{t} -\gamma_0 - \gamma_2 W_{t} -  \gamma_3 R_{t}}{(\theta + 1) (\alpha_1 + \alpha_2 Z^R_{t}) + \gamma_1}\right]Z + \chi_1 \left[\frac{\alpha_0 + \alpha_3 Y_{t} -\gamma_0 - \gamma_2 W_{t} -  \gamma_3 R_{t}}{(\theta + 1) (\alpha_1 + \alpha_2 Z^R_{t}) + \gamma_1}\right] + \chi_2 W_{t} + \chi_3 R_{t} + \chi_4 Y + \chi_5\nonumber\\
%         &=[\alpha_0 + \alpha_3 Y_{t} -\gamma_0 - \gamma_2 W_{t} -  \gamma_3 R_{t}]Z^R_{t} + \chi_1 [\alpha_0 + \alpha_3 Y_{t} -\gamma_0 - \gamma_2 W_{t} -  \gamma_3 R_{t}]\\
%         &\quad+ [(\theta + 1) (\alpha_1 + \alpha_2 Z^R_{t}) + \gamma_1]\chi_2 W_{t} + [(\theta + 1) (\alpha_1 + \alpha_2 Z^R_{t}) + \gamma_1]\chi_3 R_{t}\\
%         &\quad\quad + [(\theta + 1) (\alpha_1 + \alpha_2 Z^R_{t}) + \gamma_1]\chi_4 Y_{t} + [(\theta + 1) (\alpha_1 + \alpha_2 Z^R_{t}) + \gamma_1]\chi_5\nonumber\\
%         &=[\alpha_0-\gamma_0+(\theta + 1)\alpha_2 \chi_5]Z^R_{t}+[\alpha_3+(\theta + 1)\alpha_2 \chi_4]Z^R_{t} Y_{t}\\
%         &\quad +[-\gamma_2+(\theta + 1)\alpha_2 \chi_2]W_{t}Z^R_{t} + [-\gamma_3+(\theta + 1)\alpha_2  \chi_3]R_{t}Z^R_{t} \\
%         &\quad\quad +[\chi_1 \alpha_3+\chi_4\gamma_1 +(\theta+1)\alpha_1 \chi_4]Y_{t}+ [-\chi_1\gamma_2+\chi_2\gamma_1+(\theta+1)\alpha_1 \chi_2]W_{t}\\
%         &\quad\quad\quad +[-\chi_1\gamma_3 +\chi_3 \gamma_1 +(\theta+1)\alpha_1 \chi_3] R_{t} +[\chi_1 (\alpha_0 -\gamma_0)+\chi_5\gamma_1 +(\theta+1)\alpha_1 \chi_5]\nonumber\\
%         &=\zeta_1 Z + \zeta_2 Z^R_{t} Y_{t} + \zeta_3 W_{t}Z + \zeta_4 R_{t}Z + \zeta_5 Y_{t} + \zeta_6 W_{t} + \zeta_7 R_{t} + \zeta_8 
%     \end{align*}
%     where
%     \begin{align*}
%         \zeta_1 &= \alpha_0-\gamma_{0}+(\theta + 1)\alpha_{2} \chi_5\\
%         \zeta_2 &= \alpha_3+(\theta + 1)\alpha_{2} \chi_4\\
%         \zeta_3 &= -\gamma_2+(\theta + 1)\alpha_{2} \chi_2\\
%         \zeta_4 & = -\gamma_3+(\theta + 1)\alpha_{2}  \chi_3\\
%         \zeta_5 & = \chi_1 \alpha_3+(\gamma_1+(\theta+1)\alpha_1 )\chi_4\\
%         \zeta_6 & = -\chi_1\gamma_2+(\gamma_1+(\theta+1)\alpha_1) \chi_2 \\
%         \zeta_7 & = -\chi_1\gamma_3+(\gamma_1+(\theta+1)\alpha_1) \chi_3\\
%         \zeta_8 & = \chi_1 (\alpha_0 -\gamma_{0})+(\gamma_1+(\theta+1)\alpha_1) \chi_5
%     \end{align*}
%     By putting $\zeta_1 = \cdots = \zeta_7 =0$, we obtain 
%     \begin{align*}
%             \chi_1 &= \frac{\gamma_1+(\theta+1)\alpha_1}{\gamma_2}\chi_2=\frac{\gamma_1 + (\theta + 1)\alpha_1}{(\theta + 1)\alpha_{2}}\\
%             \chi_2 &= \frac{\gamma_2}{(\theta + 1)\alpha_{2}}\\
%             \chi_3 &= \frac{\gamma_3}{(\theta + 1)\alpha_{2}}\\
%             \chi_4 &= -\frac{\alpha_3}{(\theta + 1)\alpha_{2}}\\
%             \chi_5 &= -\frac{\alpha_0 - \gamma_{0}}{(\theta + 1)\alpha_{2}}
%     \end{align*}
    
%     By substituting these into (A1),  we have
%     \begin{align*}
%         &Z^{R}_{t} Q_{t} + \frac{\gamma_1 +(\theta + 1)\alpha_1}{(\theta + 1)\alpha_{2}}Q_{t} +  \frac{\gamma_2}{(\theta + 1)\alpha_{2}} W_{t}+  \frac{\gamma_3}{(\theta + 1)\alpha_{2}}R_{t} -\frac{\alpha_3}{(\theta + 1)\alpha_{2}}Y_{t} -\frac{\alpha_0 - \gamma_{0}}{(\theta + 1)\alpha_{2}} \\
%         =&\frac{ (\theta + 1)\alpha_{2}Z^{R}_{t} Q_{t} + [(\theta + 1)\alpha_1 + \gamma_1]Q_{t}  -\alpha_3 Y_{t} + \gamma_2 W_{t}+ \gamma_3 R_{t} - \alpha_0 + \gamma_{0}}{(\theta + 1)\alpha_{2}}\\
%         =& \frac{[(\theta + 1)(\alpha_1 + \alpha_{2} Z) + \gamma_1]Q_{t}  -\alpha_3 Y_{t} + \gamma_2 W_{t}+ \gamma_3 R_{t} - \alpha_0 + \gamma_{0}}{(\theta + 1)\alpha_{2}}\\
%         =& \frac{(\theta + 1)(\alpha_1 + \alpha_{2} Z) + \gamma_1}{(\theta + 1)\alpha_{2}}\left[ Q_{t} - \frac{\alpha_0 + \alpha_3 Y_{t} - \gamma_{0}- \gamma_2 W_{t}- \gamma_3 R_{t}}{(\theta + 1)(\alpha_1 + \alpha_{2} Z) + \gamma_1}\right]\\
%         =& 0,
%     \end{align*}
%     because $Q_{t} = \frac{\alpha_0 + \alpha_3Y_{t} -\gamma_{0} - \gamma_2 W_{t}-  \gamma_3 R_{t}}{(\theta + 1) (\alpha_1 + \alpha_{2} Z) + \gamma_1}$. 
%     Thus, (A1) holds under nonzero coefficients, which implies that $W_{t}, R_{t}, Z^{R}_{t}Q_{t},Q_{t}$, and $Y_{t}$ are linear dependent.
    
% \end{proof}
\section{Online appendix}\label{sec:appendix}
\subsection{Omitted proof of Proposition 1}
\begin{proof}
    By the definition of linear independence, we need to check whether the following holds:
\begin{align}
    \chi_1 Z_{t}^R Q + \chi_2 Q_{t} + \chi_3 W_{t} + \chi_4 R_{t} + \chi_5 = 0, \label{eq:linear_independence}
\end{align}
then $\chi_1 = \chi_2 = \cdots = \chi_5 = 0$.

By substituting Equation \eqref{eq:quantity_linear} into Equation \eqref{eq:linear_independence}, we have
\begin{align*}
    0 &= \zeta_1 Z_{t}^R + \zeta_2 Z_{t}^RY_{t} + \zeta_3 W_{t}Z_{t}^R + \zeta_4 R_{t}Z_{t}^R + \zeta_5 Y_{t} + \zeta_6 W_{t} + \zeta_7 R_{t} + \zeta_8, 
\end{align*}
where 
\begin{align*}
    \zeta_1 &= (\alpha_0 - \gamma_0)\chi_1  + (\theta +1 )\alpha_2 \chi_5 ,\\
    \zeta_2 &= \alpha_3\chi_1,\\
    \zeta_3 &= -\gamma_2 \chi_1 + (\theta + 1)\alpha_2\chi_3,\\
    \zeta_4 &= -\gamma_3 \chi_1 + (\theta + 1)\alpha_2\chi_4,\\
    \zeta_5 &=  \alpha_3\chi_2,\\
    \zeta_6 &= -\gamma_2 \chi_2 + [(1 + \theta) \alpha_1 +\gamma_1]\chi_3,\\
    \zeta_7 &= -\gamma_3 \chi_2 +  [(1 + \theta) \alpha_1 +\gamma_1]\chi_4,\\
    \zeta_8 &=  (\alpha_0 - \gamma_0)\chi_2 +[(1 + \theta)\alpha_1 +\gamma_1] \chi_5.
\end{align*}

First, by Assumption (ii), $\zeta_1 = \cdots = \zeta_8 = 0$.
Second, as parameters are nonzero by Assumption (i), $\chi_1 = \chi_2 =0$ by $\zeta_2 = \zeta_5 = 0$.
Third, by $\zeta_1 = \zeta_3 = \zeta_4 = 0$, $(\theta + 1 )\alpha_2\chi_5 = (\theta + 1 )\alpha_2\chi_3 = (\theta + 1 )\alpha_2\chi_4 = 0.$
As $(\theta + 1)\alpha_2 \ne 0$ by Assumption (i), $\chi_3 = \chi_4 = \chi_5 = 0$.
This completes the proof.
\end{proof}

\subsection{Simulation and estimation procedure}

\begin{table}[!htbp]
    \caption{True parameters and distributions}
    \label{tb:parameter_setting}
    \begin{center}
    \subfloat[Parameters]{
    \begin{tabular}{cr}
            \hline
            & linear  \\
            $\alpha_0$ & $10.0$  \\
            $\alpha_1$ & $1.0$  \\
            $\alpha_2$ & $1.0$ \\
            $\alpha_3$ & $1.0$  \\
            $\gamma_0$ & $1.0$ \\
            $\gamma_1$ & $1.0$  \\
            $\gamma_2$ & $1.0$ \\
            $\gamma_3$ & $1.0$\\
            $\theta$ & $0.5$ \\
            \hline
        \end{tabular}
    }
    \subfloat[Distributions]{
    \begin{tabular}{crr}
            \hline
            & linear\\
            Demand shifter&  \\
            $Y_t$ & $N(0,1)$  \\
            Demand rotation instrument&   \\
            $Z^{R}_{t}$ & $N(10,1)$ \\
            Cost shifter&    \\
            $W_{t}$ & $N(3,1)$  \\
            $R_{t}$ & $N(0,1)$   \\
            $H_{t}$ & $W_{t}+N(0,1)$  \\
            $K_{t}$ & $R_{t}+N(0,1)$   \\
            Error&  &  \\
            $\varepsilon^{d}_{t}$ & $N(0,\sigma)$  \\
            $\varepsilon^{c}_{t}$ & $N(0,\sigma)$ \\
            \hline
        \end{tabular}
    }
    \end{center}
    \footnotesize
    Note: $\sigma=\{0.001, 0.5, 2.0\}$. $N:$ Normal distribution. $U:$ Uniform distribution.
\end{table}

We begin by setting the true parameters and distributions according to Table \ref{tb:parameter_setting}. Following PS, we generate 1000 data sets and separately estimate the demand and supply equations using 2SLS estimation. For the demand estimation, we use instrument variables $Z^{d}{t} = (Z^{R}_{t}, Y_t, H_{t}, K_{t})$, and for the supply estimation, we use instrument variables $Z^{c}{t} = (Z^{R}_{t}, W_{t}, R_{t}, Y_t)$.

To generate the simulation data, we first generate the exogenous variables $Y_t, Z^{R}_{t}, W_t, R{t}, H_t$, and $K_t$ and the error terms $\varepsilon_{t}^c$ and $\varepsilon_{t}^d$ based on the data generation process specified in Table \ref{tb:parameter_setting}. We compute the equilibrium quantity $Q_{t}$ using \eqref{eq:quantity_linear}, and then compute the equilibrium price $P_t$ by substituting $Q_{t}$ and other variables into the demand function \eqref{eq:linear_demand}.

We estimate the equations using the \texttt{ivreg} package in \texttt{R}. Notably, an important feature of the model is the interaction term between the endogenous variable $Q_{t}$ and the instrument variable $Z^{R}_{t}$. The \texttt{ivreg} package automatically detects the endogenous variables as $Q_{t}$ and the interaction term $Z^{R}_{t}Q_{t}$, and runs the first stage regression for each endogenous variable using the same instruments. We confirmed this by manually writing R code that implements the 2SLS. When the first stage includes only the regression of $Q_{t}$, the estimation results from our code differ from those of \texttt{ivreg}. However, when we modified the code to regress $Z^{R}_{t}Q_{t}$ on the instrument variables and estimate the second stage using the predicted values of $Q_{t}$ and $Z^{R}_{t}Q_{t}$, the results from our code and \texttt{ivreg} coincided.


\subsection{Other experiments}

\begin{table}[!htbp]
    \caption{Estimation results in Table 2 of from PS}
    \label{tb:linear_linear_sigma_Perloff_Shen}
    \begin{center}
        \begin{tabular}{cllll}
            \hline
            & $\sigma=0.001$ & $\sigma=0.5$ & $\sigma=1$ & $\sigma=2$ \\
            $\alpha_0$ & $10.00\ (0.001)$ & $9.96\ (0.33)$ & $9.86\ (0.65)$ & $9.46 (1.20)$ \\
            $\alpha_1$ & $1.00\ (0.004)$ & $0.99\ (1.98)$ & $0.97\ (3.96)$ & $0.88 (7.80)$ \\
            $\alpha_2$ & $1.00\ (0.004)$ & $0.99\ (0.21)$ & $0.97\ (0.42)$ & $0.87\ (0.82)$ \\
            $\gamma_1$ & $0.46\ (0.88)$ & $0.46\ (0.91)$ & $0.47\ (0.93)$ & $0.49\ (1.04)$ \\
            $\gamma_2$ & $5.85\ (7.89)$ & $5.85\ (8.15)$ & $5.78\ (8.21)$ & $5.73\ (8.66)$ \\
            $\theta$ & $-0.31\ (1.31)$ & $-0.29\ (1.34)$ & $0.09\ (11.48)$ & $-1.53\ (30.41)$ \\
            \hline
        \end{tabular}
    \end{center}\footnotesize
    Note: True parameters: $\alpha_1 = \alpha_2 = \gamma_0 = \gamma_1 = \gamma_2  = \gamma_3 = 1, \alpha_0 = 10, \alpha_3 = 0,  \theta = 0.5$. PS exclude $Y_t$. We change the parameter notations from the original paper. Note that PS do not provide $\gamma_0$ and $\gamma_3$.
\end{table}

First, we replicate the results in PS, reporting the mean and standard deviation (SD) for comparison. 
To replicate the results, we exclude the demand shifter $Y_t$ and assume the coefficient $\alpha_3$ of $Y_t$ is zero, meaning that there is no demand shifter for the supply estimation. 
We reference Table \ref{tb:linear_linear_sigma_Perloff_Shen} from PS but modify some notations.
The table shows the mean and SD of the 2SLS estimators from 1000 simulations, with a sample size of 50 in each simulation data.
The results indicate that the demand estimation becomes more accurate as the value of the SD of the error terms $\sigma$ decreases, but the supply-side estimation is still biased, and the SD of the conduct parameter becomes larger as the value of $\sigma$ increases.

Table \ref{tb:linear_linear_sigma_1_without_demand_shifter_y} presents our replication results. 
Each panel shows the simulation results under different SDs of the error terms, using the same data generation process as PS. 
To verify that we correctly replicated the results in PS, we focus on the first two columns in each panel, which display the mean and SD of the simulation results when the sample size is 50. 
Although the demand parameter can be accurately estimated, even when the value of $\sigma$ is large, the supply-side parameter is still biased.
Notably, when $\sigma$ is large and the sample size is small, the SD of the parameters in the supply-side equation becomes large. 
Thus, we observe the patterns in PS that lack specific details.

Since PS fixes the sample size at 50, we examine the effect of changing the sample size.
As expected, increasing the sample size given a value of $\sigma$ decreases the SD of the parameter in the supply equation. 
However, no simulation result is close to the true values of the supply parameters and the conduct parameter. 
These results are consistent with those in PS.

\begin{table}[!htbp]
  \begin{center}
      \caption{Estimation results of the linear model without demand shifter}
      \label{tb:linear_linear_sigma_1_without_demand_shifter_y} 
      \subfloat[$\sigma=0.001$]{
\begin{tabular}[t]{lrrrrrrrr}
\toprule
  & (1) $n=50$ / Mean & (1) $n=50$ / SD & (2) $n=100$ / Mean & (2) $n=100$ / SD & (3) $n=200$ / Mean & (3) $n=200$ / SD & (4) $n=1000$ / Mean & (4) $n=1000$ / SD\\
\midrule
$\alpha_{0}$ & 10.000 & 0.0009 & 10.000 & 0.0006 & 10.000 & 0.0004 & 10.000 & 0.0002\\
$\alpha_{1}$ & 1.000 & 0.004 & 1.000 & 0.003 & 1.000 & 0.002 & 1.000 & 0.0009\\
$\alpha_{2}$ & 1.000 & 0.0005 & 1.000 & 0.0003 & 1.000 & 0.0002 & 1.000 & 0.0001\\
$\gamma_{0}$ & 5.446 & 6.981 & 5.388 & 7.986 & 5.423 & 7.825 & 5.063 & 6.801\\
$\gamma_{1}$ & 0.506 & 0.775 & 0.512 & 0.888 & 0.509 & 0.869 & 0.549 & 0.756\\
$\gamma_{2}$ & 0.506 & 0.776 & 0.512 & 0.887 & 0.509 & 0.869 & 0.549 & 0.756\\
$\theta$ & -0.241 & 1.164 & -0.231 & 1.331 & -0.237 & 1.304 & -0.177 & 1.134\\
$R^{2}$ (demand) & 1.000 & 0.0000004 & 1.000 & 0.0000003 & 1.000 & 0.0000002 & 1.000 & 8e-08\\
$R^{2}$ (supply) & 1.000 & 0.000008 & 1.000 & 0.00001 & 1.000 & 0.00001 & 1.000 & 0.000008\\
Sample size ($T$) &  & 50 &  & 100 &  & 200 &  & 1000\\
\bottomrule
\end{tabular}
}\\
      \subfloat[$\sigma=0.5$]{
\begin{tabular}[t]{lrrrrrrrr}
\toprule
  & (1) $n=50$ / Mean & (1) $n=50$ / SD & (2) $n=100$ / Mean & (2) $n=100$ / SD & (3) $n=200$ / Mean & (3) $n=200$ / SD & (4) $n=1000$ / Mean & (4) $n=1000$ / SD\\
\midrule
$\alpha_{0}$ & 9.993 & 0.466 & 9.993 & 0.312 & 10.001 & 0.215 & 10.002 & 0.093\\
$\alpha_{1}$ & 0.963 & 2.138 & 0.965 & 1.484 & 1.012 & 1.023 & 0.991 & 0.441\\
$\alpha_{2}$ & 1.002 & 0.243 & 1.002 & 0.168 & 0.999 & 0.118 & 1.002 & 0.049\\
$\gamma_{0}$ & 5.332 & 10.459 & 5.227 & 11.592 & 5.112 & 15.871 & 5.470 & 7.476\\
$\gamma_{1}$ & 0.405 & 3.214 & 0.434 & 1.989 & 0.474 & 1.744 & 0.516 & 1.102\\
$\gamma_{2}$ & 0.517 & 1.157 & 0.528 & 1.222 & 0.546 & 1.816 & 0.504 & 0.830\\
$\theta$ & -0.210 & 1.879 & -0.206 & 1.951 & -0.186 & 2.705 & -0.247 & 1.238\\
$R^{2}$ (demand) & 0.720 & 0.088 & 0.725 & 0.061 & 0.726 & 0.041 & 0.728 & 0.018\\
$R^{2}$ (supply) & 0.160 & 7.674 & -0.119 & 19.529 & -0.724 & 30.775 & 0.491 & 2.041\\
Sample size (n) &  & 50 &  & 100 &  & 200 &  & 1000\\
\bottomrule
\end{tabular}
}\\
  \end{center}\footnotesize
  Note: True parameters: $\alpha_1 = \alpha_2 =  \gamma_0 = \gamma_1 = \gamma_2  =  1, \alpha_0 = 10, \theta = 0.5.$ and $\alpha_3 =0$. For comparison, we report mean and SD.
\end{table} 

\begin{table}[!htbp]
  \ContinuedFloat
  \begin{center}
      \caption{Estimation results of the linear model without demand shifter (Continued)}
      \subfloat[$\sigma=1.0$]{
\begin{tabular}[t]{lrrrrrrrr}
\toprule
  & Mean & SD & Mean  & SD  & Mean   & SD   & Mean    & SD   \\
\midrule
$\alpha_{0}$ & 9.975 & 0.964 & 9.953 & 0.636 & 10.007 & 0.441 & 9.991 & 0.189\\
$\alpha_{1}$ & 1.120 & 4.491 & 0.942 & 2.885 & 0.883 & 2.055 & 1.035 & 0.902\\
$\alpha_{2}$ & 0.981 & 0.492 & 0.993 & 0.326 & 1.015 & 0.227 & 0.993 & 0.101\\
$\gamma_{0}$ & 5.631 & 9.410 & 5.520 & 7.580 & 5.161 & 9.226 & 5.556 & 7.424\\
$\gamma_{1}$ & -0.107 & 19.285 & 0.129 & 5.240 & 0.488 & 3.541 & 0.489 & 1.210\\
$\gamma_{2}$ & 0.476 & 1.043 & 0.495 & 0.835 & 0.540 & 1.030 & 0.494 & 0.820\\
$\theta$ & -0.201 & 3.603 & -0.217 & 1.478 & -0.183 & 1.528 & -0.260 & 1.229\\
$R^{2}$ (demand) & 0.205 & 0.357 & 0.234 & 0.221 & 0.232 & 0.150 & 0.245 & 0.060\\
$R^{2}$ (supply) & -0.920 & 17.898 & -0.395 & 5.271 & -0.904 & 12.486 & -0.421 & 12.047\\
Sample size (n) &  & 50 &  & 100 &  & 200 &  & 1000\\
\bottomrule
\end{tabular}
}\\
    \subfloat[$\sigma=2.0$]{
\begin{tabular}[t]{lrrrrrrrr}
\toprule
  & Mean & SD & Mean  & SD  & Mean   & SD   & Mean    & SD   \\
\midrule
$\alpha_{0}$ & 9.515 & 6.752 & 9.912 & 1.479 & 9.987 & 0.943 & 9.987 & 0.396\\
$\alpha_{1}$ & 0.362 & 19.344 & 0.710 & 6.192 & 1.154 & 4.363 & 0.986 & 1.728\\
$\alpha_{2}$ & 0.934 & 1.092 & 1.004 & 0.743 & 0.981 & 0.494 & 0.998 & 0.204\\
$\gamma_{0}$ & 5.658 & 6.892 & 5.464 & 8.387 & 5.695 & 8.243 & 5.572 & 10.796\\
$\gamma_{1}$ & 0.956 & 52.166 & 1.715 & 42.062 & -0.056 & 11.467 & 0.388 & 3.140\\
$\gamma_{2}$ & 0.479 & 0.827 & 0.496 & 0.907 & 0.486 & 0.902 & 0.497 & 1.185\\
$\theta$ & -0.296 & 5.941 & -0.439 & 5.106 & -0.235 & 2.034 & -0.256 & 1.771\\
$R^{2}$ (demand) & -3.456 & 87.362 & -0.513 & 1.557 & -0.436 & 0.563 & -0.376 & 0.185\\
$R^{2}$ (supply) & -1.104 & 5.881 & -2.311 & 26.606 & -1.993 & 26.973 & -3.591 & 49.060\\
Sample size ($T$) &  & 50 &  & 100 &  & 200 &  & 1000\\
\bottomrule
\end{tabular}
}
  \end{center}\footnotesize
  Note: True parameters: $\alpha_1 = \alpha_2 =  \gamma_0 = \gamma_1 = \gamma_2  =  1, \alpha_0 = 10, \theta = 0.5.$ and $\alpha_3 =0$. For comparison, we report mean and SD.
\end{table} 



\end{document}