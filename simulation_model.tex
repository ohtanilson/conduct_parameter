\documentclass[11pt]{article}
\usepackage[utf8]{inputenc}
\usepackage{graphicx}
\renewcommand{\familydefault}{\rmdefault}
\usepackage[T1]{fontenc}
\usepackage{geometry}
\geometry{verbose,tmargin=70pt,bmargin=70pt,lmargin=70pt,rmargin=70pt}

%\usepackage{mlmodern}
\usepackage{bm}
\usepackage{amsmath}
\usepackage{amsthm}
\usepackage{amssymb}
\usepackage{setspace}
\usepackage{stmaryrd}
\usepackage{csvsimple}
\usepackage[english]{babel}
\hyphenation{mis-spec-i-fi-ca-tion}
\hyphenation{dif-fer-en-ti-a-ble}
\usepackage{multirow}
\usepackage{mathrsfs}
\usepackage{caption}
\usepackage{booktabs}
\usepackage{mathtools}
\usepackage[dvipsnames]{xcolor}
\usepackage[colorlinks=true, linkcolor= Periwinkle, citecolor = Periwinkle, filecolor = Periwinkle, urlcolor = Periwinkle, pdfpagemode = FullScreen, pagebackref,hypertexnames = true]{hyperref}

\usepackage{natbib} % For reference 
% https://gking.harvard.edu/files/natnotes2.pdf


\setstretch{1.4}
% Header
%\usepackage{fancyhdr}
%\pagestyle{fancy}
%\lhead{The Third-Year Paper Proposal}
%\rhead{Yuri Matsumura}

\makeatletter
%%%%%%%%%%%%%%%%%%%%%%%%%%%%%% Textclass specific LaTeX commands.
%\numberwithin{equation}{section}
\numberwithin{figure}{section}
\theoremstyle{definition}

%%%%%%%%%%%%%%%%%%%%%%%%%%%%%% User specified LaTeX commands.
\newcommand{\indep}{\perp \!\!\! \perp}
\newcommand{\argmax}{\operatornamewithlimits{arg\ max}}
\newcommand{\argmin}{\operatornamewithlimits{arg\ min}}

\newcommand{\bbA}{\mathbb{A}}
\newcommand{\bbB}{\mathbb{B}}
\newcommand{\bbC}{\mathbb{C}}
\newcommand{\bbD}{\mathbb{D}}
\newcommand{\bbE}{\mathbb{E}}
\newcommand{\bbF}{\mathbb{F}}
\newcommand{\bbG}{\mathbb{G}}
\newcommand{\bbH}{\mathbb{H}}
\newcommand{\bbI}{\mathbb{I}}
\newcommand{\bbJ}{\mathbb{J}}
\newcommand{\bbK}{\mathbb{K}}
\newcommand{\bbL}{\mathbb{L}}
\newcommand{\bbM}{\mathbb{M}}
\newcommand{\bbN}{\mathbb{N}}
\newcommand{\bbO}{\mathbb{O}}
\newcommand{\bbP}{\mathbb{P}}
\newcommand{\bbQ}{\mathbb{Q}}
\newcommand{\bbR}{\mathbb{R}}
\newcommand{\bbS}{\mathbb{S}} 
\newcommand{\bbT}{\mathbb{T}} 
\newcommand{\bbU}{\mathbb{U}} 
\newcommand{\bbV}{\mathbb{V}}
\newcommand{\bbW}{\mathbb{W}}
\newcommand{\bbX}{\mathbb{X}}
\newcommand{\bbY}{\mathbb{Y}}
\newcommand{\bbZ}{\mathbb{Z}}

\newcommand{\0}{\mathbf{0}}

\newcommand{\scrA}{\mathscr{A}}
\newcommand{\scrB}{\mathscr{B}}
\newcommand{\scrC}{\mathscr{C}}
\newcommand{\scrD}{\mathscr{D}}
\newcommand{\scrE}{\mathscr{E}}
\newcommand{\scrF}{\mathscr{F}}
\newcommand{\scrG}{\mathscr{G}}
\newcommand{\scrH}{\mathscr{H}}
\newcommand{\scrI}{\mathscr{I}}
\newcommand{\scrJ}{\mathscr{J}}
\newcommand{\scrK}{\mathscr{K}}
\newcommand{\scrL}{\mathscr{L}}
\newcommand{\scrM}{\mathscr{M}}
\newcommand{\scrN}{\mathscr{N}}
\newcommand{\scrO}{\mathscr{O}}
\newcommand{\scrP}{\mathscr{P}}
\newcommand{\scrQ}{\mathscr{Q}}
\newcommand{\scrR}{\mathscr{R}}
\newcommand{\scrS}{\mathscr{S}} 
\newcommand{\scrT}{\mathscr{T}} 
\newcommand{\scrU}{\mathscr{U}} 
\newcommand{\scrV}{\mathscr{V}}
\newcommand{\scrW}{\mathscr{W}}
\newcommand{\scrX}{\mathscr{X}}
\newcommand{\scrY}{\mathscr{Y}}
\newcommand{\scrZ}{\mathscr{Z}}


\newcommand{\calA}{\mathcal{A}}
\newcommand{\calB}{\mathcal{B}}
\newcommand{\calC}{\mathcal{C}}
\newcommand{\calD}{\mathcal{D}}
\newcommand{\calE}{\mathcal{E}}
\newcommand{\calF}{\mathcal{F}}
\newcommand{\calG}{\mathcal{G}}
\newcommand{\calH}{\mathcal{H}}
\newcommand{\calI}{\mathcal{I}}
\newcommand{\calJ}{\mathcal{J}}
\newcommand{\calK}{\mathcal{K}}
\newcommand{\calL}{\mathcal{L}}
\newcommand{\calM}{\mathcal{M}}
\newcommand{\calN}{\mathcal{N}}
\newcommand{\calO}{\mathcal{O}}
\newcommand{\calP}{\mathcal{P}}
\newcommand{\calQ}{\mathcal{Q}}
\newcommand{\calR}{\mathcal{R}}
\newcommand{\calS}{\mathcal{S}} 
\newcommand{\calT}{\mathcal{T}} 
\newcommand{\calU}{\mathcal{U}} 
\newcommand{\calV}{\mathcal{V}}
\newcommand{\calW}{\mathcal{W}}
\newcommand{\calX}{\mathcal{X}}
\newcommand{\calY}{\mathcal{Y}}
\newcommand{\calZ}{\mathcal{Z}}

\newcommand{\Perms[2]}{\tensor[_{#2}]P{_{#1}}}
\newcommand{\Combi[2]}{\tensor[_{#2}]C{_{#1}}}

\newcommand{\bmA}{\bm{A}}
\newcommand{\bmB}{\bm{B}}
\newcommand{\bmC}{\bm{C}}
\newcommand{\bmD}{\bm{D}}
\newcommand{\bmE}{\bm{E}}
\newcommand{\bmF}{\bm{F}}
\newcommand{\bmG}{\bm{G}}
\newcommand{\bmH}{\bm{H}}
\newcommand{\bmI}{\bm{I}}
\newcommand{\bmJ}{\bm{J}}
\newcommand{\bmK}{\bm{K}}
\newcommand{\bmL}{\bm{L}}
\newcommand{\bmM}{\bm{M}}
\newcommand{\bmN}{\bm{N}}
\newcommand{\bmO}{\bm{O}}
\newcommand{\bmP}{\bm{P}}
\newcommand{\bmQ}{\bm{Q}}
\newcommand{\bmR}{\bm{R}}
\newcommand{\bmS}{\bm{S}} 
\newcommand{\bmT}{\bm{T}} 
\newcommand{\bmU}{\bm{U}} 
\newcommand{\bmV}{\bm{V}}
\newcommand{\bmW}{\bm{W}}
\newcommand{\bmX}{\bm{X}}
\newcommand{\bmY}{\bm{Y}}
\newcommand{\bmZ}{\bm{Z}}


\newcommand{\bma}{\bm{a}}
\newcommand{\bmb}{\bm{b}}
\newcommand{\bmc}{\bm{c}}
\newcommand{\bmd}{\bm{d}}
\newcommand{\bme}{\bm{e}}
\newcommand{\bmf}{\bm{f}}
\newcommand{\bmg}{\bm{g}}
\newcommand{\bmh}{\bm{h}}
\newcommand{\bmi}{\bm{i}}
\newcommand{\bmj}{\bm{j}}
\newcommand{\bmk}{\bm{k}}
\newcommand{\bml}{\bm{l}}
\newcommand{\bmm}{\bm{m}}
\newcommand{\bmn}{\bm{n}}
\newcommand{\bmo}{\bm{o}}
\newcommand{\bmp}{\bm{p}}
\newcommand{\bmq}{\bm{q}}
\newcommand{\bmr}{\bm{r}}
\newcommand{\bms}{\bm{s}} 
\newcommand{\bmt}{\bm{t}} 
\newcommand{\bmu}{\bm{u}} 
\newcommand{\bmv}{\bm{v}}
\newcommand{\bmw}{\bm{w}}
\newcommand{\bmx}{\bm{x}}
\newcommand{\bmy}{\bm{y}}
\newcommand{\bmz}{\bm{z}}

\newcommand{\bmal}{\bm{\alpha}}
\newcommand{\bmbe}{\bm{\beta}}
\newcommand{\bmga}{\bm{\gamma}}
\newcommand{\bmvare}{\bm{\varepsilon}}
\newcommand{\bmth}{\bm{\theta}}

\makeatother

\newtheorem{theorem}{Theorem}
\newtheorem{assumption}{Assumption}
\newtheorem{lemma}{Lemma}
\newtheorem{definition}{Definition}
\newtheorem{proposition}{Proposition}
\newtheorem{claim}{Claim}
\newtheorem{corollary}{Corollary}

\usepackage{enumitem}
\newlist{legal}{enumerate}{10}
\setlist[legal]{label*=\arabic*.}


\begin{document}

In this note, we investigate several Monte Carlo simulations in the previous literature.
The researcher has data with $T$ markets with homogeneous products.
Assume that there are $N_t$ firms in each market.
Let $t = 1,\ldots, T$ be the index of markets and $j = 1, \ldots, N_t$ be the index of firms in market $t$.
Firm $j$ solves a profit maximization problem such that
\begin{align*}
    \max_{q_{jt}} \ \pi_{jt}(q_{jt}, q_{-jt}) \equiv (P(Q_t) - mc_{jt}(q_{jt}))q_{jt},
\end{align*}
where $Q_t = \sum_{j = 1}^{N_t} q_{jt}$ is the aggregate quantity, $P(Q_t)$ the inverse demand function, and $mc_{jt}(q_{jt})$ the marginal cost function.
From the maximization problem, we obtain the first-order condition,
\begin{align*}
    0 = P(Q_{t}) - mc_{jt}(q_{jt}) + P'(Q_{t})q_{jt}.
\end{align*}
By summing up the first-order condition across firms, 
\begin{align*}
    0 &= NP(Q_t) - \sum_{j = 1}^{N_t} mc_{jt}(q_{jt}) + P'(Q_t)Q_{t}\\
    & =  P(Q_t) - \frac{\sum_{j = 1}^{N_t} mc_{jt}(q_{jt})}{N_{t}} + \frac{P'(Q_t)}{N_t}Q_{t}
\end{align*}


\section{Replication of Perloff and Shen (2012)}
This paper points out that the conduct parameter can not be estimated accurately when the demand and marginal cost are linear due to multicolinearity. Note that we change the notation from the original paper
\begin{itemize}
    \item Demand relation: $p = \alpha_0 - [\alpha_1 + \alpha_2Z] Q + \alpha_3Y + \varepsilon_d$
    \item Marginal cost: $ MC = \gamma_0  + \gamma_1 Q + \gamma_2 w + \gamma_3 r + \varepsilon_c$
    \item Supply relation: $p = \gamma_0 + [\theta(\alpha_1 + \alpha_2Z)+ \gamma_1] Q   + \gamma_2 w + \gamma_3 r + \varepsilon_c$
    \item $w \sim N (3, 1), r \sim N (0, 1), Z \sim N (10, 1)$
    \item $\alpha_1 = \alpha_2 = \gamma_0 = \gamma_1 = \gamma_2  = \gamma_3 = 1, \alpha_3 = 0, \alpha_0 = 10, \theta = 0.5.$
    \item $\varepsilon_c\sim N(0,\sigma_d)$, $\varepsilon_d \sim N(0,\sigma_c)$, and $\sigma = \sigma_d = \sigma_c$
\end{itemize}

From the supply relation and the demand relation, we have
\begin{align}
    \alpha_0 - [\alpha_1 + \alpha_2Z] Q + \alpha_3Y + \varepsilon_d = \gamma_0 + [\theta(\alpha_1 + \alpha_2Z)+ \gamma_1] Q   + \gamma_2 w + \gamma_3 r + \varepsilon_c,
\end{align}
which gives us the equilibrium aggregate quantity $Q$ such that 
\begin{align*}
    Q_t =  \frac{\alpha_0 + \alpha_3 Y - \gamma_0 - \gamma_2 w - \gamma_3 r + (\varepsilon_d - \varepsilon_c)}{(1 + \theta) (\alpha_1 + \alpha_2Z) + \gamma_1}.
\end{align*}
As $\alpha_3 = 0$, $Y$ can be ignored.

To generate simulation data, after calculating the equilibrium aggregate quantity $Q$ for market $t$, the equilibrium price is obtained from the demand relation.
To estimate the parameters, we use the two-step least squares estimation. Note that we can separately estimate the parameters in the demand and supply relation. First, regress the exogenous variable $Z$ and the cost shifters $w,r$ on the aggregate quantity $Q$, which gives a predicted value of $Q$ denoted as $\hat{Q}$. Next, estimate the parameters in the demand relation. 
By regressing $(\alpha_1 + \alpha_2 Z_t) Q, Q, w$ and $r$ on the price $p$, we estimate the parameters in the supply relation.

The two-stage least square estimation 
\begin{itemize}
    \item Regress $Q_t = \pi_0 + \pi_1 Z + \pi_2 w + \pi_3 r + \nu$ and obtain $\hat{Q}$
    \item Estimate the demand parameters by regressing $P_t = \alpha_0 - [\alpha_1 + \alpha_2Z] \hat{Q} + \alpha_3Y + \varepsilon_d$
    \item Given the demand parameter, regress $p = \gamma_0 + \theta(\hat{\alpha}_1 + \hat{\alpha}_2Z)Q + \gamma_1Q + \gamma_2 w + \gamma_3 r + \varepsilon_c$
\end{itemize}



\section{Replication of Hyde and Perloff (1995)}
Hyde and Perloff (1995) consider a model with log demand and a Cobb-Douglas cost function. 
The demand relation is given as a log-demand, 
\[\log P_{t} = \alpha_0 - (\alpha_1 + \alpha_2 Z_t) \log Q_t + \varepsilon_{Dt} \equiv \log P_t^* + \varepsilon_{Dt}\]
The marginal cost has a Cobb-Douglas form,
\begin{align*}
    \log MC_t &= -\frac{1}{\gamma}\log A + \frac{1-\gamma}{\gamma}\log Q_t + \frac{\alpha}{\gamma} \log \left(\frac{w}{\alpha}\right) + \frac{\beta}{\gamma} \log \left(\frac{r}{\beta}\right) + \varepsilon_{st}\\
        & = \left( -\frac{1}{\gamma}\log A - \frac{\alpha}{\gamma}\log \alpha -  \frac{\beta}{\gamma}\log\beta    \right) + \frac{1-\gamma}{\gamma}\log Q_t + \frac{\alpha}{\gamma} \log w + \frac{\beta}{\gamma} \log r + \varepsilon_{st} \\
        &\equiv \gamma_0 + \gamma_1 \log Q_t +  \gamma_2 \log w + \gamma_3 \log r + \varepsilon_{st}
\end{align*}

From the supply relation and the demand relation, we have
\begin{align*}
   &\alpha_0 - (\alpha_1 + \alpha_2 Z_t)\log Q_t + \varepsilon_{dt} + \log (1 - \theta (\alpha_1 + \alpha_2 Z_t))\\
   &\quad = \gamma_0 + \gamma_1 \log Q_t +  \gamma_2 \log w + \gamma_3 \log r + \varepsilon_{st},
\end{align*}
which gives the aggregate quantity 
\begin{align*}
    \log Q_t &= \frac{ \alpha_0 + \log (1 - \theta (\alpha_1 + \alpha_2 Z_t)) - \gamma_0  -  \gamma_2 \log w - \gamma_3 \log r +  \varepsilon_{dt} - \varepsilon_{st}}{\gamma_1+ \alpha_1 + \alpha_2 Z_t}\\
    &\equiv \log Q_t^* + \frac{\varepsilon_{dt} - \varepsilon_{st}}{\gamma_1+ \alpha_1 + \alpha_2 Z_t}
\end{align*}
To obtain the equilibrium price, substitute $\log Q_t^*$ into $\log P_t^*$ in the demand equation.


The parameter values in the simulation are
\begin{itemize}
    \item $A = 1.2, \alpha/\gamma = 1/3, \beta/\gamma = 2/3, \alpha_0 = 1.8, \alpha_1 = 1.2,$ and $\alpha_2 = -0.5$
    \item As the marginal cost is a Cobb-Douglas function, $\alpha + \beta = \gamma $, which implies that $\alpha = 1, \beta = 2, \gamma = 3$
\end{itemize}

\end{document}