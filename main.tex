\documentclass[11pt, a4paper]{article}
\usepackage[utf8]{inputenc}
\usepackage{amsmath,setspace,geometry}
\usepackage{amsfonts}
\usepackage[shortlabels]{enumitem}
%\usepackage[dvipdfmx]{hyperref,graphicx}
\usepackage{graphicx}
\usepackage{bbm}
\usepackage[dvipsnames]{xcolor}
\usepackage[colorlinks=true, linkcolor= purple, citecolor = purple, filecolor = purple, urlcolor = purple,hypertexnames = true]{hyperref}
\usepackage[]{natbib} 
\bibpunct[:]{(}{)}{,}{a}{}{,}
\geometry{left = 1.0in,right = 1.0in,top = 1.0in,bottom = 1.0in}
%\onehalfspacing
% \usepackage{setspace}
%\doublespacing
%\renewcommand{\baselinestretch}{0.3}
\usepackage[english]{babel}
\usepackage{float}
\usepackage{subfig}
\usepackage{booktabs}
\usepackage{pdfpages}
\usepackage{threeparttable}
\usepackage{lscape}
\setstretch{1.4}



\newtheorem{theorem}{Theorem}
\newtheorem{assumption}{Assumption}
\newtheorem{lemma}{Lemma}
\newtheorem{definition}{Definition}
\newtheorem{proposition}{Proposition}
\newtheorem{claim}{Claim}
\newtheorem{corollary}{Corollary}
\newtheorem{example}{Example}

\title{Conduct Parameter Project}
\author{Yuri Matsumura \and Suguru Otani}

\begin{document}

\maketitle

\begin{abstract}
    XXX
\end{abstract}

\section{Introduction}
Measuring competitiveness in markets is one of important tasks in Empirical Industrial Organization literature.
Markup is regarded as a useful measure of competitiveness. 
However, markup cannot be measured directly from data because data lacks the information of marginal cost.
Therefore, the researchers have tried to estimate the markup in markets.



\section{Model}
We consider Cournot competition in the spot market. Let $Q_{mt}\in\mathbb{R}_{+}$ and $P_{mt}\in\mathbb{R}_{+}$ be quantity and price in market $m$ in year $t$. Let $N_{mt}\in\mathbb{N}$ be the number of firms in market $m$ in year $t$. Each firm has marginal cost $mc_{imt}(q_{imt};\gamma_0,\gamma_1)$ where $\gamma_0$ and $\gamma_1$ are cost parameters and $q_{imt}$ is firm $i$'s quantity. All variables are known to all firms. All firms compete in quantity. As in \cite{bresnahan1982oligopoly}, the equilibrium condition adding up each firm's first order condition and dividing $N_{mt}$ provides supply relationship which is defined as 
\begin{align}
    P_{m t}=mc_{mt}(Q_{mt},N_{mt};\gamma_0,\gamma_1)-\theta \frac{Q_{mt}}{N_{mt}} \frac{\partial P\left(Q_{m t}\right)}{\partial Q_{m t}}+\varepsilon_{imt}^{S},\label{eq:supply}
\end{align}
where $\varepsilon_{imt}^{S}$ is some structural error and $mc_{mt}=\frac{\sum_{i=1}^{N_{mt}}mc_{imt}}{N_{mt}}=\gamma_0+\gamma_1 \frac{Q_{mt}}{N_{mt}}$ is average marginal cost and conduct parameter $\theta$ nests the perfect competition ($\theta=0$), monopoly which is equivalent to $N$-firm collusion ($\theta=1$), and cournot competition.

The inverse demand function is defined as 
\begin{align}
    P_{mt}=P_{mt}(Q_{mt};\alpha_0,\alpha_1,\alpha_2)+\varepsilon_{imt}^{D}\label{eq:demand}
\end{align}
where $\varepsilon_{imt}^{D}$ is some structural error and $\alpha_0$ and $\alpha_1$ are cost parameters.

Then, parameters $\alpha_0,\alpha_1,\alpha_2,\gamma_0,\gamma_1$ and $\theta$ are estimated by the simultaneous equation model of \eqref{eq:supply} and \eqref{eq:demand} with instrumental variables (IV).

To make the problem clear, we define \textit{Bresnahan class} as the set of specifications using linear demand and supply equations with some nonseparable demand rotation instrument. Also, we define \textit{Lau class} as the set of specifications using twice differentiable demand and supply equations with some nonseparable demand rotation instrument. These two classes belong to the large class in which the conduct parameter and cost parameters are separable. We study another large class in which the conduct parameter and cost parameters are nonseparable. \textcolor{blue}{The class is used in several papers.}

\begin{table}[!ht]
\caption{Summary of results}
\centering
\scriptsize
\begin{tabular}{c|c|c|c} 
 \hline
 Large Class & Class and Papers & Demand: $P_{mt}$ &  Supply: $mc_{imt}$ \\ 
 \hline
 $\theta$ and $\gamma$ are separable &Bresnahan class \citep{merel2009measuring}& Linear  &  Linear \\ 
 $\theta$ and $\gamma$ are separable&? \citep{perloff2012collinearity} & Linear  &  Log-Linear \\  
 $\theta$ and $\gamma$ are nonseparable&Our class & log P Linear  &  Linear \\ 
 $\theta$ and $\gamma$ are separable &Lau class & log P Linear  &  Log-Linear \\ 
 $\theta$ and $\gamma$ are nonseparable&Our class & log Q Linear  &  Linear \\ 
 $\theta$ and $\gamma$ are separable&Lau class \citep{coccorese2013multimarket}& log Q Linear  &  Log-Linear\\ 
 $\theta$ and $\gamma$ are nonseparable&Our class\citep{okazaki2022excess,merel2009measuring}\textcolor{blue}{??}& log-log Linear  &  Linear  \textcolor{blue}{??} \\ 
  $\theta$ and $\gamma$ are separable&Lau class \citep{hyde1995can}& log-log Linear  &  Log-Linear\\ 
 \hline
\end{tabular}
\label{tb:summary_of_results}
\\
\footnotesize Note: The specifications are as follows:
\begin{align*}
    \text{Linear demand:}& P_{mt}=\alpha_0+(\alpha_1+\alpha_2 Z_{mt}) Q_{mt}\\
    \text{log P Linear demand:}&
    \log P_{mt}=\alpha_0+(\alpha_1+\alpha_2 Z_{mt}) Q_{mt}\\
    \text{log Q Linear demand:}&    P_{mt}=\alpha_0+(\alpha_1+\alpha_2 Z_{mt}) \log Q_{mt}\\
    \text{log-log Linear demand:}&
    \log P_{mt}=\alpha_0+(\alpha_1+\alpha_2 Z_{mt})\log Q_{mt}\\
    \text{linear marginal cost}:&
    \gamma_0+\gamma_1 q_{imt}\\
    \text{log-linear marginal cost}:&
    \exp(\gamma_0+\gamma_1 q_{imt})
\end{align*}
\end{table}

For linear demand and linear marginal cost case, \cite{perloff2012collinearity} find severe collinearity problem and shows the simulation results that the estimated coefficients are likely to be highly unstable and unreliable due to nearly perfect collinearity. They suggest that the problem can be avoided if at least one of the equations is log-linear or has some other functional form.

\textcolor{blue}{We show that their conjecture needs to take much care of specification. Also, we provide additional conditions for identification for non-identified cases.
A key in identification is that the conduct parameter and the cost parameters are need to be separable.}
As we will discuss, the class in \citet{bresnahan1982oligopoly} and \citet{lau1982identifying} separates the conduct parameters and the cost parameter, although demand parameters and the conduct parameter are not separated, which requires the demand rotation IV.
However, the supply relationship we are considering is
\begin{align}
    P_{mt} \left( 1 + \theta\frac{\partial f(Q_{mt}, Z_{mt})}{\partial Q_{mt}}\right) = mc_{mt} \label{eq:supply_in_our_class}
\end{align}
where $f(Q_{mt},Z_{mt})$ is a function based on the demand relationship, under which not only demand parameters but also marginal cost parameters are not separated from the conduct parameter.
This class is not trivial because log-log demand is included in the class.
\textcolor{blue}{We illustrate how different identification results the demand and marginal cost specifications induce in Table \ref{tb:summary_of_results}. }


\subsection{Linear demand}
The inverse demand is defined as linear demand form $P_{mt}=\alpha_0+\alpha_1 Q_{mt}$. Then, $\frac{\partial P\left(Q_{m t}\right)}{\partial Q_{mt}}=\frac{1}{\alpha_1}$. Substituting this into the equilibrium condition, we obtain
\begin{align*}
    P_{m t}&=\gamma_0+\gamma_1 \frac{Q_{mt}}{N_{mt}}-\theta \frac{Q_{mt}}{N_{mt}}\frac{1}{\alpha_1}=\gamma_0+(\gamma_1-\frac{\theta}{\alpha_1}) \frac{Q_{mt}}{N_{mt}}
\end{align*}
Then, $\gamma_0$ and $(\gamma_1-\frac{\theta}{\alpha_1})$ are estimated. However, given $\gamma_0$ and $(\gamma_1-\frac{\theta}{\alpha_1})$, $\theta$ and $\gamma_1$ cannot be identified.

\cite{bresnahan1982oligopoly} provides the trick to identify $\theta$. The inverse demand is defined as linear demand form $P_{mt}=\alpha_0+(\alpha_1+\alpha_2 Z_{mt}) Q_{mt}$, where the demand shifter is denoted as $Z_{mt}$ which is correlated with $P_{mt}$ but uncorrelated with $\varepsilon_{imt}^D$. Then, $\frac{\partial P\left(Q_{m t}\right)}{\partial Q_{mt}}=\frac{1}{(\alpha_1+\alpha_2 Z_{mt})}$. Substituting this into the equilibrium condition, we obtain
\begin{align*}
    P_{m t}&=\gamma_0+\gamma_1 \frac{Q_{mt}}{N_{mt}}-\theta \frac{Q_{mt}}{N_{mt}}\frac{1}{(\alpha_1+\alpha_2 Z_{mt})}
\end{align*}
Then, $\gamma_0$, $\gamma_1$, and $\theta$ can be estimated. However,  \cite{perloff2012collinearity} find severe collinearity problem and shows the simulation results that the estimated coefficients are likely to be highly unstable and unreliable due to nearly perfect collinearity.

\subsection{Case 1: Log-log demand and linear marginal cost}
The inverse demand is defined as log-linear demand form $\log P_{mt}=\alpha_0+\alpha_1 \log Q_{mt}$. Then, $\frac{\partial P\left(Q_{m t}\right)}{\partial Q_{mt}}=\frac{\partial \log P\left(Q_{m t}\right)}{\partial \log Q_{mt}}\frac{P_{mt}}{Q_{mt}}=\frac{P_{mt}}{\alpha_1 Q_{mt}}$. Substituting this into the equilibrium condition, we obtain
    \begin{align}
        P_{m t}&=\gamma_0+\gamma_1 \frac{Q_{mt}}{N_{mt}}-\theta \frac{Q_{mt}}{N_{mt}}\frac{P_{mt}}{\alpha_1 Q_{mt}}=\gamma_0+\gamma_1 \frac{Q_{mt}}{N_{mt}}+ \frac{\theta}{\alpha_1}\frac{P_{mt}}{N_{mt}}.
    \end{align}
    We need to solve this for $P_{mt}$. Then we obtain
    \begin{align*}
        P_{m t}&=\left(1-\frac{\theta}{\alpha_1 N_{mt}}\right)^{-1}\gamma_0+\left(1-\frac{\theta}{\alpha_1 N_{mt}}\right)^{-1}\gamma_1 \frac{Q_{mt}}{N_{mt}}\\ 
        & =\frac{\alpha_1 \gamma_0 N_{mt}}{\alpha_1 N_{mt}-\theta}+\frac{\alpha_1 \gamma_1 N_{mt}}{\alpha_1 N_{mt}-\theta} \frac{Q_{mt}}{N_{mt}}\\
        & =\frac{\alpha_1 \gamma_0 N_{mt}}{\alpha_1 N_{mt}-\theta}+\frac{\alpha_1 \gamma_1}{\alpha_1 N_{mt}-\theta} Q_{mt}
    \end{align*}
    which is the form of Equation \eqref{eq:supply_in_our_class}.
    Then, $A_0\equiv\frac{\alpha_1 \gamma_0 N_{mt}}{\alpha_1 N_{mt}-\theta}$ and $A_1\equiv\frac{\alpha_1 \gamma_1}{\alpha_1 N_{mt}-\theta}$ are identified. However, $\gamma_0,\gamma_1$ and $\theta$ cannot be identified. 
\subsection{Case 2: Log-log demand and linear marginal cost with demand rotation IV}
Next, we consider the log-log demand with demand rotation IV as in \cite{bresnahan1982oligopoly}.
$\log P_{mt}=\alpha_0+(\alpha_1+\alpha_2 Z_{mt}) \log Q_{mt}$. Then, $\frac{\partial P\left(Q_{m t}\right)}{\partial Q_{mt}}=\frac{\partial \log P\left(Q_{m t}\right)}{\partial \log Q_{mt}}\frac{P_{mt}}{Q_{mt}}=\frac{P_{mt}}{(\alpha_1+\alpha_2 Z_{mt}) Q_{mt}}$. Substituting this into the equilibrium condition, we obtain
    \begin{align}
        P_{m t}&=\gamma_0+\gamma_1 \frac{Q_{mt}}{N_{mt}}-\theta \frac{Q_{mt}}{N_{mt}}\frac{P_{mt}}{(\alpha_1+\alpha_2 Z_{mt}) Q_{mt}}=\gamma_0+\gamma_1 \frac{Q_{mt}}{N_{mt}}+ \frac{\theta}{(\alpha_1+\alpha_2 Z_{mt})}\frac{P_{mt}}{N_{mt}}
    \end{align}
    We need to solve this for $P_{mt}$. Then we obtain
    \begin{align*}
        P_{m t}&=\left(1-\frac{\theta}{(\alpha_1+\alpha_2 Z_{mt}) N_{mt}}\right)^{-1}\gamma_0+\left(1-\frac{\theta}{(\alpha_1+\alpha_2 Z_{mt}) N_{mt}}\right)^{-1}\gamma_1 \frac{Q_{mt}}{N_{mt}}\\
        & =\frac{(\alpha_1+\alpha_2 Z_{mt}) \gamma_0 N_{mt}}{(\alpha_1+\alpha_2 Z_{mt}) N_{mt}-\theta}+\frac{(\alpha_1+\alpha_2 Z_{mt}) \gamma_1 N_{mt}}{(\alpha_1+\alpha_2 Z_{mt}) N_{mt}-\theta} \frac{Q_{mt}}{N_{mt}}\\
        &=\frac{(\alpha_1+\alpha_2 Z_{mt}) \gamma_0 N_{mt}}{(\alpha_1+\alpha_2 Z_{mt}) N_{mt}-\theta}+\frac{(\alpha_1+\alpha_2 Z_{mt}) \gamma_1}{(\alpha_1+\alpha_2 Z_{mt}) N_{mt}-\theta} Q_{mt}
    \end{align*}
    which is the form of Equation \eqref{eq:supply_in_our_class}.
    Then, $B_0 \equiv \frac{(\alpha_1+\alpha_2 Z_{mt}) \gamma_0 N_{mt}}{(\alpha_1+\alpha_2 Z_{mt}) N_{mt}-\theta}$ and $B_1\equiv\frac{(\alpha_1+\alpha_2 Z_{mt}) \gamma_1}{(\alpha_1+\alpha_2 Z_{mt}) N_{mt}-\theta}$ are identified. However, $\gamma_0,\gamma_1$ and $\theta$ cannot be identified even with demand rotation IV.



\subsection{Additional identifying assumption}

By assuming that the average marginal cost is linear, we have seen the cases where the joint identification of marginal cost parameters and conduct parameter is impossible.
Especially, when we take logarithm to the market price, $P_{mt}$, the marginal cost parameters and the conduct parameter becomes nonseparable, and thus identification of these parameters is impossible even when we have a demand rotation IV. We provide additional identifying assumptions: (1) normalization and (2) log-linear marginal cost.

\paragraph{The first aid: Normalization}

For Case 1, when we normalize one of the parameters, we can identify the rest of them. Let's normalize $\gamma_0 = 1$. Because the ratio of $A_0$ and $A_1$ are known, $\gamma_1$ is expressed as 
    \begin{align*}
        \frac{A_0}{A_1} &\equiv  \frac{\frac{\alpha_1 \gamma_0 N_{mt}}{\alpha_1 N_{mt}-\theta}}{\frac{\alpha_1 \gamma_1}{\alpha_1 N_{mt}-\theta}} = \frac{\alpha_1 N_{mt}}{\alpha_1 N_{mt}-\theta}\\
        \Longrightarrow \gamma_1 &= \frac{N_{mt}A_1}{A_0}.
    \end{align*}
    Then the conduct parameter is expressed as 
    \begin{align*}
        \theta = \alpha_1 N_{mt} \left(1 - \frac{1}{A_0}\right).
    \end{align*}
    \textcolor{blue}{Note that we do not need demand rotation IV under the normalization.}

For Case 2, when we normalize one of the parameters, we can identify the rest of them. Let's normalize $\gamma_0 = 1$.  Then we have 
    \begin{align*}
    \frac{B_0}{B_1} = \frac{\frac{(\alpha_1+\alpha_2 Z_{mt}) \gamma_0 N_{mt}}{(\alpha_1+\alpha_2 Z_{mt}) N_{mt}-\theta}}{\frac{(\alpha_1+\alpha_2 Z_{mt}) \gamma_1}{(\alpha_1+\alpha_2 Z_{mt}) N_{mt}-\theta}} = \frac{N_{mt}}{\gamma_1}
    \end{align*}
    As we know the value of $B_0/B_1$, we can recover $\gamma_1 = N_{mt}B_1/B_0$. Then the conduct parameter is also identified because \begin{align*}
        B_1 = \frac{(\alpha_1+\alpha_2 Z_{mt})N_{mt}B_1/B_0 }{(\alpha_1+\alpha_2 Z_{mt}) N_{mt}-\theta}\\
        \Longrightarrow \theta = (\alpha_1+\alpha_2 Z_{mt})N_{mt}\left(1 -\frac{1}{B_0}\right),
    \end{align*}
    whose right hand side are consists of the observed and identified parameters.


\paragraph{The second aid: log-linear marginal cost}

Instead of the assumption, let's assume that the log of marginal cost is linear, 
\begin{align*}
\log mc_{mt} = \gamma_0 + \gamma_1 \frac{Q_t}{N_{mt}} +\varepsilon_{mt}^{S}.
\end{align*}

For Case 1, from the first-order condition, we have
    \begin{align*}
        mc_{mt}& = P_{mt}\left( 1-\frac{\theta}{\alpha_1 N_{mt}} \right)\\
        \Longrightarrow\log P_{mt}  &=  \Gamma_0 + \gamma_1 \frac{Q_t}{N_{mt}} +\varepsilon_{mt}^{S} 
    \end{align*}
    where $\Gamma_0 = -\log \left( 1-\frac{\theta}{\alpha_1 N_{mt}} \right) +\gamma_0$.
    Since the demand parameters are independently identified by using cost shifters, the value of $\alpha_1$ is known. However, we cannot identify the conduct parameter $\theta$ and cost parameter $\gamma_0$ because only the value of $\Gamma_0$ is identified. The case faces the same problem in \cite{bresnahan1982oligopoly}.
    For Case 2, the supply relationship is 
    \begin{align*}
            mc_{mt}& = P_{mt}\left(  1 - \frac{\theta}{(\alpha_1+\alpha_2 Z_{mt})N_{mt}} \right)\\
            \Longrightarrow \log P_{mt}& = -\log \left(  1 - \frac{\theta}{(\alpha_1+\alpha_2 Z_{mt})N_{mt}} \right) +\gamma_0 + \gamma_1 Q_{mt} + \varepsilon_{mt}^{S}. 
    \end{align*}
    Again the demand parameters are independently identified. Denote $\frac{1}{(\alpha_1+\alpha_2 Z_{mt})N_{mt}} = C_{mt}$, which is an exogenous variable. The supply relationship is written as 
    \begin{align*}
        \log P_{mt} = -\log \left(  1 - \theta C_{mt} \right) +\gamma_0 + \gamma_1 Q_{mt} + \varepsilon_{mt}^{S}. 
    \end{align*}
    which shows that the supply relation is not linear in parameters. \textcolor{blue}{The model does not use the standard identification results of linear regressions.} Instead, we need to assume the following for Method of Moment:
    \begin{itemize}
        \item Assume that $1 - \frac{\theta}{(\alpha_1+\alpha_2 Z_{mt})N_{mt}}=0$ for all $m$ and $t$. Also, parameters $(\theta,\gamma_0,\gamma_1)$ are identified.
    \end{itemize}
    Given this assumption, the moment condition is constructed as $$m_{mt}(\theta,\gamma_0,\gamma_1)=K_{mt}\left(\log P_{mt} +\log \left(  1 - \frac{\theta}{(\alpha_1+\alpha_2 Z_{mt})N_{mt}} \right) -\gamma_0 - \gamma_1 Q_{mt}\right)$$ where $K_{mt}$ is an instrument orthogonal to supply side shock $\varepsilon_{mt}^{S}$. Then, we can estimate $\theta,\gamma_0,\gamma_1$ by method of moment.

\section{Numerical experimentation}
We provide numerical comparison across specifications. We estimate the model using two-stage least square (2SLS), three-stage least
squares (3SLS), and nonlinear three-stage least squares (NL3SLS).
    



\section{Comments}
    \begin{itemize}
        \item \textcolor{red}{ The identification of $\theta$ is based on the identification of nonlinear model or the GMM model. Estimation can be done by nonlinear regression or GMM?}
        \item Can we use the argument in the identification of GMM model to prove the identification of the model such that  $y = \alpha_0 + \log(\alpha_1 + \alpha_2 x +\varepsilon)?$
        \item \cite{perloff2012collinearity} considers a case where the error terms in the demand and the supply relationship take specific values. But does it happen with positive probability?  
        \item Under the setting in \cite{perloff2012collinearity}, $Q$ and $ZQ$ have linear relationship, 
        \begin{align*}
            \theta \alpha_1 Q + \theta \alpha_2 ZQ = \frac{\theta}{1 + \theta} [\alpha_0 - \gamma_0].
        \end{align*}
        On the other hand, under the log-log demand and the log marginal cost case, it seems that colinearity does not happen because 
        \begin{align*}
            \alpha_1 \log Q + \alpha_2 Z\log Q = \gamma_0 - \alpha_0 -  \log \left(  1 - \theta Q^*_{mt} \right),
        \end{align*}
        which implies that both $\log Q$ and $Z\log Q$ do not have a linear relationship.
        \item \cite{lau1982identifying} requires that the marginal cost function is twice continuously differentiable. The assumption is used to show that there exists a function $F$ such that $g^*(Q,z_2) = F(g(Q,z_2))$. The fact is proven by Goldmand and Uzawa (1964), which is cited in footnote 5 in Lau (1982). Therefore, twice continuously differentiable assumption on the inverse demand function and the marginal cost function is a key assumption to identify the conduct parameter. For example, log-log inverse demands and log-marginal costs satisfy the assumption because in general they are written as 
        \begin{align*}
            P_t &= \exp(\alpha_0+\beta_d X_t + \varepsilon_{d,t})Q_t^{(\alpha_1 + \alpha_2 Z_t)},\\
            MC_t &= \exp(\gamma_0+\beta_c W_t + \varepsilon_{c,t})Q_t^{\gamma_1},
        \end{align*}
        which are obviously twice continuously differentiable. 
        But the linear marginal cost does not satisfy the assumption. \textcolor{red}{Why is identification failed by dropping twice continuously differentiability on the marginal cost?}
        \item Under the combination of Linear demand and log marginal cost,the first-order condition is 
        \begin{align*}
            P_t +\theta\frac{Q_{mt}}{(\alpha_1 + \alpha_2 Z_{mt})N_{mt}} = \exp(\gamma_0 + \varepsilon_{c,mt})Q_t^{\gamma_1}.
        \end{align*}
        Define a new exogenous variable $Q^*_{mt} = \frac{Q_{mt}}{(\alpha_1 + \alpha_2 Z_{mt})N_{mt}}$, and then the supply relationship becomes
        \begin{align*}
            P_t = -\theta Q^*_{mt}+ \exp(\gamma_0 + \varepsilon_{c,mt})Q_t^{\gamma_1}
        \end{align*}
        \textcolor{red}{Can we identify the parameters under the equation?} When we fix the conduct parameter, we can identify the cost parameters. However, joint identification of the conduct parameter and the marginal cost parameters may be difficult.
        \item Can the conduct parameter in the trans-log model (i.e., log-log linear marginal cost) be identified?
        \begin{align*}
            \ln C=a_0+\sum_{i=1}^N a_i \ln w_i+a_y \ln y+\frac{1}{2} \sum_{i=1}^N \sum_{j=1}^N a_{i j} \ln w_i \ln w_j+\sum_{i=1}^N a_{i y} \ln w_i \ln y+\frac{1}{2} a_{y y} \ln y \ln y
        \end{align*}
        \item Identification of the conduct parameter and marginal cost parameters is done under the assumption that the demand function is twice continuously differentiable and the marginal cost is linear along with the proof in \citet{lau1982identifying}. The proof works because the conduct parameter and the marginal cost parameters are separable. In this case, the supply relationship becomes a linear function of the conduct and the marginal cost parameters. On the other hand, what we need to consider is identification when the conduct parameter and the marginal cost parameters are not separable, as in section 2.2 and 2.3.
    \end{itemize}
    
\section{Conclusion}

XXX

\appendix


\section{Formal proof}
\citet{lau1982identifying} assumes that both demand and marginal cost functions are twice continuously differentiable.

\begin{claim}
Assume that the inverse demand function satisfies twice continuously differentiability and the demand rotation IV and the aggregate quantity are non-separable, and the marginal cost function is linear. The conduct parameter and the marginal cost parameters are identified.
\end{claim}

Let $P = f(Q,z_1)$ be a twice continuously differentiable function and $MC = \gamma_0 + \gamma_1 Q + \gamma_2 z_2$.
For the sake of a contradiction, suppose that the parameters are not identified. This implies that for $(\theta, \gamma_0, \gamma_1, \gamma_2)\ne (\theta', \gamma_0', \gamma_1', \gamma_2')$, two first-order conditions
\begin{align}
    f(Q,z_1) &= - \theta \frac{\partial f}{\partial Q}(Q,z_1) Q + \gamma_0 + \gamma_1 Q + \gamma_2 z_2 \label{eq:non_identification_1}\\
    f(Q,z_1) &= - \theta' \frac{\partial f}{\partial Q}(Q,z_1) Q + \gamma_0' + \gamma_1' Q + \gamma_2' z_2 \label{eq:non_identification_2},
\end{align}
where the reduced form function $Q= h_2(z_1,z_2)$ and $Q= h_2'(z_1,z_2)$ from \eqref{eq:non_identification_1} and \eqref{eq:non_identification_2} respectively are identical.

This in turn implies
\begin{align}\label{eq:equivalence_1}
    \frac{\partial h_2}{\partial z_1} = \frac{\frac{\partial f}{\partial z_1} + \theta \frac{\partial^2 f}{\partial z_1\partial Q}Q}{(1 + \theta)\frac{\partial f}{\partial Q} + \theta \frac{\partial^2 f}{\partial Q^2}Q - \gamma_0} = \frac{\partial h_2'}{\partial z_1} = \frac{\frac{\partial f}{\partial z_1} + \theta' \frac{\partial^2 f}{\partial z_1\partial Q}Q}{(1 + \theta')\frac{\partial f}{\partial Q} + \theta' \frac{\partial^2 f}{\partial Q^2}Q - \gamma_0'}
\end{align}
and 
\begin{align}\label{eq:equivalence_2}
    \frac{\partial h_2}{\partial z_2} = -\frac{\gamma_2}{(1 + \theta)\frac{\partial f}{\partial Q} + \theta \frac{\partial^2 f}{\partial Q^2}Q - \gamma_0} = \frac{\partial h_2'}{\partial z_2} = -\frac{\gamma_2'}{(1 + \theta')\frac{\partial f}{\partial Q} + \theta' \frac{\partial^2 f}{\partial Q^2}Q - \gamma_0'}
\end{align}

From \eqref{eq:equivalence_1} and \eqref{eq:equivalence_2}, we have
\begin{align}\label{eq:identification}
    &\frac{\frac{\partial f}{\partial z_1} + \theta \frac{\partial^2 f}{\partial z_1\partial Q}Q}{\frac{\partial f}{\partial z_1} + \theta' \frac{\partial^2 f}{\partial z_1\partial Q}Q} = \frac{\gamma_2}{\gamma_2'}\nonumber\\
    \Longrightarrow\ & \frac{\theta \gamma_2' - \theta' \gamma_2}{\gamma_2' - \gamma_2}\frac{\partial^2 f}{\partial z_1\partial Q}Q  +    \frac{\partial f}{\partial z_1} = 0
\end{align}
Recall that we assumed that $\gamma_2\ne \gamma_2'$ and $\theta\ne \theta'$. When $ \theta \gamma_2' - \theta' \gamma_2 = 0$, the left hand side of \eqref{eq:identification} implies that $\frac{\partial f}{\partial z_1} = 0$ for all $z_1$, which contradicts to the fact that $f$ is twice continuously differentiable. When  $ \theta \gamma_2' - \theta' \gamma_2 \ne 0$, \eqref{eq:identification} implies that for $i \ne j$, 
\begin{align}
    \frac{\partial }{\partial Q}\left( {\frac{\partial f}{\partial z_{1i}}}/{\frac{\partial f}{\partial z_{1j}}}    \right) = 0
\end{align}
which implies when $Q$ and $z$ are separable. However this also contradicts to the assumption.
Therefore, we can conclude that the parameters are identified.




\bibliographystyle{aer}

\bibliography{conduct_parameter}


% https://github.com/Yuri-Matsumura
\end{document}