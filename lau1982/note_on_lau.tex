\documentclass[11pt, a4paper]{article}
\usepackage[utf8]{inputenc}
\usepackage{amsmath,setspace,geometry}
\usepackage{amsthm}
\usepackage{amsfonts}
\usepackage{mathtools}
\mathtoolsset{showonlyrefs}
\usepackage[shortlabels]{enumitem}
\usepackage{rotating}
\usepackage{pdflscape}
\usepackage{graphicx}
\usepackage{bbm}
\usepackage[dvipsnames]{xcolor}
\usepackage{hyperref}
\hypersetup{colorlinks=true, linkcolor= BrickRed, citecolor = BrickRed, filecolor = BrickRed, urlcolor = BrickRed, hypertexnames = true}
\usepackage[]{natbib} 
\bibpunct[:]{(}{)}{,}{a}{}{,}
\geometry{left = 1.0in,right = 1.0in,top = 1.0in,bottom = 1.0in}
\usepackage[english]{babel}
\usepackage{float}
\usepackage{caption}
\usepackage{subcaption}
\usepackage{tikz}
\usepackage{booktabs}
\usepackage{pdfpages}
\usepackage{threeparttable}
\usepackage{lscape}
\usepackage{bm}
\setstretch{1.4}
%\usepackage[tablesfirst,nolists]{endfloat}

\newtheorem{theorem}{Theorem}
\newtheorem{assumption}{Assumption}
\newtheorem{lemma}{Lemma}
\newtheorem{definition}{Definition}
\newtheorem{proposition}{Proposition}
\newtheorem{claim}{Claim}
\newtheorem{corollary}{Corollary}
\newtheorem{example}{Example}
\DeclareMathOperator{\rank}{rank}

\theoremstyle{remark}
\newtheorem{remark}{Remark}


\title{Note on Lau (1982)}
\author{Yuri Matsumura\thanks{Department of Economics, Rice University. Email: \href{mailto:}{yuri.matsumura23@gmail.com}} \and Suguru Otani \thanks{\href{mailto:}{suguru.otani@e.u-tokyo.ac.jp}, Market Design Center, University of Tokyo
\\Declarations of interest: none %this is for Economics Letters
}}

\begin{document}

\maketitle
\begin{abstract}
    TBA
\end{abstract}

\noindent\textbf{Keywords:} Conduct parameters, Homogenous Goods Market, Mathematical Programming with Equilibrium Constraints, Monte Carlo simulation
\vspace{0in}
\newline
\noindent\textbf{JEL Codes:} C5, C13, L1

\bigskip

\section{Introduction}
TBA
\section{Model}
Consider a homogeneous product market.
The aggregate demand and aggregate marginal cost function is given as
\begin{align}
    P = f(Q, X^{d}), \label{eq:demand}
    \\
    MC = g(Q, X^{c}),\label{eq:marginal_cost}
\end{align}
where $Q$ is the aggregate quantity, $X^{d}$ and $X^{c}$ are the vector of exogenous variables.
Assume that $X^{d}$ and $X^{s}$ are exclusive, that is, there is no common variable in $X^{d}$ and $X^{s}$.
Then, we obtain the supply equation:
\begin{align}
     P + \theta\frac{\partial P(Q)}{\partial Q}Q = MC,\label{eq:supply_equation}
\end{align}
where $\theta\in[0,1]$, which is the conduct parameter.

\citet{lau1982identifying} state the definition of non-identification in this model:
\begin{definition}
    To show necessity we observe that non-identification implies
    \begin{align}
    f(Q, X^{d}) + \theta \frac{\partial f}{\partial Q}(Q, X^{d})Q &= g(Q, X^{s}),\label{eq:foc_alpha}\\
    f(Q, X^{d}) + \theta' \frac{\partial f}{\partial Q}(Q, X^{d})Q &= g'(Q, X^{s}), \label{eq:foc_beta}
    \end{align}
    where $\theta \neq \theta'$ and the reduced form functions $Q = h_2(X^{d}, X^{s})$ and $Q = h_2'(X^{d}, X^{s})$ defined by \eqref{eq:foc_alpha} and \eqref{eq:foc_beta} respectively are identical.
\end{definition}

\citet{lau1982identifying} provide the condition where the conduct parameter cannot be identified:
\begin{theorem}\label{theorem_lau}
    Under the assumption that the industry inverse demand and cost functions are twice continuously differentiable, the index of competitiveness $\theta$ cannot be identified from data on industry price and output and other exogenous variables alone if and only if the industry inverse demand function is separable in $X^{d}$, 
    \begin{align}
        P = f(Q, r(X^{d})), \label{eq:demand_separable}
    \end{align}
    for some function $r(\cdot)$, but does not take the form
    \begin{align}
        P = Q^{-1/\theta}r(X^{d}) + s(Q). \label{eq:identification_separable}
    \end{align}
\end{theorem}
This theorem implies that at least when the demand function is not separable, the conduct parameter is identified.
Even when the demand function is separable, the conduct parameter is identified if it satisfies \eqref{eq:identification_separable}.

\begin{remark}
    The model of \citet{bresnahan1982oligopoly} is not a special case of \citet{lau1982identifying} because the twice continuous differentiability does not hold.
    \citet{bresnahan1982oligopoly} consider a non-separable demand function such that $P = \alpha_0 + (\alpha_1 + \alpha_2 X^0) Q + \alpha_3 X^1 + \varepsilon$ where $X^{d} = (X^0, X^1)$ and $X^0$ is called a demand rotation parameter.
\end{remark}


\section{Summary of Goldman and Uzawa (1964)}
Before investigating the proof in \citet{lau1982identifying}, we provide results from \citet{goldmanNote1964}.
They play an important role in Lau's proof.

\citet{goldmanNote1964} investigate separability concepts in demand analysis.
Let $N$ be a set of product, $N = \{1,\ldots, n\}$.
Consider a partition of $N$ into $S$ parts, $\{N^1, \ldots, N^S\}$ such that $N = \bigcup_{s=1}^S N^s$ and $N^s \cap N^t = \emptyset$.

A utility function is weakly separable with respect to a partition if 
\begin{align}
    \frac{\partial u_i(x)/ u_j(x)}{\partial x_k} = 0, \quad i,j\in N^s, k \notin N^s.
\end{align}
This implies that the marginal rate of substitution between $i$ and $j$ in the same partition is independent of the quantities of commodities outside $N^s$.

To show Theorem 2, \citet{goldmanNote1964} prove the following lemma:
\begin{lemma}\label{lemma_1_GU}
    Let $f(x)$ and $g(x)$ be two continuously twice-differentiable functions of $n$ variables $x=(x_1, \dots, x_n)$. If each indifference surface is connected, and if there exists a function $\lambda(x)$ such that
    \begin{align}
    f_i(x) &= \lambda(x)g_i(x), \quad i=1, \dots, n, \quad \text{for all } x,
    \end{align}
    then $f(x)$ is a transformation of $g(x)$; namely, there exists a function $F(t)$ of one variable such that
    \begin{align}
    f(x) &= F(g(x)), \quad \text{for all } x.
    \end{align}
    Hence, in particular, the function $\lambda(x)$ satisfying (4) must be of the form:
    \begin{align}
        \lambda(x) &= \Lambda(g(x)), \quad \text{for all } x,
    \end{align}
    with some function $\Lambda(t)$ of one variable.
\end{lemma}

Theorem 2 of \citet{goldmanNote1964} shows the utility function that satisfies the weak separability:
\begin{theorem}[Theorem 2 in \citet{goldmanNote1964}]\label{thorem_2_GU}
    A utility function $u(x)$ is weakly separable with respect to a partition $\{N^1, .. ., N^s\}$ if, and only if, $u(x)$ is of the form: 
    \begin{align}
        u(X) = \Phi(u^1(x^{(1)}),\ldots, u^s(x^{(s)})   )
    \end{align} where $\Phi(u^1,\ldots, u^s)$ is a function of $S$ variables and, for each $s$, $u^s(x^{(s)})$ is a function subvector $x^{(s)}$ alone.
\end{theorem}

To see the relationship to the model in \citet{lau1982identifying}, let $\{Q, X_{1}^{d},\ldots, X_{K}^{d}\}$ be the set of dependent variables and divide the variables into two groups $(Q, X^{d})$.
In this case, the separable definition \ref{eq:demand_separable} in Theorem \ref{theorem_lau} satisfies the weak-separability.

For example, the linear demand function in \citet{bresnahan1982oligopoly} is not separable.
\begin{align}
    \frac{f_Q}{f_Y} = \frac{(\alpha_1 + \alpha_2 Z)}{\alpha_3} \Longrightarrow \frac{\partial f_Q/f_Y}{\partial Z} = \frac{\alpha_2}{\alpha_3} \ne 0.
\end{align}
$\alpha_2/\alpha_3 \ne 0$ comes from the sufficient condition in \citet{matsumura2023resolving}.
Thus the conduct parameter can be identified.



\section{Proof in Lau(1982)}
Now, let's investigate the proof of \citet{lau1982identifying}.
Taking derivatives with $X^{d}$ and $X^{s}$ for both \eqref{eq:foc_alpha}
and \eqref{eq:foc_beta}.
From \eqref{eq:foc_alpha},
\begin{align}
    & \frac{\partial f}{\partial Q}\frac{\partial h_2}{\partial X^{d}} + \frac{\partial f}{\partial X^{d}} + \theta\left(\frac{\partial f}{\partial Q}\frac{\partial h_2}{\partial X^{d}}  + \frac{\partial^2 f}{\partial Q^2}Q\frac{\partial h_2}{\partial X^{d}} + \frac{\partial^2 f}{\partial X^{d}\partial Q}Q \right) = \frac{\partial g}{\partial Q}\frac{\partial h_2}{\partial X^{d}}\label{eq:foc_derivative_demand}\\
    & \frac{\partial f}{\partial Q}\frac{\partial h_2}{\partial X^{s}} + \theta \left(\frac{\partial f}{\partial Q}\frac{\partial h_2}{\partial X^{s}}  + \frac{\partial^2 f}{\partial Q^2}Q\frac{\partial h_2}{\partial X^{s}} \right) =  \frac{\partial g}{\partial Q}\frac{\partial h_2}{\partial X^{s}} + \frac{\partial g}{\partial X^{s}}\label{eq:foc_derivative_supply}.
\end{align}
\eqref{eq:foc_derivative_demand} and \eqref{eq:foc_derivative_supply} also hold for \eqref{eq:foc_beta}.
As the non-identification implies that $h_2(X^{d}, X^{s}) = h_2'(X^{d}, X^{s})$ for all $X^{d}$ and $X^{s}$,
\begin{align}
    \frac{\partial h_2}{\partial X^{d}} & = - \frac{\frac{\partial f}{\partial X^{d}} + \theta \frac{\partial^2 f}{\partial X^{d}\partial Q}Q }{(1+\theta)\frac{\partial f}{\partial Q} + \frac{\partial^2 f}{\partial Q^2}Q - \frac{\partial g}{\partial Q}}\\
    & = - \frac{\frac{\partial f}{\partial X^{d}} + \theta' \frac{\partial^2 f}{\partial X^{d}\partial Q}Q }{(1+\theta')\frac{\partial f}{\partial Q} + \frac{\partial^2 f}{\partial Q^2}Q - \frac{\partial g'}{\partial Q}}= \frac{\partial h_2'}{\partial X^{d}},
\end{align}
and
\begin{align}
    \frac{\partial h_2}{\partial X^{s}} & = -\frac{-\frac{\partial g}{\partial X^{s}}}{(1+\theta)\frac{\partial f}{\partial Q} + \frac{\partial^2 f}{\partial Q^2}Q - \frac{\partial g}{\partial Q}} \\
     & = -\frac{-\frac{\partial g'}{\partial X^{s}}}{(1+\theta')\frac{\partial f}{\partial Q} + \frac{\partial^2 f}{\partial Q^2}Q - \frac{\partial g'}{\partial Q}} = \frac{\partial h_2'}{\partial X^{s}}.
\end{align}
Let
\begin{align}
    \mu(Q,X^{s}) \equiv \frac{(1+\theta)\frac{\partial f}{\partial Q} + \frac{\partial^2 f}{\partial Q^2}Q - \frac{\partial g}{\partial Q}}{(1+\theta')\frac{\partial f}{\partial Q} + \frac{\partial^2 f}{\partial Q^2}Q - \frac{\partial g'}{\partial Q}}.
\end{align}
Then 
\begin{align}
    \frac{\partial g}{\partial X^{s}} = \lambda(Q, X^{s}) \frac{\partial g'}{\partial X^{s}}. \label{eq:transformation_g}
\end{align}
By Lemmna \ref{lemma_1_GU}, there is a function $F$ such that
\begin{align}
    g(Q,X^{s}) = F(g'(Q,X^{s}), Q).
\end{align}
By substituting \eqref{eq:foc_alpha} and \eqref{eq:foc_beta},
\begin{align}
    f(Q, X^{d}) + \theta \frac{\partial f}{\partial Q}(Q, X^{d})Q  = F\left(f(Q, X^{d}) + \theta' \frac{\partial f}{\partial Q}(Q, X^{d})Q, Q \right) \label{eq:transform_theta}
\end{align}
Differentiate \eqref{eq:transform_theta} with respect to $X^{d}$, obtaining
\begin{align}
\frac{\partial f}{\partial X^{d}}(Q, X^{d}) + \theta' \frac{\partial^2 f}{\partial X^{d} \partial Q}(Q, X^{d})Q &= F_{1} \cdot \left[\frac{\partial f}{\partial X^{d}}(Q, X^{d}) + \theta \frac{\partial^2 f}{\partial X^{d} \partial Q}(Q, X^{d})Q\right]\label{eq:transform_theta_derivative_first}\\
(1 - F_{1}) \frac{\partial f}{\partial X^{d}}(Q, X^{d}) & = (F_{1} \theta - \theta')\frac{\partial^2 f}{\partial X^{d} \partial Q}(Q, X^{d})Q,\label{eq:transform_theta_derivative_second}
\end{align}
where $F_1$ is the derivative of $F$ with respect to the first element. 

Pick $i$ and $j$th elements in \eqref{eq:transform_theta_derivative_second} and take a ratio of them implies
\begin{align}
\frac{(\partial f/\partial X^{d}_{i})(Q, X^{d})}{(\partial f/\partial X^{d}_{j})(Q, X^{d})} & = \frac{(\partial f/\partial X^{d}_{i} \partial Q)}{(\partial f/\partial X^{d}_{j} \partial Q)},\label{eq:ratio_foc}
\end{align}
which turn implies
\begin{align}
    \frac{d}{dQ}\left(\frac{(\partial f/\partial X^{d}_{i})(Q, X^{d})}{(\partial f/\partial X^{d}_{j})(Q, X^{d})}\right) =0.\label{eq:derivative_separable}
\end{align}
Thus $f$ must be a separable function due to Theorem \ref{thorem_2_GU}.

\subsection{Questions to the proof}
First, note that the left-hand side of \eqref{eq:foc_derivative_demand} and \eqref{eq:transform_theta_derivative_first} are both the derived of the left-hand side of \eqref{eq:foc_alpha} with respect to $X^{d}$.
However, the former allows the differentiation of $Q = h_2(X^{d}, X^{s})$ with respect to $X^{d}$, the latter does not.
In other words, the former views $f(Q, X^{d})$ as $f(h_2(X^{d}, X^{s}), X^{d})$ but the latter assumes that $Q$ does not depend on $X^{d}$.
The independence between $Q$ and $X^{d}$ is the key to show the separability of $f$.
In fact, separability in this model does not hold because $Q$ is an equilibrium object and thus always depends on $X^{d}$. 
Thus $f(Q, r(X^{d}))$ is always written as $f(h_2(X^{d}, X^{s}), r(X^{d}))$, which contradict to the separability assumption.

When we allow the differentiation of $Q$ with respect to $X^{d}$, \eqref{eq:ratio_foc} becomes
\begin{align}
    \frac{\frac{\partial f}{\partial X^{d}_{i}} + \frac{\partial f}{\partial Q} \frac{\partial Q}{\partial X^{d}_{i}}}{\frac{\partial f}{\partial X^{d}_{j}} + \frac{\partial f}{\partial Q} \frac{\partial Q}{\partial X^{d}_{j}}} = \frac{\frac{\partial f}{\partial Q}\frac{\partial Q}{\partial X^{d}_{i}}  + \frac{\partial^2 f}{\partial Q^2}Q\frac{\partial Q}{\partial X^{d}_{i}} + \frac{\partial^2 f}{\partial X^{d}_{i}\partial Q}Q }{\frac{\partial f}{\partial Q}\frac{\partial Q}{\partial X^{d}_{j}}  + \frac{\partial^2 f}{\partial Q^2}Q\frac{\partial Q}{\partial X^{d}_{j}} + \frac{\partial^2 f}{\partial X^{d}_{j}\partial Q}Q }.
\end{align}

Additionally, \eqref{eq:derivative_separable} shows weak-separability, but why this holds from \eqref{eq:ratio_foc}?

\subsection{A counterexample}

I show that the conduct parameter is identified when the inverse demand function is separable. 

Consider the following inverse demand function and marginal cost function:
\begin{align}
    P & = \exp(\varepsilon_{d}) Q^{\alpha_0} X_{d1}^{\alpha_1}X_{d2}^{\alpha_2}\label{eq:counter_demand}\\
    MC & = \exp(\varepsilon_{s})Q^{\beta_0} X_{s1}^{\beta_1} X_{s2}^{\beta_2}.\label{eq:counter_mc}
\end{align}
Assume that $\alpha_0 <0$.
The inverse demand function is separable because
\begin{align}
    \frac{d}{dQ} \left(\frac{\partial P/\partial X_{d1}}{\partial P/\partial X_{d2}} \right) = \frac{d}{dQ} \left(\frac{\alpha_{1}\exp(\varepsilon_{d}) Q^{-\alpha_0} X_{d1}^{\alpha_1-1}X_{d2}^{\alpha_2}}{\alpha_2\exp(\varepsilon_{d}) Q^{-\alpha_0} X_{d1}^{\alpha_1}X_{d2}^{\alpha_2-1}} \right) =  \frac{d}{dQ}\left(\frac{\alpha_1}{\alpha_2} \frac{X_{d2}}{X_{d1}} \right)=0.
\end{align}
By taking logarithm to \eqref{eq:counter_demand}, we have a demand equation such that 
\begin{align}
    \log P = \alpha_0 \log Q + \alpha_1 \log X_{d1}  + \alpha_2 \log X_{d2} + \varepsilon_{d}.\label{eq:counter_demand_equation}
\end{align}


From the first-order condition, 
\begin{align}
    P + \theta (\alpha_0 \exp(\varepsilon_{d})Q^{\alpha_0-1}X_{d1}^{\alpha_1}X_{d2}^{\alpha_2}) Q & = Q^{\beta_0} X_{s1}^{\beta_1}X_{s2}^{\beta_2}\exp(\varepsilon_{s})\\
    P + \theta \alpha_0 P & = Q^{\beta_0} X_{s1}^{\beta_1}X_{s2}^{\beta_2}\exp(\varepsilon_{s})\\
    (1 + \theta\alpha_0) P & = Q^{\beta_0} X_{s1}^{\beta_1}X_{s2}^{\beta_2}\exp(\varepsilon_{s}).
\end{align}
By taking a logarithm, we obtain a supply equation,
\begin{align}
    \log P = - \log(1 + \theta\alpha_0) + \beta_0 \log Q + \beta_1 \log X_{s1}+\beta_2 \log X_{s2} + \varepsilon_{s}.\label{eq:counter_supply_equation}
\end{align}
By solving \eqref{eq:counter_demand_equation} and \eqref{eq:counter_supply_equation}, the equilibrium quantity $Q$ is obtained as
\begin{align}
    \log Q = \frac{\beta_1 \log X_{s1}+\beta_2 \log X_{s2} - \log(1 + \theta\alpha_0)+ \varepsilon_{s} - \alpha_1 \log X_{d1}  - \alpha_2 \log X_{d2} - \varepsilon_{d} }{\alpha_0 - \beta_0}.
\end{align}

Note that the demand parameters can be identified when $X^s$ is a vector of exclusive demand instruments.
Thus we can assume that $\alpha_0, \alpha_1$, and $\alpha_2$ is known.  

Then it is easy to see that $- \log(1 + \theta\alpha_0), \beta_0$, and $\beta_1$ are identified with a vector of exclusive supply instrument $X^d$.
Let $\gamma = - \log(1 + \theta\alpha_0)$, then $\theta = (\exp(-\gamma) - 1)/\alpha_0$.
Because $\alpha_0$ is identified, the parameter $\theta$ is also identified, which contradicts Theorem \ref{theorem_lau}.

\begin{remark}
    The above inverse demand function has a constant-elasticity property. 
    Note that $\alpha_0 \ne \frac{1}{\theta}$ holds.
    Thus it is not the separable function that allows identification in Theorem \ref{theorem_lau}.
    However, \citet{lau1982identifying} states that under this functional form, the conduct parameter cannot be identified in his comment (2) in p 98.
\end{remark}




\newpage
\bibliographystyle{aer}
\bibliography{conduct_parameter}

\end{document}