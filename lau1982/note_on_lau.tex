\documentclass[11pt, a4paper]{article}
\usepackage[utf8]{inputenc}
\usepackage{amsmath,setspace,geometry}
\usepackage{amsthm}
\usepackage{amsfonts}
\usepackage{mathtools}
\mathtoolsset{showonlyrefs}
\usepackage[shortlabels]{enumitem}
\usepackage{rotating}
\usepackage{pdflscape}
\usepackage{graphicx}
\usepackage{bbm}
\usepackage[dvipsnames]{xcolor}
\usepackage{hyperref}
\hypersetup{colorlinks=true, linkcolor= BrickRed, citecolor = BrickRed, filecolor = BrickRed, urlcolor = BrickRed, hypertexnames = true}
\usepackage[]{natbib} 
\bibpunct[:]{(}{)}{,}{a}{}{,}
\geometry{left = 1.0in,right = 1.0in,top = 1.0in,bottom = 1.0in}
\usepackage[english]{babel}
\usepackage{float}
\usepackage{caption}
\usepackage{subcaption}
\usepackage{tikz}
\usepackage{booktabs}
\usepackage{pdfpages}
\usepackage{threeparttable}
\usepackage{lscape}
\usepackage{bm}
\setstretch{1.4}
%\usepackage[tablesfirst,nolists]{endfloat}

\newtheorem{theorem}{Theorem}
\newtheorem{assumption}{Assumption}
\newtheorem{lemma}{Lemma}
\newtheorem{definition}{Definition}
\newtheorem{proposition}{Proposition}
\newtheorem{claim}{Claim}
\newtheorem{corollary}{Corollary}
\newtheorem{example}{Example}
\DeclareMathOperator{\rank}{rank}

\theoremstyle{remark}
\newtheorem{remark}{Remark}


\title{Note on Lau (1982)}
\author{Yuri Matsumura\thanks{\href{mailto:}{yuri.matsumura23@gmail.com}, Department of Economics, Rice University.} \and Suguru Otani \thanks{\href{mailto:}{suguru.otani@e.u-tokyo.ac.jp}, Market Design Center, University of Tokyo
\\Declarations of interest: none %this is for Economics Letters
}}

\begin{document}

\maketitle
\begin{abstract}
    TBA
\end{abstract}

\noindent\textbf{Keywords:} Conduct parameters, Homogenous Goods Market, Mathematical Programming with Equilibrium Constraints, Monte Carlo simulation
\vspace{0in}
\newline
\noindent\textbf{JEL Codes:} C5, C13, L1

\bigskip

\section{Introduction}
This note provides a counterexample to \citet{lau1982identifying}.
Measuring market power is a first-order task in the literature on industrial organization.
The conduct parameter approach is a way of measuring market power and the empirical literature also investigates the identification of the parameter.
The seminal paper is \citet{bresnahan1982oligopoly} which identifies the parameter in a homogeneous product market with a linear demand function and a linear marginal cost function.
\citet{lau1982identifying} generalizes the identification in \citet{bresnahan1982oligopoly}.
\citet{lau1982identifying} investigated the setting where the parameter cannot be identified.
However, we find a counterexample to his result. 

First, we check the model introduced in \citet{lau1982identifying}.
While \citet{bresnahan1982oligopoly} considers a linear demand and a linear marginal cost model, \citet{lau1982identifying} considers a more general demand and marginal cost function.
Because the demand function can be identified, the main problem is the simultaneous identification of the conduct parameter and marginal cost function from the supply function, which is derived from the first-order condition of the firm's maximization problem. 

\citet{lau1982identifying} claimed that when the demand function is separable, the conduct parameter and the marginal cost function cannot be identified.



\section{Model}
Consider a homogeneous product market.
The aggregate demand and aggregate marginal cost function is given as
\begin{align}
    P = f(Q, X^{d}), \label{eq:demand}
    \\
    MC = g(Q, X^{c}),\label{eq:marginal_cost}
\end{align}
where $Q$ is the aggregate quantity, $X^{d}$ and $X^{c}$ are the vector of exogenous variables.
Assume that $X^{d}$ and $X^{s}$ are exclusive, that is, there is no common variable in $X^{d}$ and $X^{s}$.
Then, we obtain the supply equation:
\begin{align}
     P + \theta\frac{\partial P(Q)}{\partial Q}Q = MC,\label{eq:supply_equation}
\end{align}
where $\theta\in[0,1]$, which is the conduct parameter.

As the demand function can be identified, \citet{lau1982identifying} tries to simultaneously identify the conduct parameter and the marginal cost function from the supply equation \eqref{eq:supply_equation}.

\citet{lau1982identifying} does not show identification but shows the condition for non-identification.
\citet{lau1982identifying} states the definition of non-identification in this model is the following:
\begin{definition}\label{def:non_identification}
    Non-identification implies
    \begin{align}
    f(Q, X^{d}) + \theta \frac{\partial f}{\partial Q}(Q, X^{d})Q &= g(Q, X^{s}),\label{eq:foc_alpha}\\
    f(Q, X^{d}) + \theta' \frac{\partial f}{\partial Q}(Q, X^{d})Q &= g'(Q, X^{s}), \label{eq:foc_beta}
    \end{align}
    where $\theta \neq \theta'$, $g \ne g'$,\footnote{This condition ins not stated in \citet{lau1982identifying}, but assuming $g =\ne g'$ makes the identification simple. See Appendix.} and the reduced form functions $Q = h(X^{d}, X^{s})$ and $Q = h'(X^{d}, X^{s})$ defined by \eqref{eq:foc_alpha} and \eqref{eq:foc_beta} respectively are identical.
\end{definition}
Intuitively, this says that the two models under different conduct parameters and marginal cost functions generate the same equilibrium quantity and price, which implies that the researcher cannot distinguish the two models from the observed data.

Then, \citet{lau1982identifying} provides the condition on the demand function where the conduct parameter cannot be identified:
\begin{theorem}\label{theorem_lau}
    Under the assumption that the industry inverse demand and cost functions are twice continuously differentiable, the index of competitiveness $\theta$ cannot be identified from data on industry price and output and other exogenous variables alone if and only if the industry inverse demand function is separable in $X^{d}$, 
    \begin{align}
        P = f(Q, r(X^{d})), \label{eq:demand_separable}
    \end{align}
    but not take the form, 
    \begin{align}
        P = Q^{-1/\theta}r(X^{d}) + s(Q). \label{eq:identification_separable}
    \end{align}
\end{theorem}
This theorem implies that the conduct parameter is identified when the demand function is not separable.
Even when the demand function is separable, the conduct parameter is identified if it satisfies \eqref{eq:identification_separable}.

\begin{remark}
    \citet{bresnahan1982oligopoly} considers a model with linear demand and marginal cost.
    He considers a demand function such that $P = \alpha_0 + (\alpha_1 + \alpha_2 Z) Q + \alpha_3 Y + \varepsilon$ where $Z$ is called a demand rotation IV.
    Under the demand, the conduct parameter and the marginal cost parameter can be identified.
    \citet{matsumura2023resolving} provide more detailed conditions for the identification.
\end{remark}

\section{Summary of Goldman and Uzawa (1964)}

In section \ref{sec:proof_lau}, we review the proof of \citet{lau1982identifying}.
As his proof relies on the definitions and results in \citet{goldmanNote1964}, which investigate separability concepts in demand analysis, we review the paper.

Let $n$ be the number of variables and $x = (x_{1},\ldots, x_{n})$ be a vector of $n$ variables.
Consider a partition of $X$ into $K$ parts, $\{x^1, \ldots, x^K\}$ such that $X = \bigcup_{k=1}^K x^k$ and $x^k \cap x^l = \emptyset$ for $k\ne l$.

A function is weakly separable with respect to a partition if 
\begin{align}
    \frac{\partial f_i(x)/ f_j(x)}{\partial x_l} = 0, \quad i,j\in x^k, l \notin x^k.
\end{align}
When $f$ is a utility function, this implies that the marginal rate of substitution between commodity $i$ and $j$ in the same partition is independent of the quantities of commodities outside $x^k$.

To show Theorem 2, \citet{goldmanNote1964} prove the following lemma:
\begin{lemma}\label{lemma_1_GU}
    Let $f(x)$ and $g(x)$ be two continuously twice-differentiable functions of $n$ variables $x=(x_1, \dots, x_n)$. If each indifference surface is connected, and if there exists a function $\lambda(x)$ such that
    \begin{align}
    f_i(x) &= \lambda(x)g_i(x), \quad i=1, \dots, n, \quad \text{for all } x, \label{eq:transform_f}
    \end{align}
    then $f(x)$ is a transformation of $g(x)$; namely, there exists a function $F(\cdot)$ of one variable such that
    \begin{align}
    f(x) &= F(g(x)), \quad \text{for all } x.
    \end{align}
    Hence, in particular, the function $\lambda(x)$ satisfying \eqref{eq:transform_f} must be of the form:
    \begin{align}
        \lambda(x) &= \Lambda(g(x)), \quad \text{for all } x,
    \end{align}
    with some function $\Lambda(\cdot)$ of one variable.
\end{lemma}

Theorem 2 of \citet{goldmanNote1964} shows the utility function that satisfies the weak separability:
\begin{theorem}[Theorem 2 in \citet{goldmanNote1964}]\label{thorem_2_GU}
    A function $f(x)$ is weakly separable with respect to a partition $\{x^1, .. ., x^K\}$ if and only if $f(x)$ is of the form: 
    \begin{align}
        f(X) = \Phi(r^1(x^{1}),\ldots, r^K(x^{K})   )
    \end{align} where $\Phi(r^1,\ldots, r^K)$ is a function of $K$ variables and, for each $k$, $r^k(x^{k})$ is a function subvector $x^{k}$ alone.
\end{theorem}

\citet{lau1982identifying} assumes that the set of variables in the demand function as  $\{Q, X_{1}^{d},\ldots, X_{K}^{d}\}$  divides the variables into two groups $(Q, X^{d})$.
Then, the separable definition of \eqref{eq:demand_separable} satisfies the weak-separability because he defines $r^1(Q) = Q$ and $r^2(X^d) = r(X^d)$.

\begin{remark}
    The linear demand function in \citet{bresnahan1982oligopoly} is not separable:
    \begin{align}
        \frac{f_Z}{f_Y} = \frac{\alpha_2 Q}{\alpha_3} \Longrightarrow \frac{\partial f_Z/f_Y}{\partial Q} = \frac{\alpha_2}{\alpha_3} \ne 0,
    \end{align}
    where $\alpha_2/\alpha_3 \ne 0$ comes from the sufficient condition for the identification in \citet{matsumura2023resolving}.
    Thus the conduct parameter can be identified.
\end{remark}



\section{Proof in Lau(1982)}\label{sec:proof_lau}
Now, let's investigate the proof of \citet{lau1982identifying}.
Note that we fill some gaps between the lines in the original proof in \citet{lau1982identifying}.

\subsection{Necessity}
\citet{lau1982identifying} shows that non-identification implies that the demand function is separable.
First, by taking the derivative of the first-order condition \eqref{eq:foc_alpha} with respect to $X^{d}$, we obtain
\begin{align}
    \frac{\partial g}{\partial Q}\frac{\partial h}{\partial X^{d}} = & \frac{\partial f}{\partial Q}\frac{\partial h}{\partial X^{d}} + \frac{\partial f}{\partial X^{d}} + \theta\left(\frac{\partial f}{\partial Q}\frac{\partial h}{\partial X^{d}}  + \frac{\partial^2 f}{\partial Q^2}Q\frac{\partial h}{\partial X^{d}} + \frac{\partial^2 f}{\partial X^{d}\partial Q}Q \right).
\end{align}
Here, we use the reduced form equation $Q = h(X^{d}, X^{s})$.
Then, by summarizing the equation by $\frac{\partial h}{\partial X^{d}}$, we obtain
\begin{align}
    \frac{\partial h}{\partial X^{d}} = & -\frac{\frac{\partial f}{\partial X^{d}} + \theta \frac{\partial^2 f}{\partial X^{d}\partial Q}Q }{(1+\theta)\frac{\partial f}{\partial Q} + \frac{\partial^2 f}{\partial Q^2}Q - \frac{\partial g}{\partial Q}}. \label{eq:foc_derivative_demand}
\end{align}

Next, by taking derivative of \eqref{eq:foc_alpha} with respect to $X^{s}$, we obtain
\begin{align}
    & \frac{\partial f}{\partial Q}\frac{\partial h}{\partial X^{s}} + \theta \left(\frac{\partial f}{\partial Q}\frac{\partial h}{\partial X^{s}}  + \frac{\partial^2 f}{\partial Q^2}Q\frac{\partial h}{\partial X^{s}} \right) =  \frac{\partial g}{\partial Q}\frac{\partial h}{\partial X^{s}} + \frac{\partial g}{\partial X^{s}}.
\end{align}
Then we also have
\begin{align}
    \frac{\partial h}{\partial X^{s}} & = \frac{\frac{\partial g}{\partial X^{s}}}{(1+\theta)\frac{\partial f}{\partial Q} + \frac{\partial^2 f}{\partial Q^2}Q - \frac{\partial g}{\partial Q}}. \label{eq:foc_derivative_supply}
\end{align}
The similar relationships hold for \eqref{eq:foc_beta}.

As the non-identification implies that $Q = h(X^{d}, X^{s}) = h'(X^{d}, X^{s})$ for all $X^{d}$ and $X^{s}$, we have
\begin{align}
    \frac{\partial h}{\partial X^{d}}  = \frac{\partial h'}{\partial X^{d}}\quad \text{  and  } \quad \frac{\partial h}{\partial X^{s}}  = \frac{\partial h'}{\partial X^{s}}.
\end{align}
Thus from \eqref{eq:foc_derivative_demand} and \eqref{eq:foc_derivative_supply}, we obtain
\begin{align}
     - \frac{\frac{\partial f}{\partial X^{d}} + \theta \frac{\partial^2 f}{\partial X^{d}\partial Q}Q }{(1+\theta)\frac{\partial f}{\partial Q} + \frac{\partial^2 f}{\partial Q^2}Q - \frac{\partial g}{\partial Q}} = - \frac{\frac{\partial f}{\partial X^{d}} + \theta' \frac{\partial^2 f}{\partial X^{d}\partial Q}Q }{(1+\theta')\frac{\partial f}{\partial Q} + \frac{\partial^2 f}{\partial Q^2}Q - \frac{\partial g'}{\partial Q}}\label{eq:derivative_q_x_d}
\end{align}
and
\begin{align}
    \frac{\frac{\partial g}{\partial X^{s}}}{(1+\theta)\frac{\partial f}{\partial Q} + \frac{\partial^2 f}{\partial Q^2}Q - \frac{\partial g}{\partial Q}} & = \frac{\frac{\partial g'}{\partial X^{s}}}{(1+\theta')\frac{\partial f}{\partial Q} + \frac{\partial^2 f}{\partial Q^2}Q - \frac{\partial g'}{\partial Q}}.\label{eq:derivative_q_x_s}
\end{align}


By using \eqref{eq:derivative_q_x_s}, \citet{lau1982identifying} claims the following:
\begin{claim}
    From \eqref{eq:derivative_q_x_s}, there exists a function $\lambda(\cdot)$ such that
    \begin{align}
        \frac{\partial g}{\partial X^{s}} = \lambda(Q, X^{s}) \frac{\partial g'}{\partial X^{s}}. \label{eq:transformation_g}
    \end{align}
    By Lemmna \ref{lemma_1_GU}, there is a function $F$ such that
    \begin{align}
        g(Q,X^{s}) = F(g'(Q,X^{s}), Q).
    \end{align}
\end{claim}

After the claim, he substitutes \eqref{eq:foc_alpha} and \eqref{eq:foc_beta} in the above equation, obtaining
\begin{align}
    f(Q, X^{d}) + \theta \frac{\partial f}{\partial Q}(Q, X^{d})Q  = F\left(f(Q, X^{d}) + \theta' \frac{\partial f}{\partial Q}(Q, X^{d})Q, Q \right).
    \label{eq:transform_theta}
\end{align}
Differentiate \eqref{eq:transform_theta} with respect to $X^{d}$, obtaining
\begin{align}
\frac{\partial f}{\partial X^{d}}(Q, X^{d}) + \theta' \frac{\partial^2 f}{\partial X^{d} \partial Q}(Q, X^{d})Q &= F_{1} \cdot \left[\frac{\partial f}{\partial X^{d}}(Q, X^{d}) + \theta \frac{\partial^2 f}{\partial X^{d} \partial Q}(Q, X^{d})Q\right]\label{eq:transform_theta_derivative_first}\\
(1 - F_{1}) \frac{\partial f}{\partial X^{d}}(Q, X^{d}) & = (F_{1} \theta - \theta')\frac{\partial^2 f}{\partial X^{d} \partial Q}(Q, X^{d})Q,\label{eq:transform_theta_derivative_second}
\end{align}
where $F_1$ is the derivative of $F$ with respect to the first element.

Pick $i$ and $j$th elements in \eqref{eq:transform_theta_derivative_second} and take a ratio of them implies
\begin{align}
\frac{\partial f/\partial X^{d}_{i}}{\partial f/\partial X^{d}_{j}} & = \frac{\partial^2 f/\partial X^{d}_{i} \partial Q}{\partial^2 f/\partial X^{d}_{j} \partial Q}.\label{eq:ratio_foc}
\end{align}
From this, we have
\begin{align}
    \frac{\partial }{\partial Q} \log\left( \frac{\partial f}{\partial X^{d}_{i}}\right) = \frac{\partial^2 f/\partial X^{d}_{i} \partial Q}{\partial f/\partial X^{d}_{i}}  = \frac{\partial^2 f/\partial X^{d}_{j} \partial Q}{\partial f/\partial X^{d}_{j}} = \frac{\partial }{\partial Q} \log\left( \frac{\partial f}{\partial X^{d}_{j}}\right).
\end{align}
Note that we can exchange the order of partial derivative due to Young's theorem.
Therefore, we have
\begin{align}
    0 & = \frac{\partial}{\partial Q}\log\left(\frac{\partial f/\partial X^{d}_{i}}{\partial f/\partial X^{d}_{j}}\right)\\
    & = \frac{1}{\frac{\partial f/\partial X^{d}_{i}}{\partial f/\partial X^{d}_{j}}} \frac{\partial}{\partial Q} \left(\frac{\partial f/\partial X^{d}_{i}}{\partial f/\partial X^{d}_{j}}\right).
    \label{eq:derivative_separable}
\end{align}
Note that it can be assume that $\partial f/\partial X^{d}_{i}\ne 0$, nevertheless, it implies that $f$ does not include $X_{i}$.
Therefore  $\frac{\partial f/\partial X^{d}_{i}}{\partial f/\partial X^{d}_{j}}$ is well-defined, and hence we have 
\begin{align}
    \frac{\partial}{\partial Q} \left(\frac{\partial f/\partial X^{d}_{i}}{\partial f/\partial X^{d}_{j}}\right) = 0.
\end{align}
Thus $f$ must be a separable function due to Theorem \ref{thorem_2_GU}, which complete the proof for the necessity.



\section{Questions to the proof}

\subsection{Claim 1}

Question is if Lau correctly uses Lemma \ref{lemma_1_GU} to show the exsistence of $F$.
First $F$ should be one-variable function in  \ref{lemma_1_GU}, although he defines $F$ as two-variable function.

Based on \eqref{eq:derivative_q_x_s}, it seems that \citet{lau1982identifying} defines
\begin{align}
    \lambda(Q, X^{s}) \equiv \frac{(1+\theta)\frac{\partial f}{\partial Q} + \frac{\partial^2 f}{\partial Q^2}Q - \frac{\partial g}{\partial Q}}{(1+\theta')\frac{\partial f}{\partial Q} + \frac{\partial^2 f}{\partial Q^2}Q - \frac{\partial g'}{\partial Q}}.
\end{align}
This expression clearly states that $\lambda$ depends on both $Q$ and $X^{s}$.
Furthermore, $g$ also depends on both $Q$ and $X^{s}$.
While $Q$ is a scalar, $X^{s}$ could be a vector, and hence we can think that he devide $K+1$ variables into two groups $Q$ and $X^{S}$.
What he claims based on Lemma \ref{lemma_1_GU} is that there exists $F$ such that
\begin{align}
    g(Q,X^{s}) = F(g'(Q,X^{s})).
\end{align}
However to show this, he only uses \eqref{eq:transformation_g}, although Lemma \ref{lemma_1_GU} also requires 
\begin{align}
    \frac{\partial g}{\partial Q} = \lambda(Q,X^{s}) \frac{\partial g'}{\partial Q}
\end{align}
becuase $g$ depends both $Q$ and $X^{s}$.

What can we obtain about $\frac{\partial g}{\partial Q}$ and $\frac{\partial g'}{\partial Q}$?
From the first-order condition, we have
\begin{align}
    0 &= (1+\theta)\frac{\partial f}{\partial Q} + \frac{\partial^2 f}{\partial Q^2}Q - \frac{\partial g}{\partial Q},\\
    0 &= (1+\theta')\frac{\partial f}{\partial Q} + \frac{\partial^2 f}{\partial Q^2}Q - \frac{\partial g'}{\partial Q},
\end{align}
which implies that 
\begin{align}
    \frac{\partial g}{\partial Q} = \frac{\partial g'}{\partial Q} + (\theta - \theta') \left(\frac{\partial f}{\partial Q} + \frac{\partial^2 f}{\partial Q^2}Q \right).
\end{align}
The second term and the derivatie of $g$ with respect to $Q$ is not multicable separable and the second term does not coincide with $\lambda$, and hence we cannot have $\frac{\partial g}{\partial Q} = \lambda(Q,X^{s}) \frac{\partial g'}{\partial Q}$.
Therefore, we cannot claim that $F$ exists.

\begin{remark}
    A possible interpretation is that he fixs $Q$ and thinks that $g$ is a function of $X^{s}$ only to obtain $F$. But this looks very tricky...
\end{remark}



\subsection{Other questions}
First, note that the left-hand side of \eqref{eq:foc_derivative_demand} and \eqref{eq:transform_theta_derivative_first} are both the derived of the left-hand side of \eqref{eq:foc_alpha} with respect to $X^{d}$.
However, the former allows the differentiation of $Q = h(X^{d}, X^{s})$ with respect to $X^{d}$, the latter does not.
In other words, the former views $f(Q, X^{d})$ as $f(h(X^{d}, X^{s}), X^{d})$ but the latter assumes that $Q$ does not depend on $X^{d}$.
When we allow the differentiation of $Q$ with respect to $X^{d}$, \eqref{eq:ratio_foc} becomes
\begin{align}
    \frac{\frac{\partial f}{\partial X^{d}_{i}} + \frac{\partial f}{\partial Q} \frac{\partial Q}{\partial X^{d}_{i}}}{\frac{\partial f}{\partial X^{d}_{j}} + \frac{\partial f}{\partial Q} \frac{\partial Q}{\partial X^{d}_{j}}} = \frac{\frac{\partial f}{\partial Q}\frac{\partial Q}{\partial X^{d}_{i}}  + \frac{\partial^2 f}{\partial Q^2}Q\frac{\partial Q}{\partial X^{d}_{i}} + \frac{\partial^2 f}{\partial X^{d}_{i}\partial Q}Q }{\frac{\partial f}{\partial Q}\frac{\partial Q}{\partial X^{d}_{j}}  + \frac{\partial^2 f}{\partial Q^2}Q\frac{\partial Q}{\partial X^{d}_{j}} + \frac{\partial^2 f}{\partial X^{d}_{j}\partial Q}Q }.
\end{align}

\section{A counterexample}

We show that the conduct parameter is identified when the inverse demand function is separable. 
Consider the following inverse demand function and marginal cost function:
\begin{align}
    P & = \exp(\varepsilon_{d}) Q^{\alpha_0} X_{d1}^{\alpha_1}X_{d2}^{\alpha_2}\label{eq:counter_demand}\\
    MC & = \exp(\varepsilon_{s})Q^{\beta_0} X_{s1}^{\beta_1} X_{s2}^{\beta_2}.\label{eq:counter_mc}
\end{align}
Assume that $\alpha_0 <0$.
The above model does not have an intercept in both the demand function and the marginal cost function. 

The inverse demand function is separable because
\begin{align}
    \frac{\partial }{\partial Q} \left(\frac{\partial P/\partial X_{d1}}{\partial P/\partial X_{d2}} \right) = \frac{\partial }{\partial Q} \left(\frac{\alpha_{1}\exp(\varepsilon_{d}) Q^{-\alpha_0} X_{d1}^{\alpha_1-1}X_{d2}^{\alpha_2}}{\alpha_2\exp(\varepsilon_{d}) Q^{-\alpha_0} X_{d1}^{\alpha_1}X_{d2}^{\alpha_2-1}} \right) =  \frac{\partial }{\partial Q}\left(\frac{\alpha_1}{\alpha_2} \frac{X_{d2}}{X_{d1}} \right)=0.
\end{align}
Thus Theorem \ref{theorem_lau} implies that the conduct parameter cannot be identified.
By taking logarithm to \eqref{eq:counter_demand}, we have a log-linear demand equation such that 
\begin{align}
    \log P = \alpha_0 \log Q + \alpha_1 \log X_{d1}  + \alpha_2 \log X_{d2} + \varepsilon_{d}.\label{eq:counter_demand_equation}
\end{align}
From the first-order condition, 
\begin{align}
    MC = Q^{\beta_0} X_{s1}^{\beta_1}X_{s2}^{\beta_2}\exp(\varepsilon_{s}) & = P + \theta (\alpha_0 \exp(\varepsilon_{d})Q^{\alpha_0-1}X_{d1}^{\alpha_1}X_{d2}^{\alpha_2}) Q\\
    & = P + \theta \alpha_0 P\\
    &= (1 + \theta\alpha_0) P.
\end{align}
By taking a logarithm, we obtain a log-linear supply equation,
\begin{align}
    \log P = - \log(1 + \theta\alpha_0) + \beta_0 \log Q + \beta_1 \log X_{s1}+\beta_2 \log X_{s2} + \varepsilon_{s}.\label{eq:counter_supply_equation}
\end{align}
By solving \eqref{eq:counter_demand_equation} and \eqref{eq:counter_supply_equation}, the equilibrium quantity $Q$ is obtained as
\begin{align}
    \log Q = \frac{\beta_1 \log X_{s1}+\beta_2 \log X_{s2} - \log(1 + \theta\alpha_0)+ \varepsilon_{s} - \alpha_1 \log X_{d1}  - \alpha_2 \log X_{d2} - \varepsilon_{d} }{\alpha_0 - \beta_0}.
\end{align}

Note that the demand parameters can be identified when $X^s$ is a vector of exclusive demand instruments.
Thus we can assume that $\alpha_0, \alpha_1$, and $\alpha_2$ is known.  

Then it is easy to see that $- \log(1 + \theta\alpha_0), \beta_0$, and $\beta_1$ are identified with a vector of exclusive supply instrument $X^d$.
Let $\gamma = - \log(1 + \theta\alpha_0)$, then $\theta = (\exp(-\gamma) - 1)/\alpha_0$.
Because $\alpha_0$ is identified, the parameter $\theta$ is also identified, which contradicts Theorem \ref{theorem_lau}.

\begin{remark}
    The above inverse demand function has a constant-elasticity property. 
    Note that $\alpha_0 \ne \frac{1}{\theta}$ holds.
    Thus it is not the separable function that allows identification in Theorem \ref{theorem_lau}.
    However, \citet{lau1982identifying} states that under this functional form, the conduct parameter cannot be identified in his comment (2) in p 98.
\end{remark}

\begin{remark}
    When the demand equation and the marginal cost equation have an intercept, the identification is impossible because the constant term in \eqref{eq:counter_supply_equation} becomes $-\log(1+\theta \alpha_0) + \beta$ and $\beta$ and $\theta$ cannot be separately identified without additional variables such as demand rotation parameter.
\end{remark}

\section{Alternative identification result}

Suppose that there are $\theta$,$\theta'$, $g$, and $g'$ that satisfy Definition \ref{def:non_identification}.
From \eqref{eq:foc_alpha} and \eqref{eq:foc_beta},
\begin{align}
    0 = (\theta - \theta')\frac{\partial f}{\partial Q}(Q,X^{d})Q + g'(Q,X^{s}) - g(Q,X^{s}). 
\end{align}
%線形の場合に,この式をダイレクトに使って,線型独立を証明しているのがMO2023
%この式自体,g' = (\theta - \theta') \partial f/\partial Q + gと変換した場合,
% g'はgのtransformationになっていることを示していそう.つまり,g' = F(g)となるようなFが存在する. 
%やっぱこれは無理そう.あるものの変換であるとは,g(a) = (b) となるa,bに対して,g'(a) = g'(b)であることを要求するが,\partial f/\partial Q (Q, a) = \partial f/\partial Q (Q, b)が成立するとは限らない. 


Differentiating this with $Q$,
\begin{align}
    0 = (\theta - \theta')\left[ \frac{\partial^2 f(Q,X^{d})}{\partial^2 Q}Q + \frac{\partial f(Q,X^{d})}{\partial Q} \right] + \frac{\partial g'(Q,X^{s})}{\partial Q} - \frac{\partial g(Q,X^{s})}{\partial Q}.\label{eq:foc_with_Q}
\end{align}
Derivative with $X^{d}$,
\begin{align}
    0 & =  (\theta - \theta')\left[ \frac{\partial^2 f(Q,X^{d})}{\partial^2 Q}\frac{\partial Q}{\partial X^{d}} Q  + \frac{\partial^2 f(Q,X^{d})}{\partial X^{d}\partial Q}Q + \frac{\partial f(Q,X^{d})}{\partial Q} \frac{\partial Q}{\partial X^{d}}\right]\\
    &\quad + \frac{\partial g'(Q,X^{s})}{\partial Q}\frac{\partial Q}{\partial X^{d}} - \frac{\partial g(Q,X^{s})}{\partial Q}\frac{\partial Q}{\partial X^{d}}\\
    & = \left[(\theta - \theta')\left[ \frac{\partial^2 f(Q,X^{d})}{\partial^2 Q}Q + \frac{\partial f(Q,X^{d})}{\partial Q} \right] + \frac{\partial g'(Q,X^{s})}{\partial Q} - \frac{\partial g(Q,X^{s})}{\partial Q}\right] \frac{\partial Q}{\partial X^{d}}\\
    &\quad + (\theta - \theta')\frac{\partial^2 f(Q,X^{d})}{\partial X^{d}\partial Q}Q \\
    & = (\theta - \theta')\frac{\partial^2 f(Q,X^{d})}{\partial X^{d}\partial Q}Q.\label{eq:foc_with_X_d}
\end{align}
Derivative with $X^{s}$,
\begin{align}
    0 = & (\theta - \theta')\left[ \frac{\partial^2 f(Q,X^{d})}{\partial^2 Q}\frac{\partial Q}{\partial X^{s}} Q  + \frac{\partial f(Q,X^{d})}{\partial Q} \frac{\partial Q}{\partial X^{s}}\right]\\
    &\quad + \frac{\partial g'(Q,X^{s})}{\partial Q}\frac{\partial Q}{\partial X^{s}} + \frac{\partial g'(Q,X^{s})}{\partial X^{s}} - \frac{\partial g(Q,X^{s})}{\partial Q}\frac{\partial Q}{\partial X^{s}} - \frac{\partial g(Q,X^{s})}{\partial X^{s}}\\
    & = \left[(\theta - \theta')\left[ \frac{\partial^2 f(Q,X^{d})}{\partial^2 Q}Q + \frac{\partial f(Q,X^{d})}{\partial Q} \right] + \frac{\partial g'(Q,X^{s})}{\partial Q} - \frac{\partial g(Q,X^{s})}{\partial Q}\right] \frac{\partial Q}{\partial X^{s}}\\
    & \quad +  \frac{\partial g'(Q,X^{s})}{\partial X^{s}} -  \frac{\partial g(Q,X^{s})}{\partial X^{s}}\\
    & = \frac{\partial g'(Q,X^{s})}{\partial X^{s}} -  \frac{\partial g(Q,X^{s})}{\partial X^{s}}.\label{eq:foc_with_X_s}
\end{align}

Because $\theta \ne \theta'$, \eqref{eq:foc_with_X_d} implies that 
\begin{align}
    \frac{\partial^2 f(Q,X^{d})}{\partial X^{d}\partial Q}Q = 0.
\end{align}
When $Q\ne 0$, this implies that $f(Q, X^{d}) = h(Q) + k(X^{d})$ for some $h$ and $k$. 
As this is the necessary condition for non-identification, we can identify the conduct parameter when this does not hold.
For example, a linear demand function $P = \alpha_0 + \alpha_1 Q + \alpha_2 X$ satisfies the functional form and we cannot identify $\theta$.
In contrast, $P = \alpha_0 + (\alpha_1 +\alpha_z Z) Q + \alpha_2 Y$ with $X = (Z,Y)$ and $P = \exp(\alpha_0)Q^{\alpha_1}X^{\alpha2}$ do not satisfy the functional form.
\citet{bresnahan1982oligopoly} and the above counterexample show that the conduct parameter can be identified under the two demand functions.

\eqref{eq:foc_with_X_s} implies $g$ and $g'$ have same slope with respect to $X^{s}$.
\textcolor{red}{I don't know what \eqref{eq:foc_with_Q} means.}
%少なくとも,gとg'がX^sに関して並行である場合には識別できないことになる.
%線形の場合には切片があって,切片が異なるgとg'のX^sの微分系数が同じだと,識別できていないことになる.

% そもそも均衡アウトカムのQに関する微分ってとれるのか?線形の時は弾力性を出すのにQに関しての微分を取っているが...外生変数の変化なしに,内生変数の変化を考慮するのはOK?
%gのQに関する微分については第1項の分だけシフトしていることになる.ただ,Lau自体もQに関する微分をそのまま識別の条件として使っているわけではないし,Qが均衡アウトカムであることを考えると,Qの変化もdemand shifter やcost shifterによって起きるものなので,X^dとX^sの微分だけを考えればよいのか?
\newpage
\bibliographystyle{aer}
\bibliography{conduct_parameter}

\end{document}